\documentclass{article}
\usepackage{settings}

\geometry{
a4paper,
total={140mm,257mm},
left=35mm,
top=20mm,
}

\title{Physics}
\author{Samuel Lindskog}

\begin{document}
\maketitle
\addtocontents{toc}{\protect\hypertarget{toc}{}}
\tableofcontents
\pagenumbering{gobble}
\clearpage
\pagenumbering{arabic}
\setcounter{page}{1}

\section{Periodic Motion}
\subsection{Frequency}
\begin{definition}[Period]
	The period \(T\) is the time to complete one cycle, measured in seconds.
\end{definition}
\begin{definition}[Angular Frequency]
	Angular frequency \(\omega\) is radians per second, with units \(\text{s}^{-1}\). Radians are unitless.
	\begin{equation*}
		\omega=\frac{2\pi}{T}.
	\end{equation*}
\end{definition}
\begin{definition}[Frequency]
	Frequency is a measure of cycles per second, or \(\omega/2\pi\). It is measured in hertz (Hz) with units \(\text{s}^{-1}\). Cycles are unitless.
\end{definition}
\begin{definition}[Restoring force]
	The restoring force \(F_x\) relative to displacement from equlibrium \(x\) is
	\begin{equation*}
		F_x=-kx.
	\end{equation*}
	\(k\) is the spring constant with units \(\text{N}/\text{m}\).
\end{definition}
\begin{definition}
	The equation used to derive the equation for simple harmonic motion with respect to time is
	\begin{equation*}
		\frac{d^2x}{dt^2}=-\frac{k}{m}x.
	\end{equation*}
	Eigenvalues for this equation are \(\pm i\sqrt{\frac{k}{m}}\), so displacement from equilibrium for simple harmonic motion with respect to time is
	\begin{equation*}
		x=c_1\text{cos}\bigg(\sqrt{\frac{k}{m}t}\bigg)+c_2\text{sin}\bigg(\sqrt{\frac{k}{m}t}\bigg).
	\end{equation*}
	In phase-amplitude form this equation is
	\begin{equation*}
		x=A\text{cos}(\omega t+\phi).
	\end{equation*}
\end{definition}
\begin{definition}[SHM frequency]
	It follows from the displacement equation for SHM that angular frequency for SHM is
	\begin{equation*}
		\omega=\sqrt{\frac{k}{m}}.
	\end{equation*}
\end{definition}
\subsection{Energy in SHM}
\begin{definition}[Mechanical energy]
	Work in Joules (Nm) is equal to displacement times force in simple scenarios. For forces that vary with one-dimensional displacement \(x\) from \(a\) to \(b\), work is expressed by equation
	\begin{equation*}
		E=\int_a^b F(x)dx.
	\end{equation*}
\end{definition}
\end{document}
