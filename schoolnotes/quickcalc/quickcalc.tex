\documentclass{article}
\usepackage{settings}

\geometry{
a4paper,
total={140mm,257mm},
left=35mm,
top=20mm,
}

\title{"Calculus: Early Transcendantals" Notes}
\author{Samuel Lindskog}

\begin{document}
\maketitle
\addtocontents{toc}{\protect\hypertarget{toc}{}}
\tableofcontents
\pagenumbering{gobble}
\clearpage
\pagenumbering{arabic}
\setcounter{page}{1}

\section{Section 1}
\subsection{Subsection 1}
\begin{theorem}[Fubini's theorem]
	Let \(f(x,y)\) be continuous on a region \(R\).
	\begin{enumerate}
		\item If \(R\) is defined by \(a\leq x\leq b,\;g_1(x)\leq y\leq g_2(x)\), with \(g_1\) and \(g_2\) continuous on \([a,b]\), then
			\begin{equation*}
				\iint_R f(x,y)dA=\int_a^b\int_{g_1(x)}^{g_2(x)}f(x,y)dydx.
			\end{equation*}
	\end{enumerate}
\end{theorem}
\begin{proposition}
	The area of a closed, bounded plane region \(R\) is
	\begin{equation*}
		A=\iint_R dA.
	\end{equation*}
\end{proposition}
\begin{definition}[Path independence]
	Let \(\textbf{F}\) be a vector field defined on an open region \(D\) in space, and suppose that for any two points \(A\) and \(B\) in \(D\) the line integral \(\int_C \textbf{F}\cdot d\textbf{r}\) along a path \(C\) from \(A\) to \(B\) in \(D\) is the same over all paths from \(A\) to \(B\). Then the integral \(\int_C\textbf{F}\cdot d\textbf{r}\) is path independent in \(D\) and the field \(\textbf{F}\) is conservative in \(D\).
\end{definition}
\begin{theorem}[Fundamental theorem of line integrals]
	Let \(C\) be a smooth curve joining the point \(A\) to the point \(B\) in the plane or in space and parametrized by \(\textbf{r}(t)\). Let \(f\) be a differentiable function with a continuous gradient vector \(\textbf{F}=\nabla f\) on a domain \(D\) containing \(C\). Then
	\begin{equation*}
		\int_C\textbf{F}\cdot d\textbf{r}=f(B)-f(A).
	\end{equation*}
\end{theorem}
\end{document}
