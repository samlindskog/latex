\documentclass{article}
\usepackage{settings}

\geometry{
a4paper,
total={140mm,257mm},
left=35mm,
top=20mm,
}

\title{"Calculus: Early Transcendantals" Notes}
\author{Samuel Lindskog}

\begin{document}
\maketitle
\addtocontents{toc}{\protect\hypertarget{toc}{}}
\tableofcontents
\pagenumbering{gobble}
\clearpage
\pagenumbering{arabic}
\setcounter{page}{1}
TODO: binomial theorem, cosh sinh, complete the square, difference of squares, partial fractions, 2.3 epsilon delta proofs
\section{Chapter 2: Limits and continuity}
\subsection{Rates of change and tangent lines to curves}
\begin{definition}[Average rate of change]
	The average rate of change of \(y=f(x)\) with respect to \(x\) over the interval \([x_1,x_2]\) is
	\begin{equation*}
		\frac{\Delta y}{\Delta x}=\frac{f(x_2)-f(x_1)}{x_2-x_1}=\frac{f(x_1+h)-f(x_1)}{h},\quad h\neq 0.
	\end{equation*}
\end{definition}
\begin{definition}[Secant line]
	A line joining two points of a curve is called a secant line.
\end{definition}
\begin{remark}
	The average rate of change of \(f\) from \(x_1\) to \(x_2\) is the slope of the secant line between these points.
\end{remark}
\subsection{Limit of a function and limit laws}
\begin{definition}[Limit]
	Let \(f(x)\) be defined on an open interval about \(c\). We say that the limit of \(f(x)\) as \(x\) approaches \(c\) is the number \(L\), and write
	\begin{equation*}
		\lim_{x\rightarrow c}f(x)=L,
	\end{equation*}
	if for every number \(\epsilon>0\) there exists \(\delta>0\) such that
	\begin{equation*}
		0<\abs{x-c}<\delta \Rightarrow \abs{f(x)-L}<\epsilon.
	\end{equation*}
\end{definition}
\begin{definition}[Limit laws]
	\label{limitlaws}
	If \(L,M,c,k\) are real numbers and \(\lim_{x\rightarrow c}f(x)=L\) and \(\lim_{x\rightarrow c}g(x)=M\), then
	\begin{IEEEeqnarray*}{ll}
		\text{Sum Rule:}&\lim_{x\rightarrow c}(f(x)+g(x))=L+M\\
		\text{Difference Rule:}&\lim_{x\rightarrow c}(f(x)-g(x))=L-M\\
		\text{Constant Multiple Rule:}\quad&\lim_{x\rightarrow c}(k\cdot f(x))=k\cdot L\\
		\text{Product Rule:}&\lim_{x\rightarrow c}(f(x)\cdot g(x))=L\cdot M\\
		\text{Quotient Rule:}&\lim_{x\rightarrow c}\frac{f(x)}{g(x)}=\frac{L}{M},\quad M\neq 0\\
		\text{Power Rule:}&\lim_{x\rightarrow c}[f(x)]^n=L^n,\quad n\in\mathbb{Q}^+
	\end{IEEEeqnarray*}
	If \(n\) is even, we assume that \(f(x)\geq 0\) for \(x\) in an interval containing \(c\).
\end{definition}
\begin{theorem}
	If \(P(x)\) is some polynomial \(a_n x^n+a_{n-1}x^{n-1}+\ldots+a_0\), then
	\begin{equation*}
		\lim_{x\rightarrow c}P(x)=P(c).
	\end{equation*}
\end{theorem}
\begin{theorem}
	If \(P(x)\) and \(Q(x)\) are polynomials and \(Q(c)\neq 0\), then
	\begin{equation*}
		\lim_{x\rightarrow c}\frac{P(x)}{Q(x)}=\frac{P(c)}{Q(c)}.
	\end{equation*}
\end{theorem}
\begin{theorem}[Sandwich theorem]
	Suppose that \(g(x)\leq f(x)\leq h(x)\) for all \(x\) in some open interval containing \(c\), except possibly at \(x=c\) itself. Suppose also that
	\begin{equation*}
		\lim_{x\rightarrow c}g(x)=\lim_{x\rightarrow c}h(x)=L.
	\end{equation*}
	Then \(\lim_{x\rightarrow c}f(x)=L\).
\end{theorem}
\subsection{One-sided limits}
\begin{definition}[Right limit]
	Assume the domain of \(f\) contains an interval \((c,d)\) to the right of \(c\). We say that \(f(x)\) has a right-handed limit \(L\) at \(c\) and write
	\begin{equation*}
		\lim_{x\rightarrow c^+}f(x)=L
	\end{equation*}
	if for all \(\epsilon>0\) there exists \(\delta>0\) such that
	\begin{equation*}
		c<x<c+\delta \abs{f(x-L)}<\epsilon.
	\end{equation*}
\end{definition}
\subsection{Continuity}
\begin{definition}[Continuity]
	Let \(c\) be a real number that is in the interval of the domain of a function \(f\). \(f\) is continuous at \(c\) if
	\begin{equation*}
		\lim_{x\rightarrow c}f(x)=f(c).
	\end{equation*}
	\(f\) is right-continuous at \(c\) if
	\begin{equation*}
		\lim_{x\rightarrow c^+}f(x)=f(c.)
	\end{equation*}
	\(f\) is left-continuous at \(c\) if
	\begin{equation*}
		\lim_{x\rightarrow c^-}f(x)=f(c).
	\end{equation*}
\end{definition}
\begin{remark}
	If a function is not continuous at a point \(c\) of its domain, we say that \(f\) is discontinuous at \(c\), and that \(f\) has a discontinuity at \(c\).
\end{remark}
\begin{proposition}[Continuity test]
	A function \(f(x)\) is continuous at a point \(x=c\) iff it meets the following three conditions:
	\begin{enumerate}
		\item \(f(c)\) exists.
		\item \(\lim_{x\rightarrow c}f(x)\) exists.
		\item \(\lim_{x\rightarrow c}f(x)=f(c)\).
	\end{enumerate}
\end{proposition}
\begin{definition}[Continuous function]
	A function is continous if it is continous at every point in its domain.
\end{definition}
\begin{theorem}
	If the function \(f\) and \(g\) are continous at \(x=c\), then the following algebraic combination are continous at \(x=c\).
	\begin{IEEEeqnarray*}{l}
		f+g\\
		f-g\\
		k\cdot f,\; k\in\mathbb{R}\\
		f\cdot g\\
		f/g,\; g(c)\neq 0\\
		f^n,\; n\in\mathbb{N}^+\\
		f^{1/n},\;\text{ if defined on an interval containing } c.
	\end{IEEEeqnarray*}
\end{theorem}
\begin{theorem}
	If \(\lim_{x\rightarrow c}f(x)=b\) and \(g\) is continous at the point \(b\), then
	\begin{equation*}
		\lim_{x\rightarrow c}g(f(x))=g(b).
	\end{equation*}
\end{theorem}
\begin{theorem}[Intermediate value theorem for continuous functions]
	If \(f\) is a continous function on a closed interval \([a,b]\), and if \(y_0\) is any value between \(f(a)\) and \(f(b)\), then \(y_0=f(c)\) for some \(c\in[a,b]\).
\end{theorem}
\subsection{Limits involving infinity}
\begin{definition}[Limits approaching infinity]
	We say that \(f(x)\) has the limit \(L\) as \(x\) approaches infinity and write
	\begin{equation*}
		\lim_{x\rightarrow\infty}f(x)=L
	\end{equation*}
	if, for all \(\epsilon>0\) there exists \(M\) such that for all \(x\) in the domain of \(f\)
	\begin{equation*}
		x>M\Rightarrow \abs{f(x)-L}<\epsilon
	\end{equation*}
\end{definition}
\begin{theorem}
	Theorem \ref{limitlaws} applies to limits that appraoch infinity.
\end{theorem}
\begin{definition}[Horizontal asymptote]
	A line \(y=b\) is a horizontal asymptote of a graph of a function \(y=f(x)\) if either
	\begin{equation*}
		\lim_{x\rightarrow\infty}f(x)=b\text{ or }\lim_{x\rightarrow -\infty}f(x)=b.
	\end{equation*}
\end{definition}
\begin{definition}[Infinite limits]
	We say that \(f(x)\) approaches infinity as \(x\) approaches \(c\) and write
	\begin{equation*}
		\lim_{x\rightarrow c}=\infty,
	\end{equation*}
	if for every number \(B>0\) there exists \(\delta>0\) such that
	\begin{equation*}
		0<\abs{x-c}<\delta\Rightarrow f(x)>B.
	\end{equation*}
\end{definition}
\begin{definition}
	A line \(x=a\) is a vertical asymptote of the graph of a function \(y=f(x)\) if either
	\begin{equation*}
		\lim_{x\rightarrow a^+}f(x)=\pm\infty\text{ or }\lim_{x\rightarrow a^-}f(x)=\pm\infty.
	\end{equation*}
\end{definition}
\section{Chapter 3: Derivatives}
\subsection{Tangent lines and the derivative}
\begin{definition}[Tangent line]
	The slope of the curve \(y=f(x)\) at the point \(P(x_0,f(x_0))\) is the number
	\begin{equation*}
		\lim_{h\rightarrow 0}\frac{f(x_0+h)-f(x_0)}{h},
	\end{equation*}
	provided the limit exists. The tangent line to the curve at \(P\) is the line through \(P\) with this slope.
\end{definition}
\begin{definition}[Derivative]
	The derivative of a function \(f\) at a point \(x_0\), denoted \(f'(x_0)\), is
	\begin{equation*}
		f'(x_0)=\lim_{h\rightarrow 0}\frac{f(x_0+h)-f(x_0)}{h},
	\end{equation*}
	provided this limit exists.
\end{definition}
\begin{definition}[Differentiability]
	A function \(y=f(x)\) is differentiable on an open interval if it has a derivative at each point of the interval. It is differentiable on a closed interval \([a,b]\) if it is differentiable on the interior \((a,b)\) and if the right/left hand limits of the derivative exist on the left/right side of the interval respectively.
\end{definition}
\begin{theorem}
	If \(f\) has a derivative at \(x=c\), then \(f\) is continuous at \(x=c\).
\end{theorem}
\section{Chapter 10: Parametric equations}
\subsection{Parametrizations of plane curves}
\begin{definition}[Parametric curve]
	If \(x\) and \(y\) are given as functions of \(t\)
	\begin{equation*}
		x=f(t),\quad y=g(t),
	\end{equation*}
	over an interval \(I\) of \(t\)-values, then the set of points \((x,y)=(f(t),g(t))\) is a parametric curve. These equations are called parametric equations for the curve.
\end{definition}
\begin{remark}
	The variable \(t\) is a parameter for the curve, and its domain \(I\) is the parameter interval. When we give parametric equations for a curve, we say that we have parameterized the curve. The equations and interval together constitute a parametrization of the curve.
\end{remark}
\subsection{Calculus with parametric curves}
\begin{remark}
A parametrized curve \(x=f(t)\) and \(y=g(t)\) is differentiable at \(t\) if \(f\) and \(g\) are differentiable at \(t\).
\end{remark}
\section{Multiple integrals}
\subsection{Double integrals over rectangles}
\begin{theorem}[Fubini's theorem]
	Let \(f(x,y)\) be continuous on a region \(R\).
	\begin{enumerate}
		\item If \(R\) is defined by \(a\leq x\leq b,\;g_1(x)\leq y\leq g_2(x)\), with \(g_1\) and \(g_2\) continuous on \([a,b]\), then
			\begin{equation*}
				\iint_R f(x,y)dA=\int_a^b\int_{g_1(x)}^{g_2(x)}f(x,y)dydx.
			\end{equation*}
	\end{enumerate}
\end{theorem}
\begin{proposition}
	The area of a closed, bounded plane region \(R\) is
	\begin{equation*}
		A=\iint_R dA.
	\end{equation*}
\end{proposition}
\section{Line integrals}
\subsection{Line integrals}
\begin{definition}[Path independence]
	Let \(\textbf{F}\) be a vector field defined on an open region \(D\) in space, and suppose that for any two points \(A\) and \(B\) in \(D\) the line integral \(\int_C \textbf{F}\cdot d\textbf{r}\) along a path \(C\) from \(A\) to \(B\) in \(D\) is the same over all paths from \(A\) to \(B\). Then the integral \(\int_C\textbf{F}\cdot d\textbf{r}\) is path independent in \(D\) and the field \(\textbf{F}\) is conservative in \(D\).
\end{definition}
\begin{theorem}[Fundamental theorem of line integrals]
	Let \(C\) be a smooth curve joining the point \(A\) to the point \(B\) in the plane or in space and parametrized by \(\textbf{r}(t)\). Let \(f\) be a differentiable function with a continuous gradient vector \(\textbf{F}=\nabla f\) on a domain \(D\) containing \(C\). Then
	\begin{equation*}
		\int_C\textbf{F}\cdot d\textbf{r}=f(B)-f(A).
	\end{equation*}
\end{theorem}
\end{document}
