\documentclass{article}
\usepackage{settings}

\geometry{
a4paper,
total={140mm,257mm},
left=35mm,
top=20mm,
}

\title{Notes}
\author{Samuel Lindskog}

\begin{document}
\maketitle
\addtocontents{toc}{\protect\hypertarget{toc}{}}
\tableofcontents
\pagenumbering{gobble}
\clearpage
\pagenumbering{arabic}
\setcounter{page}{1}
TODO: binomial theorem, cosh sinh, complete the square, difference of squares, partial fractions, 2.3 epsilon delta proofs
\section{Chapter 2: Limits and continuity}
\subsection{Rates of change and tangent lines to curves}
\begin{definition}[Average rate of change]
	The average rate of change of \(y=f(x)\) with respect to \(x\) over the interval \([x_1,x_2]\) is
	\begin{equation*}
		\frac{\Delta y}{\Delta x}=\frac{f(x_2)-f(x_1)}{x_2-x_1}=\frac{f(x_1+h)-f(x_1)}{h},\quad h\neq 0.
	\end{equation*}
\end{definition}
\begin{definition}[Secant line]
	A line joining two points of a curve is called a secant line.
\end{definition}
\begin{remark}
	The average rate of change of \(f\) from \(x_1\) to \(x_2\) is the slope of the secant line between these points.
\end{remark}
\subsection{Limit of a function and limit laws}
\begin{definition}[Limit]
	Let \(f(x)\) be defined on an open interval about \(c\). We say that the limit of \(f(x)\) as \(x\) approaches \(c\) is the number \(L\), and write
	\begin{equation*}
		\lim_{x\rightarrow c}f(x)=L,
	\end{equation*}
	if for every number \(\epsilon>0\) there exists \(\delta>0\) such that
	\begin{equation*}
		0<\abs{x-c}<\delta \Rightarrow \abs{f(x)-L}<\epsilon.
	\end{equation*}
\end{definition}
\begin{definition}[Limit laws]
	\label{limitlaws}
	If \(L,M,c,k\) are real numbers and \(\lim_{x\rightarrow c}f(x)=L\) and \(\lim_{x\rightarrow c}g(x)=M\), then
	\begin{IEEEeqnarray*}{ll}
		\text{Sum Rule:}&\lim_{x\rightarrow c}(f(x)+g(x))=L+M\\
		\text{Difference Rule:}&\lim_{x\rightarrow c}(f(x)-g(x))=L-M\\
		\text{Constant Multiple Rule:}\quad&\lim_{x\rightarrow c}(k\cdot f(x))=k\cdot L\\
		\text{Product Rule:}&\lim_{x\rightarrow c}(f(x)\cdot g(x))=L\cdot M\\
		\text{Quotient Rule:}&\lim_{x\rightarrow c}\frac{f(x)}{g(x)}=\frac{L}{M},\quad M\neq 0\\
		\text{Power Rule:}&\lim_{x\rightarrow c}[f(x)]^n=L^n,\quad n\in\mathbb{Q}^+
	\end{IEEEeqnarray*}
	If \(n\) is even, we assume that \(f(x)\geq 0\) for \(x\) in an interval containing \(c\).
\end{definition}
\begin{theorem}
	If \(P(x)\) is some polynomial \(a_n x^n+a_{n-1}x^{n-1}+\ldots+a_0\), then
	\begin{equation*}
		\lim_{x\rightarrow c}P(x)=P(c).
	\end{equation*}
\end{theorem}
\begin{theorem}
	If \(P(x)\) and \(Q(x)\) are polynomials and \(Q(c)\neq 0\), then
	\begin{equation*}
		\lim_{x\rightarrow c}\frac{P(x)}{Q(x)}=\frac{P(c)}{Q(c)}.
	\end{equation*}
\end{theorem}
\begin{theorem}[Sandwich theorem]
	Suppose that \(g(x)\leq f(x)\leq h(x)\) for all \(x\) in some open interval containing \(c\), except possibly at \(x=c\) itself. Suppose also that
	\begin{equation*}
		\lim_{x\rightarrow c}g(x)=\lim_{x\rightarrow c}h(x)=L.
	\end{equation*}
	Then \(\lim_{x\rightarrow c}f(x)=L\).
\end{theorem}
\subsection{One-sided limits}
\begin{definition}[Right limit]
	Assume the domain of \(f\) contains an interval \((c,d)\) to the right of \(c\). We say that \(f(x)\) has a right-handed limit \(L\) at \(c\) and write
	\begin{equation*}
		\lim_{x\rightarrow c^+}f(x)=L
	\end{equation*}
	if for all \(\epsilon>0\) there exists \(\delta>0\) such that
	\begin{equation*}
		c<x<c+\delta \abs{f(x-L)}<\epsilon.
	\end{equation*}
\end{definition}
\subsection{Continuity}
\begin{definition}[Continuity]
	Let \(c\) be a real number that is in the interval of the domain of a function \(f\). \(f\) is continuous at \(c\) if
	\begin{equation*}
		\lim_{x\rightarrow c}f(x)=f(c).
	\end{equation*}
	\(f\) is right-continuous at \(c\) if
	\begin{equation*}
		\lim_{x\rightarrow c^+}f(x)=f(c.)
	\end{equation*}
	\(f\) is left-continuous at \(c\) if
	\begin{equation*}
		\lim_{x\rightarrow c^-}f(x)=f(c).
	\end{equation*}
\end{definition}
\begin{remark}
	If a function is not continuous at a point \(c\) of its domain, we say that \(f\) is discontinuous at \(c\), and that \(f\) has a discontinuity at \(c\).
\end{remark}
\begin{proposition}[Continuity test]
	A function \(f(x)\) is continuous at a point \(x=c\) iff it meets the following three conditions:
	\begin{enumerate}
		\item \(f(c)\) exists.
		\item \(\lim_{x\rightarrow c}f(x)\) exists.
		\item \(\lim_{x\rightarrow c}f(x)=f(c)\).
	\end{enumerate}
\end{proposition}
\begin{definition}[Continuous function]
	A function is continous if it is continous at every point in its domain.
\end{definition}
\begin{theorem}
	If the function \(f\) and \(g\) are continous at \(x=c\), then the following algebraic combination are continous at \(x=c\).
	\begin{IEEEeqnarray*}{l}
		f+g\\
		f-g\\
		k\cdot f,\; k\in\mathbb{R}\\
		f\cdot g\\
		f/g,\; g(c)\neq 0\\
		f^n,\; n\in\mathbb{N}^+\\
		f^{1/n},\;\text{ if defined on an interval containing } c.
	\end{IEEEeqnarray*}
\end{theorem}
\begin{theorem}
	If \(\lim_{x\rightarrow c}f(x)=b\) and \(g\) is continous at the point \(b\), then
	\begin{equation*}
		\lim_{x\rightarrow c}g(f(x))=g(b).
	\end{equation*}
\end{theorem}
\begin{theorem}[Intermediate value theorem for continuous functions]
	If \(f\) is a continous function on a closed interval \([a,b]\), and if \(y_0\) is any value between \(f(a)\) and \(f(b)\), then \(y_0=f(c)\) for some \(c\in[a,b]\).
\end{theorem}
\subsection{Limits involving infinity}
\begin{definition}[Limits approaching infinity]
	We say that \(f(x)\) has the limit \(L\) as \(x\) approaches infinity and write
	\begin{equation*}
		\lim_{x\rightarrow\infty}f(x)=L
	\end{equation*}
	if, for all \(\epsilon>0\) there exists \(M\) such that for all \(x\) in the domain of \(f\)
	\begin{equation*}
		x>M\Rightarrow \abs{f(x)-L}<\epsilon
	\end{equation*}
\end{definition}
\begin{theorem}
	Theorem \ref{limitlaws} applies to limits that appraoch infinity.
\end{theorem}
\begin{definition}[Horizontal asymptote]
	A line \(y=b\) is a horizontal asymptote of a graph of a function \(y=f(x)\) if either
	\begin{equation*}
		\lim_{x\rightarrow\infty}f(x)=b\text{ or }\lim_{x\rightarrow -\infty}f(x)=b.
	\end{equation*}
\end{definition}
\begin{definition}[Infinite limits]
	We say that \(f(x)\) approaches infinity as \(x\) approaches \(c\) and write
	\begin{equation*}
		\lim_{x\rightarrow c}=\infty,
	\end{equation*}
	if for every number \(B>0\) there exists \(\delta>0\) such that
	\begin{equation*}
		0<\abs{x-c}<\delta\Rightarrow f(x)>B.
	\end{equation*}
\end{definition}
\begin{definition}
	A line \(x=a\) is a vertical asymptote of the graph of a function \(y=f(x)\) if either
	\begin{equation*}
		\lim_{x\rightarrow a^+}f(x)=\pm\infty\text{ or }\lim_{x\rightarrow a^-}f(x)=\pm\infty.
	\end{equation*}
\end{definition}
\section{Chapter 3: Derivatives}
\subsection{Tangent lines and the derivative}
\begin{definition}[Tangent line]
	The slope of the curve \(y=f(x)\) at the point \(P(x_0,f(x_0))\) is the number
	\begin{equation*}
		\lim_{h\rightarrow 0}\frac{f(x_0+h)-f(x_0)}{h},
	\end{equation*}
	provided the limit exists. The tangent line to the curve at \(P\) is the line through \(P\) with this slope.
\end{definition}
\begin{definition}[Derivative]
	The derivative of a function \(f\) at a point \(x_0\), denoted \(f'(x_0)\), is
	\begin{equation*}
		f'(x_0)=\lim_{h\rightarrow 0}\frac{f(x_0+h)-f(x_0)}{h},
	\end{equation*}
	provided this limit exists.
\end{definition}
\begin{definition}[Differentiability]
	A function \(y=f(x)\) is differentiable on an open interval if it has a derivative at each point of the interval. It is differentiable on a closed interval \([a,b]\) if it is differentiable on the interior \((a,b)\) and if the right/left hand limits of the derivative exist on the left/right side of the interval respectively.
\end{definition}
\begin{theorem}
	If \(f\) has a derivative at \(x=c\), then \(f\) is continuous at \(x=c\).
\end{theorem}
\section{Chapter 10: Parametric equations}
\subsection{Parametrizations of plane curves}
\begin{definition}[Parametric curve]
	If \(x\) and \(y\) are given as functions of \(t\)
	\begin{equation*}
		x=f(t),\quad y=g(t),
	\end{equation*}
	over an interval \(I\) of \(t\)-values, then the set of points \((x,y)=(f(t),g(t))\) is a parametric curve. These equations are called parametric equations for the curve.
\end{definition}
\begin{remark}
	The variable \(t\) is a parameter for the curve, and its domain \(I\) is the parameter interval. When we give parametric equations for a curve, we say that we have parameterized the curve. The equations and interval together constitute a parametrization of the curve.
\end{remark}
\subsection{Calculus with parametric curves}
\begin{remark}
A parametrized curve \(x=f(t)\) and \(y=g(t)\) is differentiable at \(t\) if \(f\) and \(g\) are differentiable at \(t\).
\end{remark}
\section{Chapter 12: Vector-valued functions}
\subsection{Derivatives of vector functions}
\begin{definition}[Limit]
	Let \(r:\mathbb{R}\rightarrow \mathbb{R}^n\) be a function. We say that \(r\) has limit \(L\) as \(t\) approaches \(t_0\), denoted \(\lim_{t\rightarrow t_0}r(t)=L\), If
	\begin{equation*}
		\forall\epsilon>0,\,\exists\delta>0,\,\big(\abs{t-t_0}<\delta\Rightarrow\abs{r(t)-L}<\epsilon\big).
	\end{equation*}
\end{definition}
\begin{definition}[Continuity]
	A vector-valued function \(r(t)\) is continous at a point \(t=t_0\) in its domain if \(\lim_{t\rightarrow t_0}r(t)=r(t_0)\). The function is continuous if it is continuous at every point in it's domain.
\end{definition}
\begin{definition}
	The vector function \(r(t)=f(t)i+g(t)j+h(t)k\) has a derivative at \(t\) if \(f,g,h\) have derivatives at \(t\). The derivative is the vector function
	\begin{equation*}
		r'(t)=\diff{f}{t}i+\diff{g}{t}j+\diff{h}{t}k.
	\end{equation*}
\end{definition}
\begin{definition}[Velocity, acceleration]
	If \(r\) is the position vector, the \(\diff{r}{t}\) is the velocity vector and \(\diff[2]{r}{t}\) is the acceleration vector.
\end{definition}
\begin{definition}[Indefinite integral]
	The indefinite integral of \(r\) with respect to \(t\) is the set of all antiderivatives of \(r\). If \(R\) is any antiderivate of \(r\), then
	\begin{equation*}
		\int r(t)dt=R(t)+C.
	\end{equation*}
\end{definition}
\begin{definition}[Definite integral]
	If the components of \(r(t)=f(t)i+g(t)j+h(t)k\) are integrable over \([a,b]\), then so is \(r\), and the definite integral of \(r\) from \(a\) to \(b\) is
	\begin{equation*}
		\int_a^b r(t)dt=\bigg(\int_a^bf(t)dt\bigg)i+\bigg(\int_a^bg(t)dt\bigg)j+\bigg(\int_a^bh(t)dt\bigg)k.
	\end{equation*}
	\subsection{Arc length in space}
	\begin{definition}[Length]
		The length of a smooth curve \(r(t)=x(t)i+y(t)j+z(t)k\) for \(a\leq t\leq b\) as \(t\) increases on this interval is
		\begin{equation*}
			L=\int_a^b\sqrt{\bigg(\diff{x}{t}\bigg)^2+\bigg(\diff{y}{t}\bigg)^2+\bigg(\diff{z}{t}\bigg)^2}dt.
		\end{equation*}
		In other words, if \(v\) is the first derivative of \(r\),
		\begin{equation*}
			L=\int_a^b\abs{v}dt.
		\end{equation*}
	\end{definition}
\end{definition}
\subsection{Velocity and acceleration in polar coordinates}
\section{Chapter 13: Partial derivatives}
\subsection{Gradient vectors}
\begin{definition}[Directional derivative]
	The derivative of \(f\) at \(P_0(x_0,y_0)\) in the direction of the unit vector \(u=u_1i+u_2j\) is the number
	\begin{equation*}
		\bigg(\diff{f}{s}\bigg)_{u,P_0}=\lim_{s\rightarrow 0}\frac{f(x_0+su_1,y_0+su_2)-f(x_0,y_0)}{s},
	\end{equation*}
	provided the limit exists.
\end{definition}
\begin{definition}[Gradient vector]
	The gradient vector of \(f(x,y)\) is the vector
	\begin{equation*}
		\nabla f=\diffp{f}{x}i+\diffp{f}{y}j.
	\end{equation*}
\end{definition}
\begin{theorem}
	If \(f(x,y)\) is differentiable in an open region containing \(P_0(x_0,y_0)\), then
	\begin{equation*}
		\bigg(\diff{f}{s}\bigg)_{u,P_0}=\Eval{\nabla f}{P_0}{}\cdot u.
	\end{equation*}
\end{theorem}
\begin{proposition}
	The derivative along a path in the domain of \(f\), paramaterized by \(r\) is
	\begin{equation*}
		\diff{}{t}f(r(t))=\nabla f(r(t))\cdot r'(t).
	\end{equation*}
\end{proposition}
\section{Chapter 14: Multiple integrals}
\subsection{Double integrals over rectangles}
\begin{theorem}[Fubini's theorem]
	Let \(f(x,y)\) be continuous on a region \(R\).
	\begin{enumerate}
		\item If \(R\) is defined by \(a\leq x\leq b,\;g_1(x)\leq y\leq g_2(x)\), with \(g_1\) and \(g_2\) continuous on \([a,b]\), then
			\begin{equation*}
				\iint_R f(x,y)dA=\int_a^b\int_{g_1(x)}^{g_2(x)}f(x,y)dydx.
			\end{equation*}
	\end{enumerate}
\end{theorem}
\begin{proposition}
	The area of a closed, bounded plane region \(R\) is
	\begin{equation*}
		A=\iint_R dA.
	\end{equation*}
\end{proposition}
\section{Chapter 15: Integrals Vector fields}
\subsection{Line integrals of scalar functions}
\begin{definition}[Line integral]
	\label{directlineint}
	If \(f\) is defined on a curve \(C\) given parametrically by \(r(t)=g(t)i+h(t)j+k(t)k\) on \(a\leq t\leq b\), then the line integral of \(f\) over \(C\) is
	\begin{equation*}
		\int_C f(x,y,z)ds=\lim_{n\rightarrow\infty}\sum_{k=1}^{n}f(x_k,y_k,z_k)\Delta s_k,
	\end{equation*}
	and can be evaluated as
	\begin{equation*}
		\int_C f(x,y,z)ds=\int_a^{b}f(g(t),h(t),k(t))\abs{v(t)}dt.
	\end{equation*}
\end{definition}
\begin{proposition}
	If a piecewise smooth curve \(C\) is made by joining a finite number of smooth curves \(C_1,\ldots,C_n\) end to end, then the integral of a function over \(C\) is the sum of the integrals over the curves \(C_1,\ldots,C_n\).
\end{proposition}
\subsection{Vector fields and line integrals}
\begin{definition}[Vector field]
	A vector field is a function that assigns a vector to each point in it's domain.
\end{definition}
\begin{proposition}
	A vector field is continuous if it's component functions are continuous. It is differentiable if each of it's component functions are differentiable.
\end{proposition}
\begin{definition}[Line integral]
	\label{vectorlineint}
	Let \(F:\mathbb{R}^n\rightarrow\mathbb{R}^n\) be a vector field with continuous components defined along a smooth curve \(C\) parametrized by \(r:\mathbb{R}\rightarrow\mathbb{R}^n\) on \(a\leq t\leq b\). If \(T\) is the unit vector tangent to \(r\) at \(s\), then the line integral of \(F\) along \(C\) is
	\begin{equation*}
		\int_C F\cdot Tds=\int_C\bigg(F\cdot\diff{r}{s}\bigg)ds=\int_CF\cdot dr.
	\end{equation*}
\end{definition}
\begin{remark}
	Note that the line integral of a vector field is different from the line integral in definition \ref{directlineint}. In said definition, we are finding the area between the curve and the function. In the case of definition \ref{vectorlineint}, we are evaluating the definite integral of a function whose gradient is the output of \(f\), dotted with the derivative of the parametrized curve, \(dr/dt\). In other words, we are taking the integral of the directional derivative of \(f\) at \(r(t)\). It should be noted that \(F\) must be a function from \(\mathbb{R}^n\) to \(\mathbb{R}^n\) because the dimension of the domain and range of the gradient must be equal.
\end{remark}
\begin{theorem}
	Let \(F=Mi+Nj+Pk\) be a vector field whose components are continuous througout an open connected region \(D\) in space. Then \(F\) is conservative if and only if \(F\) is a gradient field \(\nabla f\) for a differentiable function \(f\).
\end{theorem}
\begin{definition}[Curl]
	Let \(F=M(x,y,z)i+N(x,y,z)j+P(x,y,z)k\) be a vector field. Then \(\text{curl}\,F\), called the curl of \(F\), is
	\begin{equation*}
		\bigg(\diffp{P}{y}-\diffp{N}{z}\bigg)i+\bigg(\diffp{M}{z}-\diffp{P}{x}\bigg)j+\big(\diffp{N}{x}-\diffp{M}{y}\bigg)k
	\end{equation*}
	For a two dimensional vector field \(F=M(x,y)+N(x,y)\), \(\text{curl}\,F\) is
	\begin{equation*}
		\bigg(\diffp{N}{x}-\diffp{M}{y}\bigg)k.
	\end{equation*}
\end{definition}
\begin{remark}
	Notice the first term describes counterclockwise rotation in the y-z plane relative to the normal \(i\), the second describes counterclockwise rotation in the z-x plane relative to the normal \(j\), and the third describes counterclockwise rotation in the x-y plane relative to the normal \(k\).
\end{remark}
\begin{theorem}
	Let \(F\) be a filed on an open simply connected domain whose component functions have continuous first partial derivatives. Then \(F\) is conservative iff \(\text{curl}\,F=0\).
\end{theorem}
\begin{definition}[Differential form]
	come back?
\end{definition}
\subsection{Green's theorem in the plane}
\begin{definition}[Circulation density]
	The circulation density of a vector field \(F=Mi+Nj\) at the point \((x,y)\) is the magnitude of the k-component of the curl of \(F\):
	\begin{equation*}
		\text{curl}\,F\cdot k=\diffp{N}{x}-\diffp{M}{y},
	\end{equation*}
\end{definition}
\begin{definition}[Divergence]
	The divergence, or flux density, of a vector field \(F=Mi+Nj\) at the point \((x,y)\) is
	\begin{equation*}
		\text{div}\,F=\diffp{M}{x}+\diffp{N}{y}.
	\end{equation*}
\end{definition}
\begin{theorem}[Green's theorem]
	Let \(C\) be a piecewise smooth, simple closed curve enclosing a region \(R\) in the plane. Let \(F=Mi+Nj\) be a vector field with \(M\) and \(N\) having continuous first derivatives in an open region containing \(R\). Then the counterclockwise circulation of \(F\) around \(C\) is
\begin{equation*}
	\oint_{C}F\cdot T\,ds=\iint_R\bigg(\diffp{N}{x}-\diffp{M}{y}\bigg)dxdy.
\end{equation*}
	Additionally, the outward flux of \(F\) across \(C\) is
\begin{equation*}
	\oint_{C}F\cdot n\,ds=\iint_R\bigg(\diffp{M}{x}+\diffp{N}{y}\bigg)dxdy.
\end{equation*}
\end{theorem}
\subsection{Path independence}
\begin{definition}[Path independence]
	Let \(\textbf{F}\) be a vector field defined on an open region \(D\) in space, and suppose that for any two points \(A\) and \(B\) in \(D\) the line integral \(\int_C \textbf{F}\cdot d\textbf{r}\) along a path \(C\) from \(A\) to \(B\) in \(D\) is the same over all paths from \(A\) to \(B\). Then the integral \(\int_C\textbf{F}\cdot d\textbf{r}\) is path independent in \(D\) and the field \(\textbf{F}\) is conservative in \(D\).
\end{definition}
\begin{theorem}[Fundamental theorem of line integrals]
	Let \(C\) be a smooth curve joining the point \(A\) to the point \(B\) in the plane or in space and parametrized by \(\textbf{r}(t)\). Let \(f\) be a differentiable function with a continuous gradient vector \(\textbf{F}=\nabla f\) on a domain \(D\) containing \(C\). Then
	\begin{equation*}
		\int_C\textbf{F}\cdot d\textbf{r}=f(B)-f(A).
	\end{equation*}
\end{theorem}
\begin{corollary}
	If \(F\) is the gradient function of some differentiable function \(f\), and \(C\) a closed curve, then
	\begin{equation*}
		\oint_C F\cdot dr=0.
	\end{equation*}
\end{corollary}
\begin{theorem}
	A vector field \(F\) is conservative iff \(F\) is a gradient field \(\nabla f\) for a differentiable function \(f\).
\end{theorem}
\end{document}
