\documentclass{article}
\usepackage[left=3cm,right=3cm,top=2cm,bottom=2cm]{geometry}
\setlength{\parindent}{0mm}

\usepackage[utf8]{inputenc}
\usepackage[english]{babel}
\usepackage{enumitem}
\usepackage{amsmath}
\usepackage{amsthm}
\usepackage{amssymb}
\usepackage{amsfonts}
\usepackage{mathrsfs}
\usepackage{mathtools}
\usepackage{esdiff}
\usepackage{IEEEtrantools}
\usepackage{geometry}
\usepackage{hyperref}

\setlist[enumerate,1]{label=(\alph*)}
\setlist[enumerate,2]{label=(\arabic*)}
\hypersetup{
	colorlinks,
	linkcolor={black},
	citecolor={black},
	filecolor={black},
	urlcolor={black},
}
\renewcommand{\IEEEQED}{\IEEEQEDopen}

\newcommand{\overbar}[1]{\mkern 1.0mu\overline{\mkern-1.0mu#1\mkern-1.0mu}\mkern 1.0mu}
\newcommand{\Mod}[1]{(\mathrm{mod}\ #1)}
\newcommand*\Eval[3]{\left.#1\right\rvert_{#2}^{#3}}
\DeclarePairedDelimiter\ceil{\lceil}{\rceil}
\DeclarePairedDelimiter\floor{\lfloor}{\rfloor}
\DeclarePairedDelimiter\abs{\lvert}{\rvert}
\DeclarePairedDelimiter\norm{\lVert}{\rVert}
\makeatletter
\let\oldabs\abs
\def\abs{\@ifstar{\oldabs}{\oldabs*}}
\let\oldnorm\norm
\def\norm{\@ifstar{\oldnorm}{\oldnorm*}}
\makeatother

\theoremstyle{plain}
\newtheorem{theorem}{Theorem}[section]
\theoremstyle{definition}
\newtheorem{lemma}[theorem]{Lemma}
\newtheorem{corollary}[theorem]{Corollary}
\newtheorem{proposition}[theorem]{Proposition}
\newtheorem{definition}[theorem]{Definition}
\newtheorem*{definition*}{Definition}
\newtheorem{example}[theorem]{Example} 
\newtheorem{summary}[theorem]{Summary}
\newtheorem{fact}[theorem]{Fact}
\newtheorem{caution}[theorem]{Caution}
\newtheorem{recall}[theorem]{Recall}
\newtheorem{question}[theorem]{Question}
\newtheorem*{remark}{Remark}
\newtheorem{notation}[theorem]{Notation}
\newtheorem{exer}[theorem]{Exercise}
\newtheorem{convention}[theorem]{Convention}
\newtheorem{assumption}[theorem]{Assumption}

\begin{document}

\title{HW 9}
\author{Samuel Lindskog}
\maketitle

\setcounter{section}{1}

\section*{Exercise 9.4.7}
First, we prove that a function \(f:\mathbb{R}\rightarrow\mathbb{R}\) defined by a real variable \(x\) taken to a nonnegative integer power is continuous. Let \(c\in\mathbb{R}\). If \(f=x^0\), then by definition \(f=1\). Thus \(\abs{f(x)-f(c)}=\abs{1-1}=0\), so
\begin{equation*}
	\forall\epsilon>0,\,\exists\delta>0,\,\forall x\in\mathbb{R},\,\big(\abs{x-c}<\delta\Rightarrow\abs{f(x)-f(c)}<\epsilon\big)
\end{equation*}
Is always true. By inductive hypothesis, suppose \(x^n\) is continuous for some \(n\in\mathbb{N}\). Then \(\lim_{x\rightarrow c}x^n=c^n\) for all \(c\in\mathbb{R}\). The function \(g:\mathbb{R}\rightarrow\mathbb{R}\) defined by \(g(x)=x\) is continuous because \(\abs{x-c}=\abs{f(x)-f(c)}\), thus for all \(\epsilon>0\), \(\abs{x-c}<\epsilon\Rightarrow\abs{f(x)-f(c)<\epsilon}\). It follows from limit laws in the text and the continuity of both \(x^n\) and \(x\) that
\begin{IEEEeqnarray*}{rCl}
	\lim_{x\rightarrow c}x^{n+1}&=&\lim_{x\rightarrow c}x^n\cdot x\\
	&=&\lim_{x\rightarrow c}x^n\cdot\lim_{x\rightarrow c}x\\
	&=&c^n\cdot c\\
	&=&c^{n+1}.
\end{IEEEeqnarray*}
This closes the induction. If follows from proposition that a continuous function multiplied by a constant is continuous. Thus
\begin{equation*}
	\sum_{i=0}^{n}c_ix^i
\end{equation*}
Is a finite linear combination of continuous functions, and is thus continuous by proposition.
\section*{Exercise 9.6.1}
\subsection*{a}

\end{document}
