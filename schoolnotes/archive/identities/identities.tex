\documentclass[nobib,notoc]{tufte-handout}
\usepackage[utf8]{inputenc}
\usepackage[english]{babel}
\usepackage{amsmath}
\usepackage{amsthm}
\usepackage{amsfonts}
\usepackage{hyperref}
\usepackage{mathrsfs}
\usepackage{IEEEtrantools}
\usepackage{enumitem}

\renewcommand{\IEEEQED}{\IEEEQEDopen}

\begin{document}

\theoremstyle{definition}\newtheorem{defi}{Definition}[section]
\theoremstyle{definition}\newtheorem{axiom}{Axiom}[section]
\theoremstyle{definition}\newtheorem{thm}{Theorem}[section]
\theoremstyle{definition}\newtheorem{cor}{Corollary}[section]
\theoremstyle{definition}\newtheorem{lem}{Lemma}[section]
\theoremstyle{remark}\newtheorem*{notat}{Notation}
\theoremstyle{remark}\newtheorem*{rema}{Remark}
\theoremstyle{definition}\newtheorem{problem}{Problem}
%\renewcommand{\theproblem}{\arabic{problem}}
\newenvironment{prob}[1]{\protect\setcounter{problem}{#1}\addtocounter{problem}{-1}\begin{problem}}{\end{problem}}

\title{thingies and doodads}
\author{Samuel Lindskog}
\maketitle
\setcounter{section}{1}
\section{Misc algebra}
\subsection{Partial fractions}
Rational expressions can be split into the sum of two rational expressions. Use the following to solve for \(A\) and \(B\).
\begin{IEEEeqnarray*}{rCl}
	\frac{x+7}{(x-3)(x+2)}&=&\frac{A}{x-3}+\frac{B}{x-2}\\
	x+7&=&A(x+2)+B(x-3)
\end{IEEEeqnarray*}
\subsection{Quadratic formula}
\begin{equation*}
	x=\frac{-b\pm\sqrt{b^2-4ac}}{2a}
\end{equation*}
\subsection{Factoring quadratics}
\subsection{Completing the square}
\subsection{Factoring cubics}

\section{Trig identities}

\section{Integration procedures}
\subsection{Integration by parts}
\subsection{Trig integrals}

\section{Standard integrals}
%complete up to here today

\section{Misc calculus}
\subsection{Taylor series}
\subsection{Convergence tests}
\subsection{Optimization}
\begin{defi}[Lagrange multipliers]
\end{defi}

\end{document}
