\documentclass{article}
\usepackage{settings}

\geometry{
a4paper,
total={140mm,257mm},
left=35mm,
top=20mm,
}

\title{Real Analysis}
\author{Samuel Lindskog}

\begin{document}
\maketitle
\addtocontents{toc}{\protect\hypertarget{toc}{}}
\tableofcontents
\pagenumbering{gobble}
\clearpage
\pagenumbering{arabic}
\setcounter{page}{1}

\section{The Natural Numbers}
\subsection{Peano Axioms}
\begin{definition}[Peano axioms]
	Using \(++\) as the successor operation, the natural numbers are defined as follows:
	\begin{enumerate}
		\item \(0\) is a natural number.
		\item If \(n\) is a natural number, then \(n++\) is also a natural number.
		\item For all natural numbers \(n\), \(n++\neq 0\).
	\end{enumerate}
\end{definition}
\begin{definition}[Addition of natural numbers]
	Let \(m\) be a natural number. \(0+m\coloneq m\) and \((n++)+m\coloneq(n+m)++\).
\end{definition}
\begin{proposition}
	\label{uniquezero}
	There is only one zero, i.e. for \(a\in\mathbb{N}\) if \(0+a=0'+a=a\), then \(0=0'\).
\end{proposition}
	\begin{IEEEproof}
		Suppose \(0\neq 0'\). Then \(0\) is a successor of \(0'\) or \(0'\) is a successor of \(0\). Because no successor of a natural number is \(0\), this is impossible.
	\end{IEEEproof}
\begin{proposition}
	\label{zerocommute}
	\(m+0=m\).
\end{proposition}
	\begin{IEEEproof}
		Let \(n\in\mathbb{N}\). \(0+0\coloneq 0\), so by inductive hypothesis \(n+0=n\). \((n++)+0\coloneq(n+0)++\), and from the inductive hypothesis equals \(n++\).
	\end{IEEEproof}
\begin{lemma}
	\label{rightaddition}
	For any natural numbers \(n\) and \(m\), \(n+(m++)=(n+m)++\).
\end{lemma}
	\begin{IEEEproof}
		Suppose \(n,m\in\mathbb{N}\). \(0+(m++)\coloneq m++=(0+m)++\). By inductive hypothesis \(n+(m++)=(n+m)++\). From the definition of addition \((n++)+(m++)=(n+(m++))++\) and from the inductive hypothesis \(n+(m++)=(n+m)++\) so we have
		\begin{IEEEeqnarray*}{rCl}
			(n++)+(m++)&=&(n+(m++))++\\
			&=&((n+m)++)++\\
			&=&((n++)+m)++
		\end{IEEEeqnarray*}
	\end{IEEEproof}
\begin{proposition}[Commutivity of addition]
	For \(n,m\in\mathbb{N}\), \(n+m=m+n\).
\end{proposition}
	\begin{IEEEproof}
		Let \(n,m\in\mathbb{N}\). From proposition \ref{zerocommute}, \(0+m=m+0\), so by inductive hypothesis \(n+m=m+n\). \((n++)+m=(n+m)++\) and from inductive hypothesis this equals \((m+n)++\). From lemma \ref{rightaddition}, this equals \(m+(n++)\).
	\end{IEEEproof}
\begin{proposition}
	If \(a,b\in\mathbb{N}\) and \(a+b=a\), then \(b=0\).
\end{proposition}
	\begin{IEEEproof}
		Suppose \(a,b\in\mathbb{N}\) with \(a+b=a\). \(\)
	\end{IEEEproof}
\begin{proposition}[Associativity of addition]
	Let \(a,b,c\in\mathbb{N}\). Then \((a+b)+c=a+(b+c)\).
\end{proposition}
	\begin{IEEEproof}
		Suppose \(a,b\in\mathbb{N}\). From here we utilize the definition of addition, and commutivity of addition for the rest of the proof. It follows that \((a+b)+0=a+b=a+(b+0)\). By inductive hypothesis suppose \((a+b)+c=a+(b+c)\) for \(c\in\mathbb{N}\). Then
		\begin{IEEEeqnarray*}{rCl}
			(a+b)+c++&=&[(a+b)+c]++\\
			&=&[a+(b+c)]++\\
			&=&a+(c+b)++\\
			&=&a+[(c++)+b]\\
			&=&a+(b+c++)
		\end{IEEEeqnarray*}
	\end{IEEEproof}
\begin{proposition}[Cancellation law]
	Let \(a,b,c\in\mathbb{N}\). Iff \(a+b=a+c\), then \(b=c\).
\end{proposition}
	\begin{IEEEproof}
		If \(0+b=0+c\) then from the definition of addition \(b=c\). By inductive hypothesis for any \(n\in\mathbb{N}\), \(n+b=n+c\). \((n++)+b=(n+b)++\) and \((n++)+c=(n+c)++\), so from the inductive hypothesis and the axioms of natural numbers, \((n++)+b=(n++)+c\).
	\end{IEEEproof}
\begin{definition}[Positive natural number]
	A natural number \(n\) is said to be positive iff it is not \(0\).
\end{definition}
\begin{definition}[Ordering of natural numbers]
	Let \(n,m\in\mathbb{N}\). We write \(n\geq m\) or \(m\geq n\) iff \(n=m+a\) for some \(a\in\mathbb{N}\).
\end{definition}
\begin{proposition}
	\label{nonzerosum}
	If \(a\) or \(b\) are not zero, then \(a+b\neq 0\).
\end{proposition}
	\begin{IEEEproof}
		Suppose \(a,b\in\mathbb{N}\) with \(b\neq 0\). If \(a=0\) then \(a+b=0+b=b\neq 0\). If \(a\neq 0\), because no natural number has zero as a successor it follows from the definition of addition that \(a+b\neq 0\).
	\end{IEEEproof}
\begin{proposition}[Trichotomy of order for natural numbers]
	Let \(a,b\in\mathbb{N}\). Then exactly one of the following statements is true: \(a<b,\,a=b,\,a>b\).
\end{proposition}
	\begin{IEEEproof}
		Suppose \(a,b\in\mathbb{N}\) and \(a<b\). Then for some \(c\in\mathbb{N}\), \(a=b+c\) with \(b\neq a\). If \(c=0\) then \(a=b\), a contradiction. If \(b<a\), then for some \(d\in\mathbb{N}\), \(b=a+d\) with \(a\neq b\). If \(d=0\) then \(a=b\), a contradiction. Because \(b=b+d+c\) and \(c,d\neq 0\), it follows from commutivity and propositions \ref{nonzerosum} and \ref{uniquezero} that this is impossible. Therefore wlog if \(a<b\) then \(a\) is not greater than or equal to \(b\).
	Suppose \(a=b\). If \(a<b\) then \(a=b+c\) for some \(c\in\mathbb{N}\) with \(b\neq c\), a contradiction. Therefore wlog if \(a=b\) then \(a\) is not less than or greater than \(b\).
	\end{IEEEproof}
\begin{proposition}[Strong principle of induction]
	Let \(m_0,m,m'\in\mathbb{N}\), and let \(P(x)\) be a property of arbitrary \(x\in\mathbb{N}\). Suppose that for each \(m\geq m_0\) the following implication holds:
	\begin{equation*}
		\bigg(\forall m'\in[m_0,m),\,P(m')\bigg)\Rightarrow P(m).
	\end{equation*}
	Then we can conclude \(P(m)\) is true for all natural numbers \(m\geq m_0\).
\end{proposition}
\subsection{Multiplication}
\begin{definition}[Multiplication of natural numbers]
	Let \(m\) be a natural number. \(0\times m\coloneq 0\) and \((n++)\times m\coloneq (n\times m)+m\).
\end{definition}
\begin{proposition}
	\label{0m}
	\(m\times 0=0\).
\end{proposition}
	\begin{IEEEproof}
		From the definition of multiplication, \(0\times 0=0\). By inductive hypothesis suppose \(m\times 0=0\). Then \((m++)\times 0=(m\times 0)+0=0\).
	\end{IEEEproof}
\begin{proposition}
	\label{rightmult}
	For \(n,m\in\mathbb{N}\), \(n\times (m++)=(n\times m)+n\).
\end{proposition}
\begin{IEEEproof}
	Let \(n,m\in\mathbb{N}\). \(0\times (m++)=0=(0\times m)+0\). By inductive hypothesis, \((n\times (m++))=(n\times m)+n\). It follows that
	\begin{IEEEeqnarray*}{rCl}
		(n++)\times(m++)&=&(n\times(m++))+(m++)\\
		&=&(n\times m)+n+(m++)\\
		&=&(n\times m)+m+(n++)\\
		&=&((n++)\times m)+(n++)
	\end{IEEEeqnarray*}
\end{IEEEproof}
\begin{proposition}
	\label{1m}
	For \(m\in\mathbb{N}\), \(1m=m\).
\end{proposition}
	\begin{IEEEproof}
		If \(m\in\mathbb{N}\) \(0\times m=0\). Then \((0++)\times m=1\times m=0+m=m\).
	\end{IEEEproof}
\begin{lemma}[Commutivity of multiplication]
	Let \(n,m\in\mathbb{N}\). Then \(n\times m=m\times n\).
\end{lemma}
	\begin{IEEEproof}
		Let \(n,m\in\mathbb{N}\). \(0\times m=m\times 0=0\). By inductive hypothesis, \(n\times m=m\times n\). It follows from proposition \ref{rightmult} that
		\begin{IEEEeqnarray*}{rCl}
			(n++)\times m&=&(n\times m)+m\\
			&=&(m\times n)+m\\
			&=&m\times(n++)
		\end{IEEEeqnarray*}
	\end{IEEEproof}
\begin{proposition}[Distributive law]
	For any natural numbers \(a,b,c,\) we have \(a(b+c)=ab+ac\).
\end{proposition}
\begin{IEEEproof}
	TODO
\end{IEEEproof}
\begin{proposition}[Associativity of multiplication]
	If \(a,b,c\in\mathbb{N}\) then \((a\times b)\times c=a\times(b\times c)\).
\end{proposition}
\begin{IEEEproof}
	TODO
\end{IEEEproof}
\begin{proposition}
	\label{posinatmult}
	If \(a,b\in\mathbb{N}^+\), then \(ab\neq 0\).
\end{proposition}
	\begin{IEEEproof}
		Let \(a\in\mathbb{N}^+\). By proposition \ref{1m} \(1a=a\) and \(a\) is positive. By inductive hypothesis if \(n\in\mathbb{N}^+\) then \(na\) is positive. \(n++\) is a successor to \(n\), and no successor of a natural number is zero, so \(n++\) is positive. \((n++)a=na+a\). Both \(na\) and \(a\) are positive and by proposition \ref{nonzerosum}, \(na+a\) is positive and thus not zero.
	\end{IEEEproof}
\begin{proposition}
	If \(a,b\) are natural numbers such that \(a<b\), and \(c\) is positive, then \(ac<bc\).
\end{proposition}
\begin{corollary}
	Let \(a,b,c\in\mathbb{N}\) such that \(ac=bc\) and \(c\) is non-zero. Then \(a=b\).
\end{corollary}
\begin{proposition}[Euclid's division lemma]
	Let \(n\) be a natural number, and let \(q\) be a positive number. Then there exist natural numbers \(m,r\) such that \(0\leq r<q\) and \(n=mq+r\).
\end{proposition}
\begin{definition}[Exponentiation for natural numbers]
	Let \(m\in\mathbb{N}\). \(m^0\coloneq 1\), and \(m^{n++}=m^n\times m\).
\end{definition}
\section{Set Theory}
\subsection{Fundamentals}
\begin{definition}[Axioms of sets]
	\label{setaxioms}
	\,
	\begin{enumerate}
		\item (Sets are objects) If \(A\) is a set, then \(A\) is also an object. In particular, given two sets \(A\) and \(B\), it is meaningful to ask whether \(A\) is also an element of \(B\).
		\item (Equality of sets) Two sets \(A\) and \(B\) are equal iff every element of \(A\) is an element of \(B\) and vice versa.
		\item (Empty set) There exists a set known as the empty set, denoted \(\emptyset\), which contains no elements. In other words, for all objects \(x\) we have \(x\notin\emptyset\).
		\item (Singleton sets) If \(a\) is an object, then there exists a set \(\{a\}\) whose only element is \(a\), i.e. for every object \(y\) we have \(y\in\{a\}\) iff \(y=a\). \(\{a\}\) is referred to as a singleton set.
		\item (Pairwise union) Given any two sets \(A\) and \(B\), there exists a set \(A\cup B\), called the union of \(A\) and \(B\), which consists of all the elements which belong to \(A\) or \(B\). In other words,
			\begin{equation*}
				x\in A\cup B\Leftrightarrow(x\in A\vee x\in B).
			\end{equation*}
		\item (Axiom of specification) Let \(A\) be a set, and for each \(x\in A\) let \(P(x)\) be a property pertaining to \(x\). Then there exists a set \(\{x\in A\,|\,P(x)\}\) whose elements are precisely the elements \(x\) in \(A\) for which \(P(x)\) is true.
		\item (Replacement) Let \(A\) be a set. For any object \(x\in A\) and any object \(y\), suppose we have a property \(P(x,y)\) that is true for at most one \(y\) for each \(x\in A\). Then
			\begin{equation*}
				z\in\{y\,|\,P(x,y),\, x\in A\}\Leftrightarrow P(x,z).
			\end{equation*}
		\item (Infinity) There exists a set \(\mathbb{N}\), whose elements are called natural numbers, as well as an object \(0\in\mathbb{N}\), and an object \(N++\) assigned to every natural number \(n\in\mathbb{N}\), such that the Peano axioms hold.
		\item (Universal specification) DANGER - Suppose for every object \(x\) we have a property \(P(x)\). Then there exists a set \(\{x\,|\,P(x)\}\).
		\item (Regularity). If \(A\) is a non-empty set, then there is at least one element \(x\) of \(A\) which is either not a set, or is disjoint from \(A\).
		\item (Power set) Let \(X\) and \(Y\) be sets. Then there exists a set, denoted \(Y^X\), which consists of all the functions from \(X\) to \(Y\), thus
			\begin{equation*}
				f\in Y^X\Leftrightarrow f\text{ is a function from } X\text{ to }Y.
			\end{equation*}
		\item (Union) Let \(A\) be a set whose elements are all sets. Then there exists a set \(\bigcup A\) defined
			\begin{equation*}
				x\in\bigcup A=\{x\,|\,\exists S\in A,\,x\in S\}.
			\end{equation*}
	\end{enumerate}
\end{definition}
\begin{remark}
	The axioms of set theory introduced, excluding universal specification, are known as the Zermelo-Fraenkel axioms of set theory.
\end{remark}
\begin{lemma}[Single choice]
\label{singlechoice}
	Let \(A\) be a non-empty set. Then there exists an object \(x\) such that \(x\in A\).
\end{lemma}
\begin{IEEEproof}
	Suppose there does not exist any object \(x\) such that \(x\in A\). Simultaneously \(x\notin\emptyset\), so \(x\in A\Leftrightarrow x\in\emptyset\) and \(A=\emptyset\), a contradiction.
\end{IEEEproof}
\begin{definition}[Subset]
	Let \(A,B\) be sets. We say that \(A\) is a subset of \(B\), denoted \(A\subseteq B\), iff every element of \(A\) is also an element of \(B\). We say that \(A\) is a proper subset of \(B\), denoted \(A\subsetneq B\), if \(A\subseteq B\) and \(A\neq B\).
\end{definition}
\begin{theorem}
	Let \(A\) be a set. Then \(\emptyset\subseteq A\).
\end{theorem}
\begin{IEEEproof}
	If \(\emptyset\subseteq A\) then for all objects \(x\),
	\begin{equation*}
		x\in\emptyset\Rightarrow x\in A.
	\end{equation*}
	This is vacuously true because there does not exist \(x\) such that \(x\in\emptyset\).
\end{IEEEproof}
\begin{definition}[Intersection]
	The intersection \(S_1\cap S_2\) of two sets is the set
	\begin{equation*}
		S_1\cap S_2=\{x\,|\,x\in S_1\wedge x\in S_2\}.
	\end{equation*}
\end{definition}
\begin{definition}[Union]
	The union \(S_1\cup S_2\) of two sets is the set
	\begin{equation*}
		S_1\cup S_2=\{x\,|\,x\in S_1\vee x\in S_2\}.
	\end{equation*}
\end{definition}
\begin{definition}[Disjoint]
	Two sets are disjoint if \(A\cap B=\emptyset\).
\end{definition}
\begin{definition}[Difference set]
	If \(A\) and \(B\) are sets, the set \(A\setminus B\) is the set \(A\) with any elements of \(B\) removed, i.e.
	\begin{equation*}
		A\setminus B\coloneq\{x\,|\,x\in A\wedge x\notin B\}.
	\end{equation*}
\end{definition}
\begin{proposition}
	Let \(A,B,C\) be subsets of set \(X\).
	\begin{enumerate}
		\item (Minimal element) \(A\cup\emptyset=A\) and \(A\cap\emptyset=\emptyset\).
		\item (Maximal element) \(A\cup X=X\) and \(A\cap X=A\).
		\item (Identity) \(A\cap A=A\) and \(A\cup A=A\).
		\item (Commutativity) \(A\cup B=B\cup A\) and \(A\cap B=B\cap A\).
		\item (Associativity) \((A\cup B)\cup C=A\cup(B\cup C)\) and \((A\cap B)\cap C=A\cap(B\cap C)\).
		\item (Distributivity) \(A\cap (B\cup C)=(A\cap B)\cup(A\cap C)\) and \(A\cup(B\cap C)=(A\cup B)\cap(A\cup C)\).
		\item (Partition) \(A\cup(X\setminus A)=X\) and \(A\cap (X\setminus A)=\emptyset\).
		\item (De Morgan Laws) \(X\setminus(A\cup B)=(X\setminus A)\cap(X\setminus B)\) and \(X\setminus (A\cap B)=(X\setminus A)\cup(X\setminus B)\).
	\end{enumerate}
\end{proposition}
\begin{definition}[Ordered pair]
	If \(x\) and \(y\) are any objects, we define the ordered pair \((x,y)\) to be a new object which consists of \(x\) as its "first component" and \(y\) as its "second component". Two ordered pairs \(x,y\) and \(x',y'\) are equal if
	\begin{equation*}
		x=x',\quad y=y'.
	\end{equation*}
\end{definition}
\begin{definition}[Cartesian product]
	Let \(A,B\) be sets. Then the cartesian product of \(A\) and \(B\), written \(A\times B\), is
	\begin{equation*}
		A\times B=\{(a,b)\,|,a\in A,\,b\in B\}.
	\end{equation*}
\end{definition}
\begin{definition}[Ordered \(n\)-tuple]
	Let \(n\) be a natural number. An ordered \(n\)-tuple \((x_i)_{1\leq i\leq n}\) is a collection of objects \(x_i\), one for every natural number \(i\) between \(1\) and \(n\). Two ordered \(n\)-tuples \((x_i)_{1\leq i\leq n}\) and \((y_i)_{1\leq i\leq n}\) are said to be equal iff \(x_i=y_i\) for all \(1\leq i\leq n\).
\end{definition}
\begin{definition}[\(n\)-fold Cartesian product]
	If \((X_i)_{1\leq i\leq n}\) is an ordered \(n\)-tuple of sets, their Cartisian product \(\prod_{i=1}^{n}X_i\) is defined
	\begin{equation*}
		\prod_{i=1}^n X_i=\{(x_i)_{1\leq i\leq n}\,|\, x_i\in X_i\}.
	\end{equation*}
\end{definition}
\begin{definition}[Indexed family]
	If for each element \(j\in J\) with \(J\neq\emptyset\), there corresponds a set \(A_j\), then
	\begin{equation*}
		\mathscr{A}=\{A_j\,|\,j\in J\}.
	\end{equation*}
	Is called an indexed family of sets with \(J\) as the index set. If \(J=\{1,2,\ldots,n\}\) we may index the set similarly to sum notation.
\end{definition}
\begin{definition}[Union and intersection of indexed family]
	The union of all sets in an indexed family \(\mathscr{A}\) with index set \(J\) is
	\begin{equation*}
		\bigcup_{j\in J}A_j=\{x\,|\,\exists A_j\in\mathscr{A},\,x\in A_j\}.
	\end{equation*}
	The intersection of all sets in \(\mathscr{A}\) is
	\begin{equation*}
		\bigcap_{j\in J}A_j=\{x\,|\,\forall A_j\in\mathscr{A},\,x\in A_j\}.
	\end{equation*}
\end{definition}
\begin{lemma}[Finite choice]
	Let \(n\geq 1\) be a natural number, and for each natural number \(1\leq i\leq n\), let \(X_i\) be a non-empty set. Then there exists an \(n\)-tuple \((x_i)_{1\leq i\leq n}\) such that \(x_i\in X_i\) for all \(1\leq i\leq n\). In other words if each \(X_i\) is non-empty, then its \(n\)-fold cartesian product is nonempty.
\end{lemma}
\begin{IEEEproof}
	Let \(\mathscr{A}=\{A_i\,|\,1\leq i\leq n\}\) with \(n\in\mathbb{N}\) be an indexed family of nonempty sets. It follows from lemma \ref{singlechoice} that for each \(A_i\), \(1\leq i\leq n\), there exists \(a_i\in A_i\). Using this fact, define an ordered \(n\)-tuple \((a_i)_{1\leq i\leq n}\).
\end{IEEEproof}
\begin{definition}[Upper and lower bound]
	Let \(S\subseteq\mathbb{R}\). If there exists a real number \(m\) such that \(m\geq s\) for all \(s\in S\), then \(m\) is called an upper bound of \(S\), and we say that \(S\) is bounded above. If \(m\leq s\) for all \(s\in S\), then \(m\) is a lower bound of \(S\) and \(S\) is bounded below. The set \(S\) is said to be bounded if it is bounded above and bounded below.
\end{definition}
\begin{definition}[Maximum and minimum]
	If an upper bound \(m\) of \(S\) is a member of \(S\), then \(m\) is called the maximum of \(S\), and we write \(m=\text{max}\, S\). If a lower bound of \(S\) is a member of \(S\), then it is called the minimum of \(S\), and we write \(m=\text{min}\, S\).
\end{definition}
\begin{definition}[Supremum and infimum]
	Let \(S\) be a nonempty subset of \(\mathbb{R}\). If \(S\) is bounded above, then the least upper bound of \(S\) is called its supremum, denoted \(\sup S\). Therefore \(m=\sup S\) iff
	\begin{enumerate}
		\item \(m\geq s\) for all \(s\in S\).
		\item If \(m'<m\), then there exists \(s'\in S\) such that \(s'>m'\).
	\end{enumerate}
	If \(S\) is bounded below, then the greatest lower bound of \(S\) is called its infimum and is denoted by \(\inf S\).
\end{definition}
\begin{theorem}[Archimedean property]
	For each \(x>0\), there exists \(n\in\mathbb{N}\) such that \(0<1/n<x\).
\end{theorem}
\subsection{Functions}
\begin{definition}[Relation]
	Let \(A,B\)	be sets. A relation between \(A\) and \(B\) is an subset of \(A\times B\).
\end{definition}
\begin{definition}[Equivalence relation]
	An equivalence relation on a set \(S\) is a relation such that for all \(x,y,z\in S\), the relation satisfies the following properties:
	\begin{enumerate}
		\item (Reflexive property) \(xRx\).
		\item (Symmetric property) \(xRy\Rightarrow yRx\).
		\item (Transitive property) \(xRy\wedge yRx\Rightarrow xRz\).
	\end{enumerate}
\end{definition}
\begin{definition}[Partition]
	A partition of a set \(S\) is a collection \(\mathscr{P}\) of nonempty subsets of \(S\) that are pairwise disjoint, and whose union is \(S\), i.e.
	\begin{enumerate}
		\item \(A=\bigcup\mathscr{P}\).
		\item \(\forall A,B\in\mathscr{P},\,A\neq B\Rightarrow A\cap B=\emptyset\).
	\end{enumerate}
\end{definition}
\begin{definition}[Function]
	A function from \(A\) to \(B\), denoted \(f:A\rightarrow B\) is a nonempty relation \(f\subseteq A\times B\) that satisfies the following properties:
	\begin{enumerate}
		\item (Existence) \(\forall a\in A,\,\exists b\in B,\,(a,b)\in f\).
		\item (Uniqueness) \((a,b)\in f\wedge(a,c)\in f\Rightarrow b=c\).
	\end{enumerate}
	Set \(A\) is called the domain of \(f\), and set \(B\) is called the codomain. The range of \(f\) is \(f(A)\), i.e. \(\{b\in B\,|\,(a,b)\in f\}\).
\end{definition}
\begin{definition}[Equality of functions]
	Two functions \(f:X\rightarrow Y\) and \(g:X'\rightarrow Y'\) are equal if their domains and codomains are equal, and furthermore that \(f(x)=g(x)\) for all \(x\in X\).
\end{definition}
\begin{definition}[Composition]
	Let \(f:X\rightarrow Y\) and \(g:Y\rightarrow Z\) be two functions such that the codomain of \(f\) is the same set as the domain of \(g\). Then the composition \(g\circ f:X\rightarrow Z\) of the two functions \(g\) and \(f\) is the function defined by the formula
	\begin{equation*}
		(g\circ f)(x)=g(f(x)).
	\end{equation*}
\end{definition}
\begin{lemma}
	Let \(f:Z\rightarrow W\), \(g:Y\rightarrow Z\), and \(h:X\rightarrow Y\) be functions. Then \(f\circ(g\circ h)=(f\circ g)\circ h\).
	\begin{IEEEproof}
		\(g\circ h\) is a function from \(X\) to \(Z\), and \(f\circ g\) is a function from \(Y\rightarrow W\), so \((f\circ g)\circ h\) and \(f\circ(g\circ h)\) are functions from \(X\) to \(W\). It follows from the definition of function composition that
		\begin{IEEEeqnarray*}{rCl}
			(f\circ(g\circ h))(x)&=&f((g\circ h)(x))\\
			&=&f(g(h(x)))\\
			&=&(f\circ g)(h(x))\\
			&=&((f\circ g)\circ h)(x)
		\end{IEEEeqnarray*}
	\end{IEEEproof}
\end{lemma}
\begin{definition}[Injective]
	A function \(f:X\rightarrow Y\) is injective (one-to-one) if for \(x,x'\in X\),
	\begin{equation*}
		x\neq x'\rightarrow f(x)\neq f(x')
	\end{equation*}
\end{definition}
\begin{definition}[Surjective]
	A function \(f:X\rightarrow Y\) is surjective (onto) if
	\begin{equation*}
		\forall y\in Y,\,\exists x\in X,\,f(x)=y.
	\end{equation*}
\end{definition}
\begin{definition}[Bijective]
	A function is bijective (invertible) if it is injective and surjective.
\end{definition}
\begin{proposition}
	Let \(f:A\rightarrow B\) and \(g:B\rightarrow C\). Then
	\begin{enumerate}
		\item If \(f\) and \(g\) are surjective, then \(g\circ f\) is surjective.
		\item If \(f\) and \(g\) are injective, then \(g\circ f\) is injective.
		\item If \(f\) and \(g\) are bijective, then \(g\circ f\) is bijective.
	\end{enumerate}
\end{proposition}
\begin{lemma}
	If \(f:X\rightarrow Y\) is bijective then \(f\) is invertible. In other words for all \(y\in Y\) there exists a unique \(x\in X\) denoted \(f^{-1}(y)\) such that \(f(x)=y\). Therefore the inverse of \(f\), \(f^{-1}:Y\rightarrow X\) exists and is defined
	\begin{equation*}
		f^{-1}(y)=x.
	\end{equation*}
\end{lemma}
\begin{definition}[Identity function]
	A function defined on a set \(A\) that maps each element in \(A\) onto itself is called the identity function on \(A\), and is denoted \(i_A\).
\end{definition}
\begin{proposition}
	Let \(f:A\rightarrow B\) be bijective. Then
	\begin{enumerate}
		\item \(f^{-1}:B\rightarrow A\) is bijective.
		\item \(f^{-1}\circ f=i_A\) and \(f\circ f^{-1}=i_B\).
	\end{enumerate}
\end{proposition}
\begin{theorem}
	Let \(f:A\rightarrow B\) and \(g:A\rightarrow B\) be bijective. Then the composition \(g\circ f:A\rightarrow C\) is bijective and \((g\circ f)^{-1}=f^{-1}\circ g^{-1}\).
\end{theorem}
%
\begin{definition}[Image]
	If \(f:X\rightarrow Y\) is a function from \(X\) to \(Y\), and \(S\subseteq X\), we define the image of \(S\) under \(f\), \(f(S)\) to be the set
	\begin{equation*}
		f(S)=\{f(x)\,|\,x\in S\}.
	\end{equation*}
\end{definition}
\begin{definition}[Inverse image]
	If \(U\) is a subset of \(Y\), we define the set \(f^{-1}(U)\) to be the set
	\begin{equation*}
		f^{-1}(U)=\{x\in X\,|\,f(x)\in U\}.
	\end{equation*}
	We call \(f^{-1}(U)\) the inverse image of \(U\).
\end{definition}
\begin{theorem}
	Suppose that \(f:A\rightarrow B\), let \(C,C_1,C_2\) be subsets of \(A\), and let \(D,D_1,D_2\) be subsets of \(B\). Then the following hold:
	\begin{enumerate}
		\item \(C\subseteq f^{-1}(f(C))\).
		\item \(f(f^{-1}(D))\subseteq D\).
		\item \(f(C_1\cap C_2)\subseteq f(C_1)\cap f(C_2)\).
		\item \(f(C_1\cup C_2)=f(C_1)\cup f(C_2)\).
		\item \(f^{-1}(D_1\cap D_2)=f^{-1}(D_1)\cap f^{-1}(D_2)\).
		\item \(f^{-1}(D_1\cup D_2)=f^{-1}(D_1)\cup f^{-1}(D_2)\).
		\item \(f^{-1}(B\setminus D)=A\setminus f^{-1}(D)\).
	\end{enumerate}
	\begin{IEEEproof}
		test
	\end{IEEEproof}
\end{theorem}
\begin{proposition}
	\label{imagessubset}
	If \(X,Y\) are sets and \(f:X\rightarrow Y\) then \(f(X)\subseteq Y\).
\end{proposition}
\begin{IEEEproof}
	\(y\in f(X)\) implies \(y\in \{y\,|\,(x,y)\in f\}\) and \(f\) is a subset of \(X\times Y\), so it follows from the definition of the cartesian product that \(y\in Y\).
\end{IEEEproof}
\begin{lemma}
	Let \(X\) be a set. Then the set
	\begin{equation*}
		\{Y\,|\,Y\subseteq X\}
	\end{equation*}
	Is a set.
\end{lemma}
\begin{IEEEproof}
	Let \(X\) be a set and \(A\subseteq X\) with \(A\neq\emptyset\). Then there exists \(p\in A\), and we can define a function \(f:X\rightarrow A\) with \(x\in X\) by
	\begin{equation*}
		f(x)=\begin{cases}
			x\in A&f(x)=x\\
			x\notin A&f(x)=p
		\end{cases}
	\end{equation*}
	Thus for all \(a\in f(X)\), \(a\in A\) or \(a=p\in A\), so \(f(X)\subseteq A\). Next, for all \(x\in X\), \((x,f(x))\in f(X)\). Because for all \(a\in A\) we have \(a\in X\) then for all \(a\in A\), \((a,f(a))=(a,a)\in f(X)\) so from the definition of an image \(A\subseteq f(X)\). Thus \(A=F(X)\). From the power set axiom in definition \ref{setaxioms}, 
	\begin{equation*}
		\{f:X\rightarrow A\,|\,A\subseteq X\wedge A\neq\emptyset\}\subseteq X^X
	\end{equation*}
	From replacement, pairwise union, and singleton set axioms in definition \ref{setaxioms}, we can define a set \(\text{P}(X)\) that is the union of all images of functions in \(X^X\), and \(\{\emptyset\}\). As established above, all nonempty subsets of \(X\) are included in this set, and from proposition \ref{imagessubset} all images of functions in \(X^X\) are subsets of \(X\).
\end{IEEEproof}
\begin{definition}[Power set]
	For a set \(X\), the set \(\{Y\,|\,Y\subseteq X\}\) is called the power set of \(X\), and is denoted \(\text{P}(X)\) or \(2^X\).
\end{definition}
\begin{definition}[Cardinality]
	We say that two sets \(X\) and \(Y\) have equal cardinality iff there exists a bijection \(f:X\rightarrow Y\) from \(X\) to \(Y\).
\end{definition}
\begin{proposition}
	Let \(X,Y,Z\) be sets.
	\begin{enumerate}
		\item \(X\) has equal cardinality with \(X\).
		\item If \(X\) has equal cardinality with \(Y\), then \(Y\) has equal cardinality with \(X\).
		\item If \(X\) has equal cardinality \(Y\) and \(Y\) has equal cardinality with \(Z\), then \(X\) has equal cardinality with \(Z\)
			\begin{IEEEproof}
			\end{IEEEproof}
	\end{enumerate}
\end{proposition}
\begin{definition}[Cardinality \(n\)]
	Let \(n\) be a natural number. A set \(X\) is said to have cardinality \(n\), if it has equal cardinality with \(\{\in\mathbb{N}\,|\,1\leq i\leq n\}\). In this case we say that \(X\) has \(n\) elements.
\end{definition}
\begin{lemma}
	Suppose that \(n\geq 1\), and set \(X\) has cardinality \(n\). Then \(X\) is non-empty, and if \(x\) is any element of \(X\), then the set \(X-\{x\}\) has cardinality \(n-1\).
\end{lemma}
\begin{proposition}
	Let \(X\) be a set with some cardinality \(n\). Then \(X\) cannot have any other cardinality, i.e. \(X\) cannot have cardinality \(m\) for any \(m\neq n\).
\end{proposition}
\begin{definition}[Finite set]
	A set is finite iff it has cardinality \(n\) for some natural number \(n\); otherwise, the set is called infinte.
\end{definition}
\begin{theorem}
	The set of natural numbers is infinite.
\end{theorem}
\section{Integers and Rationals}
\subsection{The Integers}
\begin{definition}[Integers]
	An integer is an expression of the form \(a-b\), where \(a\) and \(b\) are natural numbers. Two integers are considered to be equal, \(a-b=c-d\), iff \(a+d=c+b\). The set of all integers is denoted \(\mathbb{Z}\).
\end{definition}
\begin{remark}
	The use of \(-\) is purely notational (until subtraction is defined). \(a-b\) can be interpreted as an ordered pair in \(\mathbb{N}\times\mathbb{N}\).
\end{remark}
\begin{definition}[Integer addition]
	The sum of two integers \((a-b)+(c-d)\) is defined by the formula
	\begin{equation*}
		(a-b)+(c-d)=(a+c)-(c+d)
	\end{equation*}
\end{definition}
\begin{definition}[Integer multiplication]
	The product of two integers \((a-b)\times (c-d)\) is defined by the formula
	\begin{equation*}
		(a-b)\times(c-d)=(ac+bd)-(ad+bc).
	\end{equation*}
\end{definition}
\begin{remark}
	We may identify the integers with natural numbers by setting \(n\equiv n-0\). Definitions of equality and previously defined operations remain consistent with each other.
\end{remark}
\begin{proposition}
	\label{addzero}
	If \(a,b\in\mathbb{Z}\) and \(a+b=b\) then \(a=0\).
\end{proposition}
\begin{lemma}
	Addition and multiplication are well defined.
\end{lemma}
\begin{definition}[Negation of integers]
	If \((a-b)\) is an integer, we define the negation \(-(a-b)\) to be the integer \(b-a\).
\end{definition}
\begin{lemma}[Trichotomy of integers]
	Let \(x\) be an integer. Then either \(x\) is zero, equal to a positive natural number, or \(x\) negated is a positive natural number.
\end{lemma}
\begin{definition}[Positive integer]
	If \(n\) is a positive natural number, we call \(n\) a positive integer, and \(-n\) a negative integer.
\end{definition}
\begin{proposition}[Integer laws for algebra]
	Let \(x,y,z\) be integers. Then the following identities hold:
	\begin{IEEEeqnarray*}{l}
		x+y=y+x\\
		(x+y)+z=x+(y+z)\\
		x+0=0+x=x\\
		x+(-x)=0\\
		xy=yx\\
		(xy)z=x(yz)\\
		1x=x\\
		x(y+z)=xy+xz\\
	\end{IEEEeqnarray*}
\end{proposition}
\begin{proposition}
	\label{posiintmult}
	If \(a,b\in\mathbb{Z}\) with \(a,b>0\), then \(ab>0\).
\end{proposition}
\begin{IEEEproof}
	If \(a,b\in\mathbb{Z}\) with \(a,b>0\), then for some \(x,y\in\mathbb{N}^+\), \(a=x-0\) and \(b=y-0\). Thus \(ab=(xy+0)=(0+0)=xy-0\). Because \(x,y\neq 0\), by proposition \ref{posinatmult} \(xy>0\) so from the definition of a positive integer \(ab>0\).
\end{IEEEproof}
\begin{proposition}
	\label{posiintadd}
	If \(a,b\in\mathbb{Z}\) with \(a,b>0\), then \(a+b>0\).
\end{proposition}
\begin{IEEEproof}
	If \(a,b\in\mathbb{Z}^+\), then for some \(x,y\in\mathbb{N}^+\) we have \(a=x-0\) and \(b=y-0\), so \(a+b=((x+y)-0)\). It follows from proposition \ref{nonzerosum} that \(x+y>0\) so from the definition of a positive integer, \(a+b>0\).
\end{IEEEproof}
\begin{proposition}
	\label{monenegation}
	If \(x\in\mathbb{Z}\) with \(x=(a-b)\) then \(-1\cdot(a-b)=-(a-b)\).
\end{proposition}
\begin{IEEEproof}
	\(-1\cdot (a-b)=(0-1)\cdot(a-b)=(0a+b)-(a+0b)=-(a-b)\).
\end{IEEEproof}
\begin{proposition}[Integers have no zero divisors]
	\label{nozerodiv}
	If \(a,b\) are integers such that \(ab=0\), then \(a=0\) or \(b=0\).
\end{proposition}
\begin{corollary}[Cancellation law]
	If \(a,b,c\) are integers such that \(ac=bc\) and \(c\) is non-zero, then \(a=b\).
\end{corollary}
\begin{IEEEproof}
	Let \(a,b,c\in\mathbb{Z}\) with \(c=\neq 0\). If \(a=0\) it follows from proposition \ref{nozerodiv} that \(ac=0\) so \(bc=0\) and thus \(b=0\), so \(a=b\). If \(a\neq 0\), suppose to the contrary that \(b\neq a\). It follows from proposition \ref{addzero} that there exists \(d\in\mathbb{Z}\) with \(d\neq 0\) such that \(a+d=b\). Using laws for algebra we see that \(ac=ac+dc\). By proposition \ref{nozerodiv} \(dc\neq 0\), a contradiction by proposition \ref{addzero}. Therefore \(a=b\).
\end{IEEEproof}
\begin{definition}[Ordering of integers]
	Let \(n,m\in\mathbb{Z}\). We say that \(n\) is greater than or equal to \(m\) and write \(n\geq m\) or \(m\leq n\) iff we have \(n=m+a\) for some natural number \(a\). We say that \(n\) is strictly greater than \(m\) and write \(n>m\) or \(m<n\) iff \(n\geq m\) and \(n\neq m\).
\end{definition}
\subsection{The Rationals}
\begin{definition}[Rational number]
	A rational number is an expression of the form \(a//b\), where \(a\) and \(b\) are integers and \(b\neq 0\). Two rational numbers are equal, \(a//b=c//d\), iff \(ad=bc\). The set of all rational numbers is denoted \(\mathbb{Q}\).
\end{definition}
\begin{remark}
We may indentify the rationals with natural numbers by setting \(n//1\equiv n\).
\end{remark}
\begin{definition}[Addition of rationals]
	If \(a//b\) and \(c//d\) are rationals, their sum is
	\begin{equation*}
		(a//b)+(c//d)=(ad+bc)//(bd).
	\end{equation*}
\end{definition}
\begin{definition}[Product of rationals]
	If \(a//b\) and \(c//d\) are rationals, their product is
	\begin{equation*}
		(a//b)\cdot(c//d)=(ac)//(bd).
	\end{equation*}
\end{definition}
\begin{definition}[Negation of rationals]
	The negation of a rational \((a//b)\), denoted \(=(a//b)\) is
	\begin{equation*}
		-(a//b)=(-a//b).
	\end{equation*}
\end{definition}
\begin{definition}[Reciprocal of rationals]
	If \(x=a//b\) is a non-zero rational number, then the reciprocal of \(x^{-1}\) of \(x\) is defined
	\begin{equation*}
		x^{-1}=b//a.
	\end{equation*}
\end{definition}
\begin{lemma}
	The sum, product, negation, and reciprocal operations on rational numbers are well-defined.
\end{lemma}
\begin{proposition}
	\label{doublenegation}
	The negation of the negation of \(x\in\mathbb{Q}\) is \(x\).
\end{proposition}
\begin{IEEEproof}
	The negation of the negation of an integer \(x=(a-b)\) is \(--(a-b)=-(b-a)=(a-b)\) so \(--x=x\). The negation of the negation of a rational number \(y=(c//d)\) is \(--(c//d)=-(-c//d)=(--c//d)=c//d\).
\end{IEEEproof}
\begin{definition}[Quotient]
	The quotient of two rationals \(x\) and \(y\) with \(y\neq 0\), denoted \(x/y\), is
	\begin{equation*}
		x/y=x\times y^{-1}.
	\end{equation*}
\end{definition}
\begin{definition}[Subtraction]
	The difference of two rationals \(x\) and \(y\), denoted \(x-y\), is defined
	\begin{equation*}
		x-y=x+(-y).
	\end{equation*}
\end{definition}
\begin{definition}[Positive rational number]
	A rational number \(x\) is said to be positive iff we have \(x=a/b\) for some positive integers \(a\) and \(b\). It is said to be negative iff \(x=-y\) for some positive rational \(y\).
\end{definition}
\begin{definition}[Ordering of rationals]
	Let \(x,y\in\mathbb{Q}\). We say that \(x>y\) iff \(x-y\) is a positive rational number, and \(x<y\) iff \(x-y\) is a positive negative rational number. We write \(x\geq y\) iff either \(x>y\) or \(x=y\), and \(x\leq y\) iff either \(x<y\) or \(x=y\).
\end{definition}
\begin{proposition}
	\label{xgzerop}
	\(x\in\mathbb{Q}\) is positive iff \(x>0\), and negative iff \(x<0\).
\end{proposition}
\begin{IEEEproof}
	If \(x= a//b\) is a positive rational number then \(a,b>0\). Because \(0=0//d\) for some \(d\in\mathbb{N}\setminus\{0\}\), \(x-0=x+0=ad//bd=a//b\) and thus \(x>0\). If \(x>0\) then \(x-0\) is positive. Because \(0=0//d\) for some \(d\in\mathbb{N}\setminus\{0\}\) we have \(x-0=x+0=ad//bd=a//b\), which is positive.
\end{IEEEproof}
\begin{proposition}[Laws of algebra for rationals]
	Let \(x,y,z\) be rationals. Then the following laws of algebra hold:
	\begin{IEEEeqnarray*}{l}
		x+y=y+x\\
		(x+y)+z=x+(y+z)\\
		x+0=x\\
		x+(-x)=0\\
		xy=yx\\
		(xy)z=d(yz)\\
		1x=x\\
		x(y+z)=xy+xz\\
	\end{IEEEeqnarray*}
\end{proposition}
\begin{proposition}
	\label{ronenegation}
	\(-1x=-x\).
\end{proposition}
\begin{IEEEproof}
	If \(x=a//b\) then \(-1\cdot x\) is \((-1//1)\cdot(a//b)=-1a//b\). From proposition \ref{monenegation}, \(-1a=-a\) so \(-1a//b=-(a//b)=-x\).
\end{IEEEproof}
\begin{proposition}
	\label{posiratadd}
	If \(a,b\in\mathbb{Q}\) with \(a>0\) and \(b>0\) then \(a+b>0\).
\end{proposition}
\begin{IEEEproof}
	Suppose \(a,b\in\mathbb{Q}\) with \(a,b>0\). It follows from the definition of positive rational number that for some positive \(x,y,z,w\in\mathbb{Z}\), \(a=x//y\) and \(b=z//w\), so \(ab=xw+zy/yw\). By proposition \ref{posiintmult} \(xw,zy,yw>0\), so by \ref{posiintadd}, \(a+b>0\).
\end{IEEEproof}
\begin{lemma}[Trichotomy of rationals]
	Let \(x\) be a rational number. Then exactly one of the following three statements is true:
	\begin{enumerate}
		\item \(x=0\).
		\item \(x\) is positive.
		\item \(x\) is negative.
	\end{enumerate}
\end{lemma}
\subsection{Absolute Value and Exponentiation}
\begin{definition}[Absolute value]
	If \(x\) is a rational number, the absolute value \(\abs{x}\) of \(x\) is defined as follows:
	\begin{equation*}
		\abs{x}=\begin{cases}
			x,&x\geq 0\\
			-x,&x<0
		\end{cases}
	\end{equation*}
\end{definition}
\begin{definition}[Distance]
	The distance between \(x,y\in\mathbb{Q}\), sometimes denoted \(d(x,y)\), is
	\begin{equation*}
		d(x,y)=\abs{x-y}.
	\end{equation*}
\end{definition}
\begin{proposition}
	For all \(x\in\mathbb{Q}\), \(\abs{x}\geq 0\).
\end{proposition}
\begin{IEEEproof}
	If \(x\geq 0\) then \(\abs{x}=x\) so \(\abs{x}\geq 0\). If \(x<0\) then \(\abs{x}=-x\). By proposition \ref{xgzerop} \(x\) is negative. Therefore there exists \(y\in\mathbb{Q}^+\) such that \(x=-y\), so by proposition \ref{doublenegation} \(-x=--y=y\) and \(-x\) is positive. By proposition \ref{xgzerop}, \(-x>0\).
\end{IEEEproof}
\begin{proposition}[Triangle inequality]
	For \(x,y\in\mathbb{Q}\), \(\abs{x+y}\leq\abs{x}+\abs{y}\).
\end{proposition}
\begin{definition}[\(\epsilon\)-closeness]
	Let \(\epsilon>0\) be a rational number, and let \(x,y\) be rational numbers. We say that \(y\) is \(\epsilon\)-close to \(x\) iff \(d(y,x)<\epsilon\).
\end{definition}
\begin{definition}[Exponentiation to a natural number]
	Let \(x\) be a rational number. To raise \(x\) to the power \(0\), we define \(x^0=1\) and for all \(n\in\mathbb{N}\), \(x^{n+1}=x^n\times x\).
\end{definition}
\begin{definition}[Exponentiation to a negative number]
	Let \(x\) be a non-zero rational number. Then for any negative integer \(-n\),
	\begin{equation*}
		x^{-n}=1/x^n.
	\end{equation*}
\end{definition}
\begin{proposition}
	If \(x\) and \(y\) are two rationals such that \(x<y\), then there exists a third rational \(z\) such that \(x<z<y\).
\end{proposition}
\begin{proposition}
	There does not exists any rational number \(x\) for which \(x^2=2\).
\end{proposition}
\section{Real Numbers}
\subsection{Cauchy Sequences}
\begin{remark}
	Many definitions here are repeated later. Ones given here are necessary for the construction of the real numbers.
\end{remark}
\begin{definition}[Sequences]
	Let \(m\) be an integer. A sequence \((a_n)_{n=m}^{\infty}\) of rational numbers is any function from the set \(\{n\in\mathbb{Z}\,|\,n\geq m\}\) to \(\mathbb{Q}\).
\end{definition}
\begin{definition}[Cauchy sequence]
	A sequence \((a_n)_{n=0}^{\infty}\) of rational numbers is said to be a Cauchy sequence iff
	\begin{equation*}
		\forall\epsilon>0,\,\exists N\in\mathbb{N},\,\forall j,k\in\mathbb{N},\,\big(j,k\geq N\Rightarrow \abs{a_j-a_k}<\epsilon\big).
	\end{equation*}
\end{definition}
\begin{definition}[Bounded sequence]
	Let \(M\geq 0\) be rational. A finite seqeunce \(a_1,a_2,\ldots\) is bounded by \(M\) iff for all \(i\in\mathbb{N}\), \(\abs{a_i}\leq M\).
\end{definition}
\begin{lemma}
	Every finite sequence \(a_1,a_2,\ldots,a_n\) is bounded by some \(M\in\mathbb{Q}\).
\end{lemma}
\begin{definition}[Equivalent sequences]
	Two sequences \((a_n)\) and \((b_n)\) are equivalent iff
	\begin{equation*}
		\forall\epsilon>0,\,\exists N\in\mathbb{N},\,\forall i\in\mathbb{N},\,\big(i,j\geq N\Rightarrow\abs{a_i-b_i}<\epsilon\big).
	\end{equation*}
\end{definition}
\begin{definition}[Real numbers]
	A real number is defined to be an object of the form \(\text{LIM}_{n\rightarrow\infty}a_n\), where \((a_n)_{n=0}^{\infty}\) is a Cauchy sequence of rational numbers. Two real numbers are said to be equivalent if the Cauchy sequences they contain are equivalent. In this context \(\text{LIM}\) is a formal limit.
\end{definition}
\begin{definition}[Real operations]
	Let \(x=\text{LIM}_{n\rightarrow\infty}a_n\) and \(y=\text{LIM}_{n\rightarrow\infty}b_n\). Then
	\begin{IEEEeqnarray*}{l}
		x+y=\text{LIM}_{n\rightarrow\infty}(a_n+b_n),\\
		xy=\text{LIM}_{n\rightarrow\infty}(a_nb_n),\\
		x^{-1}=\text{LIM}_{n\rightarrow\infty}{a_n^{-1}}\\
		x/y=x\cdot y^{-1},\;y\neq 0.
	\end{IEEEeqnarray*}
\end{definition}
\begin{definition}[Bounded away from zero]
	A sequence \((a_n)\) is said to be bounded away from zero iff there exists a rational number \(c>0\) such that \(\abs{a_n}\geq c\) for all \(n\geq 1\).
\end{definition}
\begin{definition}[Positive real number]
	A real number \(x\) is said to be positive iff it can be written as a real number for some Cauchy sequence positively bounded away from zero.
\end{definition}
\begin{definition}[Absolute value]
	Let \(x\) be a real number. We define the absolute value \(\abs{x}\) of \(x\) to equal \(x\) if \(x\) is positive, \(-x\) when \(x\) is negative, and \(0\) when \(x\) is zero.
\end{definition}
\begin{definition}[Ordering of reals]
	Let \(x\) and \(y\) be real numbers. We say that \(x\) is greater than \(y\) iff \(x-y\) is a positive real number, and \(x<y\) if \(x-y\) is a negative real number. We define \(x\geq y\) iff \(x>y\) or \(x=y\).
\end{definition}
\begin{definition}[Archemedian property]
	Let \(x\) be a real number, and let \(\epsilon\) be a positive real number. Then there exists a positive integer \(M\) such that \(M\epsilon>x\).
\end{definition}
\begin{definition}[Real exponentiation by an integer]
	Let \(x\) be a real number. Then
	\begin{IEEEeqnarray*}{l}
		x^0=1;\\
		x^{n+1}=x^n\cdot x.
	\end{IEEEeqnarray*}
	If \(n\in\mathbb{Z}\) and \(x\in\mathbb{R}\) is nonzero, we define
	\begin{equation*}
		x^{-n}=1/x^n.
	\end{equation*}
\end{definition}
\begin{definition}[\(n\)th root]
	Let \(x\geq 0\) be a non-negative real, and let \(n\geq 1\) be a positive integer. We define \(x^{1/n}\) as
	\begin{equation*}
		x^{1/n}=\sup\{y\in\mathbb{R}\,|\,y\geq 0\wedge y^n\leq x\}.
	\end{equation*}
\end{definition}
\begin{lemma}
	\(x^{1/n}\) is a real number.
\end{lemma}
\begin{definition}[Rational exponents]
	Let \(x>0\) be a positive real number, and let \(q=a/b\) be a rational number. Then
	\begin{equation*}
		x^q=(x^{1/b})^a.
	\end{equation*}
\end{definition}
\section{Sequences}
\subsection{Sequences}
\begin{definition}[Sequence]
	A sequence is a function whose domain is the set \(\mathbb{N}\) of natural numbers, and can denoted \((s_n)_{n=a}^{b}\) for \(a\in\mathbb{N},b\in\mathbb{N}\cup\{\infty\}\). \((s_n)\) will be used here as shorthand for \((s_n)_{n=0}^{\infty}\).
\end{definition}
\begin{definition}[Limit of a sequence]
	A sequence \((s_n)\) is said to converge to \(s\in\mathbb{R}\) iff
	\begin{equation*}
		\forall\epsilon>0,\,\exists N\in\mathbb{N},\,\forall n\in\mathbb{N},\big(n\geq N\Rightarrow\abs{s_n-s}<\epsilon\big).
	\end{equation*}
	If \((s_n)\) converges to \(s\), then \(s\) is called the limit of the sequence \((s_n)\), and we write \(\lim_{n\rightarrow\infty}s_n=s\). If a sequence does not converge, it is said to diverge.
\end{definition}
\begin{definition}[Real exponentiation]
	Let \(x>0\) be real, and let \(\alpha\) be a real number. We define the quantity \(x^{\alpha}\) by
	\begin{equation*}
		x^{\alpha}=\lim_{n\rightarrow\infty}x^{q_n},
	\end{equation*}
	where \((q_n)_{n=0}^{\infty}\) is any sequence of rational numbers converging to \(\alpha\).
\end{definition}
\begin{lemma}[Continuity of exponentiation]
	Let \(x>0\), and let \(\alpha\) be a real number. Let \((q_n)_{n=1}^{\infty}\) be any sequence of rational numbers converging to \(\alpha\). Then \((x^{q_n})_{n=1}^{\infty}\) is also a convergent sequence. Furthermore, if \((q'_n)_{n=1}^{\infty}\) is a sequence converging to \(\alpha\), then \((x^{q'_n})_{n=1}^{\infty}\) has the same limit as \((x^{q_n})_{n=1}^{\infty}\).
\end{lemma}
\begin{definition}[Divergence to Infinity]
	A sequence \((s_n)\) is said to diverge to infinity, and we write \(\lim_{n\rightarrow\infty}s_n=\infty\) if
	\begin{equation*}
		\forall M\in\mathbb{R},\,\exists N\in\mathbb{N},\,\forall n\in\mathbb{N},\,\big(n\geq N\Rightarrow s_n>M\big).
	\end{equation*}
\end{definition}
\begin{definition}[Bounded sequence]
	A sequence \((s_n)\) is said to be bounded if
	\begin{equation*}
		\exists M\geq 0,\forall n\in\mathbb{N},\,\big(\abs{s_n}\leq M\big).
	\end{equation*}
\end{definition}
\begin{theorem}
	If a sequence converges, it is bounded.
\end{theorem}
\begin{theorem}
	If a sequence converges, its limit is unique.
\end{theorem}
\subsection{Monotone and Cauchy sequences}
\begin{theorem}
Suppose that \((s_n)\) and \((t_n)\) are convergent sequences with \(\lim_{n\rightarrow\infty} s_n=s\) and \(\lim_{n\rightarrow\infty} t_n=t\). Then
	\begin{enumerate}
		\item \(\lim s_n+t_n=s+t\).
		\item \(\lim ks_n=ks\).
		\item \(\lim s_nt_n=st\).
		\item \(\lim s_n/t_n=s/t\quad\text{iff }t\neq 0\text{ and }\forall n\in\mathbb{N},\,t_n\neq 0.\)
	\end{enumerate}
\end{theorem}
\begin{theorem}
	Let \((s_n)\) be a sequence of positive numbers. Then \(\lim_{n\rightarrow\infty}s_n=\infty\) iff \(\lim_{n\rightarrow\infty}1/s_n=0\).
\end{theorem}
\begin{definition}[Increasing sequence]
	A sequence \((s_n)\) of real numbers is increasing if \(s_n\leq s_{n+1}\) for all \(n\in\mathbb{N}\). A sequence is monotone if it is increasing or decreasing.
\end{definition}
\begin{theorem}
	A monotone sequence is convergent iff it is bounded.
\end{theorem}
\begin{definition}[Cauchy sequence]
	A sequence \((s_n)\) of real numbers is a Cauchy sequence if
	\begin{equation*}
		\forall\epsilon>0,\,\exists N\in\mathbb{N},\,\forall n,m\in\mathbb{N},\,\big(m,n\geq N\Rightarrow \abs{s_n-s_m}<\epsilon\big).
	\end{equation*}
\end{definition}
\begin{lemma}
	Every convergent sequence is a Cauchy sequence
\end{lemma}
\begin{lemma}
	Every Cauchy sequence is bounded.
\end{lemma}
\begin{theorem}
	A sequence of real numbers is convergent iff it is a Cauchy sequence.
\end{theorem}
\begin{definition}[Subsequence]
	Let \((s_n)_{n=1}^{\infty}\) be a sequence and let \((n_k)_{k=1}^{\infty}\) be any sequence of natural numbers such that \(n_1<n_2<n_2<\ldots\). The sequence \((s_{n_k})_{k=1}^{\infty}\) is called a subsequence of \((s_n)_{n=1}^{\infty}\).
\end{definition}
\begin{theorem}
	If a sequence converges to a real number \(s\), then every subsequence of \((s_n)\) also converges to \(s\).
\end{theorem}
\begin{theorem}
	Every bounded sequence has a convergent subsequence.
\end{theorem}
\subsection{Limit superior and inferior}
\begin{definition}[Limsup and liminf]
	A subsequential limit of \((s_n)\) is any real number that is the limit of some subsequence of \((s_n)\). If \(S\) is the set of all subsequential limits of \((s_n)\), then we define the limit superior of \((s_n)\) to be
	\begin{equation*}
		\lim\sup s_n=\sup S.
	\end{equation*}
	The limit inferior is defined
	\begin{equation*}
		\lim\inf s_n=\inf S.
	\end{equation*}
\end{definition}
\begin{theorem}
	Let \((s_n)\) be a bounded sequence and let \(m=\lim\sup s_n\). Then the following properties hold:
	\begin{enumerate}
		\item For every \(\epsilon>0\) there exists a natural number \(N\) such that \(n\geq N\) implies that \(s_n<m+\epsilon\).
		\item For every \(\epsilon>0\) there exists an integer \(k>i\) such that \(s_k>m-\epsilon\).
	\end{enumerate}
\end{theorem}
\begin{theorem}
	Suppose that \((r_n)\) converges to a positive number \(r\) and \((s_n)\) is a bounded sequence. Then
	\begin{equation*}
		\lim\sup r_ns_n=r\lim\sup s_n
	\end{equation*}
\end{theorem}
\section{Series}
\subsection{Convergence tests}
\begin{definition}[Convergence of series]
	Let \(\sum_{n=m}^{\infty}a_n\) be a formal infinite series. For any integer \(N\geq m\), we define the \(N\)th partial sum \(S_N\) of this series to be
	\begin{equation*}
		S_N=\sum_{n=m}^{N}a_n.
	\end{equation*}
	If the sequence \((S_N)_{N=m}^{\infty}\) converges to \(L\), then we say that the infinite series \(\sum_{n=m}^{\infty}a_n\) is convergent, and converges to \(L\). We also write \(L=\sum_{n=m}^{\infty}a_n=L\). If the parial sums \(S_N\) diverge, we say the infinite series \(\sum_{n=m}^{\infty}a_n\) is divergent, and do not assign any real number to it.
\end{definition}
\begin{proposition}
	Let \(\sum_{n=m}^{\infty}a_n\) be a formal series of real numbers. \(\sum_{n=m}^{\infty}a_n\) converges iff for every real number \(\epsilon>0\), there exists an integer \(N\geq m\) such that for all \(p,q\geq N\),
	\begin{equation*}
		\abs{\sum_{n=p}^{q}a_n}\leq\epsilon.
	\end{equation*}
\end{proposition}
\begin{corollary}
	Let \(\sum_{n=m}^{\infty}a_n\) be a convergent series of real numbers. Then we must have \(\lim_{n\rightarrow\infty}a_n=0\).
\end{corollary}
\begin{definition}[Absolute convergence]
	Let \(\sum_{n=m}^{\infty}a_n\) be a formal series of real numbers. We say that this series is absolutely convergent iff the series \(\sum_{n=m}^{\infty}\abs{a_n}\) is convergent.
\end{definition}
\begin{proposition}
	Let \(\sum_{n=m}^{\infty}a_n\) be a formal series of real numbers. If the series is absolutely convergent, then it is also convergent. Furthermore,
	\begin{equation*}
		\abs{\sum_{n=m}^{\infty}a_n}\leq\sum_{n=m}^{\infty}\abs{a_n}.
	\end{equation*}
\end{proposition}
\begin{proposition}[Alternating series test]
	Let \((a_n)_{n=m}^{\infty}\) be a sequence of real numbers which are non-negative and decreasing, thus \(a_n\geq 0\) and \(a_n\geq a_{n+1}\) for every \(n\geq m\). Then the series
	\begin{equation*}
		\sum_{n=m}^{\infty}(-1)^na_n
	\end{equation*}
	is convergent iff the sequence \(a_n\) converges to \(0\) as \(n\rightarrow\infty\).
\end{proposition}
\begin{IEEEproof}
	The sequence of partial sums is Cauchy, thus converges.
\end{IEEEproof}
\begin{proposition}
	Let \(\sum_{n=m}^{\infty}a_n\) be a formal series of non-negative real numbers. Then this series is convergent iff there is a real number \(M\) such that for all \(N\geq m\),
	\begin{equation*}
		\sum_{n=m}^{N}a_n\leq M.
	\end{equation*}
\end{proposition}
\begin{corollary}[Comparison test]
	Let \(\sum_{n=m}^{\infty}a_n\) and \(\sum_{n=m}^{\infty}b_n\) be two formal series of real numbers, and suppose that \(\abs{a_n}\leq b_n\) for all \(n\geq m\). Then if \(\sum_{n=m}^\infty b_n\) is convergent, then \(\sum_{n=m}^{\infty}a_n\) is absolutely convergent, and
	\begin{equation*}
		\abs{\sum_{n=m}^{\infty}a_n}\leq\sum_{n=m}^{\infty}\abs{a_n}\leq\sum_{n=m}^{\infty}b_n.
	\end{equation*}
\end{corollary}
\begin{definition}[Geometric series]
	The geometric series is defined
	\begin{equation*}
		\sum_{n=0}^{\infty}x^n,
	\end{equation*}
	Where \(x\in\mathbb{R}\).
\end{definition}
\begin{lemma}
	Let \(x\) be a real number. If \(\abs{x}\geq 1\), then the series \(\sum_{n=0}^{\infty}x^n\) is divergent. If \(\abs{x}<1\), then the series is absolutely convergent, and
	\begin{equation*}
		\sum_{n=0}^{\infty}x^n=\frac{1}{1-x}.
	\end{equation*}
\end{lemma}
\begin{proposition}
	Let \((a_n)_{n=1}^{\infty}\) be a decreasing sequence of non-negative real numbers. Then the series \(\sum_{n=1}^{\infty}a_n\) is convergent iff
	\begin{equation*}
		\sum_{k=0}^{\infty}2^ka_{2^k}
	\end{equation*}
	is convergent.
\end{proposition}
\begin{corollary}
	Let \(q>0\) be a real number. Then the series \(\sum_{n=1}^{\infty}\frac{1}{n^q}\) is convergent when \(q>1\) and divergent when \(q\leq 1\).
\end{corollary}
\begin{theorem}[Root test]
	Let \(\sum_{n=m}^{\infty}a_n\) be a series of real numbers, and let \(\alpha=\lim\sup\abs{a_n}^{1/n}\).
	\begin{enumerate}
		\item If \(\alpha<1\), then the series \(\sum_{n=m}^{\infty}a_n\) is absolutely convergent.
		\item If \(\alpha>1\), then the series \(\sum_{n=m}^{\infty}a_n\) is not convergent.
		\item If \(\alpha=1\), we cannot assert any conclusion.
	\end{enumerate}
\end{theorem}
\section{Topology Shit}
\subsection{Heine-Borel Theorem}
\begin{definition}
	Let \(\epsilon>0\). A neighborhood of \(x\) is a set of the form
	\begin{equation*}
		N(x;\epsilon)=\{y\in\mathbb{R}\,|\,\abs{x-y}<\epsilon\},
	\end{equation*}
	where \(\epsilon\) is referred to as the radius.
\end{definition}
\begin{definition}[Deleted neighborhood]
	Let \(x\in\mathbb{R}\) and \(\epsilon>0\). A deleted neighborhood of \(x\) is the set
	\begin{equation*}
		N^*(x;\epsilon)=N(x;\epsilon)\setminus\{x\},
	\end{equation*}
	i.e.
	\begin{equation*}
		N^*(x;\epsilon)=\{y\in\mathbb{R}\,|\,0<\abs{x-y}<\epsilon\}.
	\end{equation*}
\end{definition}
\begin{definition}[Interior point]
	Let \(S\subseteq\mathbb{R}\). A point \(x\in\mathbb{R}\) is an interior point of \(S\) if there exists a neighborhood \(N\) of \(x\) such that \(N\subseteq S\).
\end{definition}
\begin{definition}[Boundary point]
	If for every neighorhood \(N\) of \(x\) we have \(N\cap S\neq\emptyset\) and \(N\cap(\mathbb{R}\setminus S)\neq\emptyset\), then \(x\) is a boundary point of \(S\).
\end{definition}
\begin{definition}[Adherent point]
	Let \(X\subseteq\mathbb{R}\), and let \(y\in\mathbb{R}\). We say that \(y\) is an adherent point of \(X\) iff
	\begin{equation*}
		\forall\epsilon>0,\,\exists x\in X,\,\big(\abs{x-y}<\epsilon\big).
	\end{equation*}
\end{definition}
\begin{definition}[Accumulation point]
	Let \(S\subseteq\mathbb{R}\). A point \(x\in\mathbb{R}\) is an accumulation point of \(S\) if every deleted neighborhood of \(x\) contains a point of \(S\).
\end{definition}
\begin{definition}[Isolated point]
	We say that \(x\) is an isolated point of \(X\) if \(x\in X\) and there exists some \(\epsilon>0\) such that \(\abs{x-y}>\epsilon\) for all \(y\in X\setminus\{x\}\).
\end{definition}
\begin{definition}[Closure]
	Let \(X\subseteq\mathbb{R}\). The closure of \(X\), denoted \(\overbar{X}\) is defined to be the set of all adherent points of \(X\).
\end{definition}
\begin{lemma}
	Let \(X\subseteq\mathbb{R}\). The set of all convergent points of sequences in \(X\) is the closure of \(X\).
\end{lemma}
\begin{theorem}[Heine-Borel]
	Let \(X\) be a subset of \(\mathbb{R}\). Then the following statements are equivalent:
	\begin{enumerate}
		\item \(X\) is closed and bounded.
		\item Given any sequence \((a_n)_{n=0}^{\infty}\) of real numbers which takes values in \(X\), there exists a subsequence \((a_{n_j})_{j=0}^{\infty}\) of the original sequence which converges to some number \(L\) in \(X\).
	\end{enumerate}
\end{theorem}
\section{Continuous Functions on \(\mathbb{R}\)}
\subsection{Limits of Functions}
\begin{definition}[Extended real numbers]
	The extended real number system \(\mathbb{R}\cup\{\infty,-\infty\}\) is denoted \(\mathbb{R}^*\). An extended real number is said to be finte iff it is a real number, and infinite iff it is equal to \(\pm\infty\).
\end{definition}
\begin{definition}[Intervals]
	Let \(a,b\in\mathbb{R}^*\). Then the closed interval \([a,b]\) is the set
	\begin{equation*}
		\{x\in\mathbb{R}^*\,|\,a\leq x\leq b\}.
	\end{equation*}
	The open interval \((a,b)\) is the set
	\begin{equation*}
		\{x\in\mathbb{R}^*\,|\,a<x<b\}.
	\end{equation*}
\end{definition}
\begin{definition}[Limit point]
	Let \(X\subseteq\mathbb{R}\). We say that \(x\in\mathbb{R}\) is a limit point of \(X\) iff it is an adherent point of \(X\setminus\{x\}\).
\end{definition}
\begin{definition}[Algebra of functions]
	Given two functions \(f:X\rightarrow\mathbb{R}\) and \(g:X\rightarrow\mathbb{R}\):
	\begin{IEEEeqnarray*}{rCl}
		(f+g)(x)&=&f(x)+g(x),\\
		(f-g)(x)&=&f(x)-f(x),\\
		(fg)(x)&=&f(x)g(x),\\
		(f/g)(x)&=&f(x)/g(x),\\
		(cf)(x)&=&cf(x).
	\end{IEEEeqnarray*}
\end{definition}
\begin{definition}[Limit of a function]
	Let \(f:D\rightarrow\mathbb{R}\) and let \(c\) be an accumulation point of \(D\). We say that a real number \(L\) is a limit of \(f\) at \(c\) if
	\begin{equation*}
		\forall\epsilon>0,\,\exists\delta>0,\,\forall x\in D,\,\big(0<\abs{x-c}<\delta\Rightarrow\abs{f(x-L)}<\epsilon\big).
	\end{equation*}
\end{definition}
\begin{remark}
	The effect of \(c\) being an accumulation point is that limits must be unique. If \(c\) wasn't an accumulation point of \(D\), it would be vacuously true that every number is a limit for \(f\) at \(c\). \(\abs{x-c}>0\) specifies that we are focusing on \(f\)'s approach to \(L\) as \(x\) approaches \(c\), and not on the value of \(f\) at \(c\).
\end{remark}
\begin{theorem}
	Let \(f:D\rightarrow\mathbb{R}\) and let \(c\) be an accumulation point of \(D\). Then \(\lim_{x\rightarrow c}f(x)=L\) iff for each neighborhood \(V\) of \(L\) there exists a deleted neighborhood \(U\) of \(c\) such that \(f(U\cap D)\subseteq V\).
\end{theorem}
\begin{theorem}
	Let \(f:D\rightarrow\mathbb{R}\) and let \(c\) be an accumulation point of \(D\). Then \(\lim_{x\rightarrow c}f(x)=L\) iff for every sequence \((s_n)\) in \(D\) that converges to \(c\) with \(s_n\neq c\) for all \(n\), the sequence \((f(s_n))\) converges to \(L\).
\end{theorem}
\begin{corollary}
	Limits are unique.
\end{corollary}
\begin{theorem}
	Let \(f:D\rightarrow\mathbb{R}\) and \(g:D\rightarrow\mathbb{R}\), and let \(c\) be an accumulation point of \(D\). If \(\lim_{x\rightarrow c}f(x)=L\), and \(\lim_{x\rightarrow c}g(x)=M\) and \(k\in\mathbb{R}\), then
	\begin{IEEEeqnarray*}{l}
		\lim_{x\rightarrow c}(f+g)(x)=L+M,\\
		\lim_{x\rightarrow c}(fg)(x)=LM,\\
		\lim_{x\rightarrow c}(f/g)(x)=L/M,\text{ if }\forall x\in D,\,g(x)\neq 0\text{ and }M\neq 0.
	\end{IEEEeqnarray*}
\end{theorem}
\begin{definition}[Right and Left limits]
	Suppose \(f:D\rightarrow\mathbb{R}\) with \(c\) an accumulation point of \(D\) and \(\lim_{x\rightarrow c}f(x)=L\) for some \(L\in\mathbb{R}\). The right hand limit of \(f\) at \(c\) is the limit of \(f\) restricted to some domain \((c,d)\) with \(d>c\) as \(x\rightarrow c\). The left hand limit of \(f\) at c is the limit of \(f\) restricted to some domain \((-d,c)\) as \(x\rightarrow c\).
\end{definition}
\subsection{Continuous functions}
\begin{definition}[Continuity]
	Let \(f:D\rightarrow\mathbb{R}\) and let \(c\in D\). We say that \(f\) is continuous at \(c\) iff
	\begin{equation*}
		\forall\epsilon>0,\,\exists\delta>0,\,\forall x\in D,\,\big(\abs{x-c}<\delta\Rightarrow\abs{f(x)-f(c)}<\epsilon\big).
	\end{equation*}
	If \(f\) is continuous at each point of a subset \(S\) of \(D\), then \(f\) is said to be continuous on \(S\). If \(f\) is continuous on its domain \(D\), then \(f\) is said to be a continuous function.
\end{definition}
\begin{theorem}
	Let \(f:D\rightarrow\mathbb{R}\) and let \(c\in D\). Then the following three conditions are equivalent:
	\begin{enumerate}
		\item \(f\) is continuous at \(c\).
		\item If \((x_n)\) is any sequence in \(D\) such that \((x_n)\) converges to \(c\), then \(\lim_{n\rightarrow\infty}f(x_n)=f(c)\).
		\item For every neighborhood \(V\) of \(f(c)\) there exists a neighborhood \(U\) of \(c\) such that \\\(f(U\cap D)\subseteq V\).
	\end{enumerate}
\end{theorem}
\begin{theorem}
	Let \(f\) and \(g\) be functions from \(D\) to \(\mathbb{R}\), and let \(c\in D\). Suppose that \(f\) and \(g\) are continuous at \(c\). Then
	\begin{enumerate}
		\item \(f+g\) and \(fg\) are continuous at \(c\).
		\item \(f/g\) is continuous at \(c\) if \(g(c)\neq 0\).
	\end{enumerate}
\end{theorem}
\begin{theorem}
	Let \(F:D\rightarrow\mathbb{R}\) and \(g:E\rightarrow\mathbb{R}\) be functions such that \(f(D)\subseteq E\). If \(f\) is continuous at a point \(c\in D\) and \(g\) is continuous at \(f(c)\), then the composition \(g\circ f:D\rightarrow\mathbb{R}\) is continuous at \(c\).
\end{theorem}
\begin{theorem}
	A function \(f:D\rightarrow\mathbb{R}\) is continuous on \(D\) iff for every open set \(G\) in \(\mathbb{R}\) there exists an open set \(H\) such that \(H\cap D=f^{-1}(G)\).
\end{theorem}
\begin{corollary}
	A function \(f:\mathbb{R}\rightarrow\mathbb{R}\) is continuous iff \(f^{-1}(G)\) is open in \(\mathbb{R}\) whenever \(G\) is open in \(\mathbb{R}\).
\end{corollary}
\begin{definition}[Lipschitz continuity]
	If \(\abs{f(x)-f(y)}\leq M\abs{x-y}\) for some \(M>0\), the function is called Lipschitz continuous.
\end{definition}
\subsection{Properties of continous functions}
\begin{theorem}
	Let \(D\) be a compact subset of \(\mathbb{R}\) and suppose that \(f:D\rightarrow\mathbb{R}\) is continuous. Then \(f(D)\) is compact.
\end{theorem}
\begin{corollary}
	Let \(D\) be a compact subset of \(\mathbb{R}\), and suppose that \(f:D\rightarrow\mathbb{R}\) is continuous. Then \(f\) assumes minimum and maximum values on \(D\).
\end{corollary}
\begin{theorem}[Intermediate value theorem]
	Suppose that \(f:[a,b]\rightarrow\mathbb{R}\) is continuous. Then if \(f(a)<k<f(b)\) or \(f(b)<k<f(a)\), then there exists \(c\in(a,b)\) such that \(f(c)=k\).
\end{theorem}
\begin{theorem}
	Let \(I\) be a compact interval, and suppose that \(f:I\rightarrow\mathbb{R}\) is a continuous function. Then the set \(f(I)\) is a compact interval.
\end{theorem}
\subsection{Uniform continuity}
\begin{definition}
	Let \(f:D\rightarrow\mathbb{R}\). We say that \(f\) is uniformly continuous on \(D\) if
	\begin{equation*}
		\forall\epsilon>0,\,\exists\delta>0,\,\forall x,y\in D,\,\big(\abs{x-y}<\delta\Rightarrow\abs{f(x)-f(y)}<\epsilon\big).
	\end{equation*}
\end{definition}
\begin{theorem}
	Suppose \(f:D\rightarrow\mathbb{R}\) is continuous on a compact set \(D\). Then \(f\) is uniformly continuous on \(D\).
\end{theorem}
\begin{theorem}
	Let \(f:D\rightarrow\mathbb{R}\) be uniformly continuous on \(D\) and suppose that \((x_n)\) is a Cauchy sequence in \(D\). Then \((f(x_n))\) is a Cauchy sequence.
\end{theorem}
\begin{theorem}
	A function \(f:(a,b)\rightarrow\mathbb{R}\) is uniformly continuous on \((a,b)\) iff it can be extended to a function \(\overbar{f}\) that is continuous on \([a,b]\).
\end{theorem}
\section{Differentiation}
\subsection{Differentiation}
\begin{definition}[Differentiation]
	Let \(I\) be an interval containing a point \(c\), and let \(f:I\rightarrow\mathbb{R}\). We say that \(f\) is differentiable at \(c\) if the limit
	\begin{equation*}
		\lim_{x\rightarrow c}\frac{f(x)-f(c)}{x-c}
	\end{equation*}
	exists and is finite. We denote the derivative of \(f\) at \(c\) by \(f'(c)\). If \(f\) is differentiable at each point of the set \(S\subseteq I\), then \(f\) is said to be differentiable on \(S\), and the function \(f':S\rightarrow\mathbb{R}\) is called the derivative of \(f\) on \(S\).
\end{definition}
\begin{theorem}
	If \(f:I\rightarrow\mathbb{R}\) is differentiable at a point \(c\in I\), then \(f\) is continuous at \(c\).
\end{theorem}
\begin{IEEEproof}
	Let \(F:I\rightarrow\mathbb{R}\) be differentiable at \(c\in I\). Then
	\begin{equation}
		\lim_{x\rightarrow c}\frac{f(x)-f(c)}{x-c}=f'(c)
	\end{equation}
	for some \(f'(c)\in\mathbb{R}\). If \(f\) is not continuous at \(c\), then
	\begin{equation*}
		\exists\epsilon>0,\,\forall\delta>0,\,\exists x\in I,\,\big(\abs{x-c}<\delta\wedge
		\abs{f(x)-f(c)}\geq\epsilon\big).
	\end{equation*}
	Thus there exists \(x\) with \(\abs{x-c}<\delta\), so
\begin{equation*}
	\abs{\frac{f(x)-f(c)}{x-c}}=\frac{\abs{f(x)-f(c)}}{\abs{x-c}}\geq\frac{\epsilon}{\delta}.
\end{equation*}
	Because \(\epsilon/\delta\) is arbitrarily large for small \(\delta\),
\begin{equation*}
	\bigg\{\frac{f(x)-f(c)}{x-c}\;\bigg|\;\abs{x-c}<\delta\bigg\}
\end{equation*}
	is unbounded for all \(\delta>0\).
\end{IEEEproof}
\begin{theorem}
	Suppose that \(f:I\rightarrow\mathbb{R}\) and \(g:I\rightarrow\mathbb{R}\) are differentiable at \(c\in I\). Then the following identities hold:
	\begin{enumerate}
		\item For \(k\in\mathbb{R}\), \(kf\) is differentiable at \(c\) and
			\begin{equation*}
				(kf')(c)=k\cdot f'(c).
			\end{equation*}
		\item The function \(f+g\) is differentiable at \(c\) and
			\begin{equation*}
				(f+g)'(c)=f'(c)+g'(c).
			\end{equation*}
		\item The function \(fg\) is differentiable at \(c\) and
			\begin{equation*}
				(fg)'(c)=f(c)g'(c)+g(c)f'(c).
			\end{equation*}
		\item If \(g(c)\neq 0\), the function \(f/g\) is differentiable at \(c\) and
			\begin{equation*}
				(f/g)'(c)=\frac{g(c)f'(c)-f(c)g'(c)}{[g(c)]^2}.
			\end{equation*}
	\end{enumerate}
\end{theorem}
\subsection{Differentiation theorems}
\begin{theorem}[Chain rule]
	Let \(I\) and \(J\) be intervals in \(\mathbb{R}\), \(f:I\rightarrow\mathbb{R}\), and \(g:J\rightarrow\mathbb{R}\), with \(f(I)\subseteq J\) and \(c\in I\). If \(f\) is differentiable at \(c\) and \(g\) is differentiable at \(f(c)\), then \(g\circ f\) is differentiable at \(c\) and
	\begin{equation*}
		(g\circ f)'(c)=g'(f(c))\cdot f'(c).
	\end{equation*}
\end{theorem}
\begin{theorem}
	If \(f\) is differentiable on an open interval \((a,b)\) and if \(f\) assumes it's maximum or minimum at a point \(c\in (a,b)\), then \(f'(c)=0\).
\end{theorem}
\begin{theorem}[Rolle's theorem]
	Let \(f\) be a continuous function on \([a,b]\) that is differentiable on \((a,b)\) and such that \(f(a)=f(b)\). Then there exists at least one point \(c\) in \((a,b)\) such that \(f'(c)=0\).
\end{theorem}
\begin{theorem}[Mean value theorem]
	Let \(f\) be a continuous function on \([a,b]\) that is differentiable on \((a,b)\). Then there exists at least one point \(c\in(a,b)\) such that
	\begin{equation*}
		f'(c)=\frac{f(b)-f(a)}{b-a}.
	\end{equation*}
\end{theorem}
\begin{theorem}[IVT for derivatives]
	Let \(f\) be differentiable on \([a,b]\) and suppose that \(k\) is a number between \(f'(a)\) and \(f'(b)\). Then there exists a point \(c\in(a,b)\) such that \(f'(c)=k\).
\end{theorem}
\begin{theorem}
	Suppose that \(f\) is differentiable on an interval \(I\) and \(f'(x)\neq 0\) for all \(x\in I\). Then \(f\) is injective, \(f^{-1}\) is differentiable of \(f(I)\), and
	\begin{equation*}
		(f^{-1})'(y)=\frac{1}{f'(x)},
	\end{equation*}
	where \(y=f(x)\).
\end{theorem}
\begin{theorem}[Cauchy mean value theorem]
	Let \(f\) and \(g\) be functions that are continuous on \([a,b]\) and differentiable on \((a,b)\). Then there exists at least one point \(c\in(a,b)\) such that
	\begin{equation*}
		[f(b)-f(a)]g'(c)=[g(b)-g(a)]f'(c).
	\end{equation*}
\end{theorem}
\begin{theorem}[L'Hospital's rule]
	Let \(f\) and \(g\) be continuous on \([a,b]\) and differentiable on \((a,b)\). Suppose that \(c\in[a,b]\) and that \(f(c)=g(c)=0\). Suppose also that \(g'(x)\neq 0\) for \(x\in U\), where \(U\) is the intersection of \((a,b)\) and some deleted neighborhood of \(c\). If
	\begin{equation*}
		\lim_{x\rightarrow c}\frac{f'(x)}{g'(x)}=L,\;L\in\mathbb{R},
	\end{equation*}
	Then
	\begin{equation*}
		\lim_{x\rightarrow c}\frac{f(x)}{g(x)}=L.
	\end{equation*}
\end{theorem}
\begin{definition}[Limit at infinity]
	Let \(f:(a,\infty)\rightarrow\mathbb{R}\). We say that \(L\in\mathbb{R}\) is the limit of \(f\) as \(x\rightarrow\infty\), and write
	\begin{equation*}
		\lim_{x\rightarrow\infty}f(x)=L,
	\end{equation*}
	if
	\begin{equation*}
		\forall\epsilon>0,\,\exists N>a,\,\forall x\in(a,\infty),\,\big(x>N\Rightarrow\abs{f(x)-L}<\epsilon\big).
	\end{equation*}
\end{definition}
\begin{definition}
	Let \(f:(a,\infty)\rightarrow\mathbb{R}\). We say that \(f\) tends to \(\infty\) as \(x\rightarrow\infty\) and write
	\begin{equation*}
		\lim_{x\rightarrow\infty}f(x)=\infty,
	\end{equation*}
	if
	\begin{equation*}
		\forall\alpha\in\mathbb{R},\,\exists N>a,\,\forall x\in(a,\infty),\,\big(x>N\Rightarrow f(x)>\alpha\big).
	\end{equation*}
\end{definition}
\begin{theorem}[L'Hospital's rule]
	Let \(f\) and \(g\) be differentiable on \(a,\infty\). Suppose that \(\lim_{x\rightarrow\infty}f(x)=\lim_{x\rightarrow\infty}g(x)=\infty\), and that \(g'(x)\neq 0\) for \(x\in (a,\infty)\). If
	\begin{equation*}
		\lim_{x\rightarrow\infty}\frac{f'(x)}{g'(x)}=L,\;L\in\mathbb{R},
	\end{equation*}
	then
	\begin{equation*}
		\lim_{x\rightarrow\infty}\frac{f(x)}{g(x)}=L.
	\end{equation*}
\end{theorem}
\begin{theorem}[Taylor's theorem]
	Let \(f\) and its first \(n\) derivatives be continuous on \([a,b]\) and differentiable on \((a,b)\), and let \(x_0\in[a,b]\). Then for each \(x\in[a,b]\) with \(x\neq x_0\) there exists a point \(c\) between \(x\) and \(x_0\) such that
	\begin{equation*}
		f(x)=\sum_{n=0}^{k}\frac{f^{(n)}(x_0)}{n!}(x-x_0)^n+\frac{f^{(k+1)}(c)}{(k+1)!}(x-x_0)^{k+1}.
	\end{equation*}
\end{theorem}
\section{Integration}
\subsection{Piecewise Constant Integrals}
\begin{definition}[Connected sets]
	Let \(X\) be a subset of \(\mathbb{R}\). We say that \(X\) is connected iff \(X\) is non-empty and whenever \(x,y\in X\) with \(x<y\), the bounded interval \([x,y]\) is a subset of \(X\).
\end{definition}
\begin{definition}[Length of interval]
	If \(I\) is a bounded interval, the length of \(I\), denoted \(\abs{I}\), is defined as follows: If \(I\) is one of the intervals \([a,b],(a,b),[a,b),(a,b]\) for some real numbers \(a<b\), then
	\begin{equation*}
		\abs{I}=b-a.
	\end{equation*}
	If \(I\) is a point or the empty set, \(abs{I}=0\).
\end{definition}
\begin{definition}[Partition]
	Let \(I\) be a bounded interval. A partition of \(I\) is a finite set \(P\) of bounded intervals contained in \(I\) such that \(\bigcup P=I\) and \(\bigcap P=\emptyset\).
\end{definition}
\begin{theorem}
	Let \(I\) be a bounded interval, \(n\) be a natural number, and let \(P\) be a partition of \(I\) of cardinality \(n\). Then
	\begin{equation*}
		\abs{I}=\sum_{J\in P}\abs{J}.
	\end{equation*}
\end{theorem}
\begin{definition}[Finer and coarser partitions]
	Let \(I\) be a bounded interval, and let \(P\) and \(P'\) be two partitions of \(I\). We say that \(P'\) is finer than \(P\), or \(P\) is coarser than \(P'\), if for every \(J\) in \(P'\), there exists \(K\) in \(P\) such that \(J\subseteq K\).
\end{definition}
\begin{definition}[Common refinement]
	Let \(I\) be a bounded interval, and let \(P\) and \(P'\) be two partitions of \(I\). We define the common refinement \(P\#P'\) of \(P\) and \(P'\) to be the set
	\begin{equation*}
		P\#P'=\{K\cap J\,|\, K\in P\wedge J\in P'\}.
	\end{equation*}
\end{definition}
\begin{definition}[Constant function]
	Let \(X\) be a subset of \(\mathbb{R}\), and let \(f:X\rightarrow R\) be a function. We say that \(f\) is constant iff there exists a real number \(c\) such that \(f(x)=c\) for all \(x\in X\). If \(E\subseteq X\), we say that \(f\) is constant on \(E\) if the restriction \(f|_E\) of \(f\) to \(E\) is constant.
\end{definition}
\begin{definition}[Piecewise constant]
	Let \(I\) be a bounded interval, let \(f:I\rightarrow\mathbb{R}\) be a function, and let \(P\) be a partition of \(I\). We say that \(f\) is piecewise constant with respect to \(P\) iff for every \(j\in P\), f is constant on \(J\). We say that \(f\) is piecewise constant if there exists a partition of its domain with which it is constant relative to.
\end{definition}
\begin{definition}[Piecewise constant integral]
	Let \(I\) be a bounded interval, and let \(P\) be a partition of \(I\). Let \(f:I\rightarrow\mathbb{R}\) be a function which is piecewise constant with respect to \(P\). Then we define the piecewise constant integral \(\int_{[P]}f\) of \(f\) with respect to the partition \(P\) by the formula
	\begin{equation*}
		\int_{[P]}f=\sum_{J\in P}c_J\abs{J},
	\end{equation*}
	where for each \(J\in P\) we let \(c_J\) be the constant value of \(f\) on \(J\).
\end{definition}
\begin{definition}[Piecewise constant integral]
	Let \(I\) be a bounded interval, and let \(f:I\rightarrow\mathbb{R}\) be a function which is piecewise constant function on \(I\). Then we define the piecewise constant integral \(\int_{I}f\) by the formula
	\begin{equation*}
		\int_{I}f=\int_{[P]}f,
	\end{equation*}
	where \(P\) is any partition of \(I\) with respect to which \(f\) is piecewise constant. To explicitly denote we are taking the piecewise constant integral of a piecewise constant function, append p.c. to the integral. However this is usually clear through context.
\end{definition}
\subsection{Upper and Lower Riemann Integrals}
\begin{definition}[Majorization of functions]
	Let \(f:I\rightarrow\mathbb{R}\) and \(g:I\rightarrow\mathbb{R}\). We say that \(g\) majorizes \(f\) on \(I\) if we have \(g(x)\geq f(x)\) for all \(x\in I\), and that \(g\) minorizes \(f\) on \(I\) if \(g(x)\leq f(x)\) for all \(x\in I\).
\end{definition}
\begin{definition}[Upper and lower Riemann integrals]
	Let \(f:I\rightarrow\mathbb{R}\) be a bounded function defined on a bounded interval \(I\). We define the upper Riemann integral \(\overline{\int_{I}}f\) by the formula
	\begin{equation*}
		\overline{\int_I}f=\inf\bigg\{\int_I g\;|\;g\text{ majorizes }f\text{ and is piecewise constant}\bigg\},
	\end{equation*}
	and the lower Riemann integral \(\underline{\int_I}f\) by the formula
	\begin{equation*}
		\underline{\int_I}f=\sup\bigg\{\int_I g\;|\;g\text{ minorizes }f\text{ and is piecewise constant}\bigg\}.
	\end{equation*}
\end{definition}
\begin{lemma}
	Let \(f:I\rightarrow\mathbb{R}\) be a function on a bounded interval \(I\) which is bounded by some real number \(M\). Then we have
	\begin{equation*}
		-M\abs{I}\leq\underline{\int_I}f\leq\overline{\int_I}f\leq M\abs{I}.
	\end{equation*}
	In particular, both the lower and upper Riemann integrals are real numbers.
\end{lemma}
\begin{definition}[Riemann integral]
	Let \(f:I\rightarrow\mathbb{R}\) be a bounded function on a bounded interval \(I\). If \(\underline{\int_I}f=\overline{\int_I}f\), then we say that \(f\) is Riemann integrable on \(I\) and define
	\begin{equation*}
		\int_If=\underline{\int_I}f=\overline{\int_I}f.
	\end{equation*}
\end{definition}
\begin{lemma}
	Let \(f:I\rightarrow\mathbb{R}\) be a piecewise constant function of a bounded interval \(I\). Then \(f\) is Riemann integrable, and \(\int_I f=\text{p.c.}\int_I f\).
\end{lemma}
\begin{definition}[Riemann sums]
	Let \(f:I\rightarrow\mathbb{R}\) be a bounded function on a bounded interval \(I\), and let \(P\) be a partition of \(I\). We define the upper Riemann sum \(U(f,P)\) and lower Riemann sum \(L(f,P)\) by
	\begin{IEEEeqnarray*}{rCl}
		U(f,P)&=&\sum_{J\in P|j\neq\emptyset}\big(\sup_{x\in J}f(x)\big)\abs{J},\\
		L(f,P)&=&\sum_{J\in P|j\neq\emptyset}\big(\inf_{x\in J}f(x)\big)\abs{J}.
	\end{IEEEeqnarray*}
\end{definition}
\begin{proposition}
	Let \(f:I\rightarrow\mathbb{R}\) be a bounded function on a bounded interval \(I\). Then
	\begin{equation*}
		\overline{\int_I}f=\int\{U(f,P)\,|\,P\text{ is a partition of I}\},
	\end{equation*}
	and
	\begin{equation*}
		\underline{\int_I}f=\sup\{L(f,P)\,|\,P\text{ is a partition of I}\}.
	\end{equation*}
\end{proposition}
\begin{theorem}
	Let \(I\) be a bounded interval, and let \(f\) be a function which is uniformly continuous on \(I\). Then \(f\) is Riemann integrable.
\end{theorem}
\begin{corollary}
	Let \(f:[a,b]\rightarrow\mathbb{R}\) be continuous. Then \(f\) is Riemann integrable.
\end{corollary}
\begin{definition}[Piecewise continuous]
	Let \(I\) be a bounded interval, and let \(f:I\rightarrow\mathbb{R}\). We say that \(f\) is piecewise continuous on \(I\) iff there exists a partition \(P\) of \(I\) such that \(f|_J\) is continuous on \(J\) for all \(J\in P\).
\end{definition}
\begin{proposition}
	Let \([a,b]\) be a closed and bounded interval and let \(f:[a,b]\rightarrow\mathbb{R}\) be a monotone function. Then \(f\) is Riemann integrable on \([a,b]\).
\end{proposition}
\begin{definition}[\(\alpha\)-length]
	Let \(I\) be a bounded interval, let \(X\) be an interval that is closed containing \(I\), and let \(\alpha:X\rightarrow\mathbb{R}\) be a monotone increasing function whenever \(x,y\in X\) are such that \(y\geq x\). Then we define the \(\alpha\)-length \(\alpha[I]\) of \(I\) be the following rules:
	\begin{enumerate}
		\item If \(I\) is empty, then \(\alpha[I]=0\).
		\item If \(I=\{a\}\) is a point, then \(\alpha[I]=\lim_{x\rightarrow a^+|x\in X}\alpha(x)-\lim_{x\rightarrow a^-|x\in X}\alpha(x)\), with the 
	\end{enumerate}
\end{definition}
\end{document}

