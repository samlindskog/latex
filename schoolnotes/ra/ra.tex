\documentclass{article}
\usepackage{settings}

\geometry{
a4paper,
total={140mm,257mm},
left=35mm,
top=20mm,
}

\title{Real Analysis}
\author{Samuel Lindskog}

\begin{document}
\maketitle
\addtocontents{toc}{\protect\hypertarget{toc}{}}
\tableofcontents
\pagenumbering{gobble}
\clearpage
\pagenumbering{arabic}
\setcounter{page}{1}

\section{The natural numbers}
\subsection{Peano axioms}
\begin{definition}[Peano axioms]
	Using \(++\) as the successor operation, the natural numbers are defined as follows:
	\begin{enumerate}
		\item \(0\) is a natural number.
		\item If \(n\) is a natural number, then \(n++\) is also a natural number.
		\item For all natural numbers \(n\), \(n++\neq 0\).
	\end{enumerate}
\end{definition}
\begin{definition}[Addition of natural numbers]
	Let \(m\) be a natural number. \(0+m\coloneq m\) and \((n++)+m\coloneq(n+m)++\).
\end{definition}
\begin{proposition}
	\label{uniquezero}
	There is only one zero, i.e. for \(a\in\mathbb{N}\) if \(0+a=0'+a=a\), then \(0=0'\).
\end{proposition}
	\begin{IEEEproof}
		Suppose \(0\neq 0'\). Then \(0\) is a successor of \(0'\) or \(0'\) is a successor of \(0\). Because no successor of a natural number is \(0\), this is impossible.
	\end{IEEEproof}
\begin{proposition}
	\label{zerocommute}
	\(m+0=m\).
\end{proposition}
	\begin{IEEEproof}
		Let \(n\in\mathbb{N}\). \(0+0\coloneq 0\), so by inductive hypothesis \(n+0=n\). \((n++)+0\coloneq(n+0)++\), and from the inductive hypothesis equals \(n++\).
	\end{IEEEproof}
\begin{lemma}
	\label{rightaddition}
	For any natural numbers \(n\) and \(m\), \(n+(m++)=(n+m)++\).
\end{lemma}
	\begin{IEEEproof}
		Suppose \(n,m\in\mathbb{N}\). \(0+(m++)\coloneq m++=(0+m)++\). By inductive hypothesis \(n+(m++)=(n+m)++\). From the definition of addition \((n++)+(m++)=(n+(m++))++\) and from the inductive hypothesis \(n+(m++)=(n+m)++\) so we have
		\begin{IEEEeqnarray*}{rCl}
			(n++)+(m++)&=&(n+(m++))++\\
			&=&((n+m)++)++\\
			&=&((n++)+m)++
		\end{IEEEeqnarray*}
	\end{IEEEproof}
\begin{proposition}[Commutivity of addition]
	For \(n,m\in\mathbb{N}\), \(n+m=m+n\).
\end{proposition}
	\begin{IEEEproof}
		Let \(n,m\in\mathbb{N}\). From proposition \ref{zerocommute}, \(0+m=m+0\), so by inductive hypothesis \(n+m=m+n\). \((n++)+m=(n+m)++\) and from inductive hypothesis this equals \((m+n)++\). From lemma \ref{rightaddition}, this equals \(m+(n++)\).
	\end{IEEEproof}
\begin{proposition}
	If \(a,b\in\mathbb{N}\) and \(a+b=a\), then \(b=0\).
\end{proposition}
	\begin{IEEEproof}
		Suppose \(a,b\in\mathbb{N}\) with \(a+b=a\). \(\)
	\end{IEEEproof}
\begin{proposition}[Associativity of addition]
	Let \(a,b,c\in\mathbb{N}\). Then \((a+b)+c=a+(b+c)\).
\end{proposition}
	\begin{IEEEproof}
		Suppose \(a,b\in\mathbb{N}\). From here we utilize the definition of addition, and commutivity of addition for the rest of the proof. It follows that \((a+b)+0=a+b=a+(b+0)\). By inductive hypothesis suppose \((a+b)+c=a+(b+c)\) for \(c\in\mathbb{N}\). Then
		\begin{IEEEeqnarray*}{rCl}
			(a+b)+c++&=&[(a+b)+c]++\\
			&=&[a+(b+c)]++\\
			&=&a+(c+b)++\\
			&=&a+[(c++)+b]\\
			&=&a+(b+c++)
		\end{IEEEeqnarray*}
	\end{IEEEproof}
\begin{proposition}[Cancellation law]
	Let \(a,b,c\in\mathbb{N}\). Iff \(a+b=a+c\), then \(b=c\).
\end{proposition}
	\begin{IEEEproof}
		If \(0+b=0+c\) then from the definition of addition \(b=c\). By inductive hypothesis for any \(n\in\mathbb{N}\), \(n+b=n+c\). \((n++)+b=(n+b)++\) and \((n++)+c=(n+c)++\), so from the inductive hypothesis and the axioms of natural numbers, \((n++)+b=(n++)+c\).
	\end{IEEEproof}
\begin{definition}[Positive natural number]
	A natural number \(n\) is said to be positive iff it is not \(0\).
\end{definition}
\begin{definition}[Ordering of natural numbers]
	Let \(n,m\in\mathbb{N}\). We write \(n\geq m\) or \(m\geq n\) iff \(n=m+a\) for some \(a\in\mathbb{N}\).
\end{definition}
\begin{proposition}
	\label{nonzerosum}
	If \(a\) or \(b\) are not zero, then \(a+b\neq 0\).
\end{proposition}
	\begin{IEEEproof}
		Suppose \(a,b\in\mathbb{N}\) with \(b\neq 0\). If \(a=0\) then \(a+b=0+b=b\neq 0\). If \(a\neq 0\), because no natural number has zero as a successor it follows from the definition of addition that \(a+b\neq 0\).
	\end{IEEEproof}
\begin{proposition}[Trichotomy of order for natural numbers]
	Let \(a,b\in\mathbb{N}\). Then exactly one of the following statements is true: \(a<b,\,a=b,\,a>b\).
\end{proposition}
	\begin{IEEEproof}
		Suppose \(a,b\in\mathbb{N}\) and \(a<b\). Then for some \(c\in\mathbb{N}\), \(a=b+c\) with \(b\neq a\). If \(c=0\) then \(a=b\), a contradiction. If \(b<a\), then for some \(d\in\mathbb{N}\), \(b=a+d\) with \(a\neq b\). If \(d=0\) then \(a=b\), a contradiction. Because \(b=b+d+c\) and \(c,d\neq 0\), it follows from commutivity and propositions \ref{nonzerosum} and \ref{uniquezero} that this is impossible. Therefore wlog if \(a<b\) then \(a\) is not greater than or equal to \(b\).
	Suppose \(a=b\). If \(a<b\) then \(a=b+c\) for some \(c\in\mathbb{N}\) with \(b\neq c\), a contradiction. Therefore wlog if \(a=b\) then \(a\) is not less than or greater than \(b\).
	\end{IEEEproof}
\begin{proposition}[Strong principle of induction]
	Let \(m_0,m,m'\in\mathbb{N}\), and let \(P(x)\) be a property of arbitrary \(x\in\mathbb{N}\). Suppose that for each \(m\geq m_0\) the following implication holds:
	\begin{equation*}
		\bigg(\forall m'\in[m_0,m),\,P(m')\bigg)\Rightarrow P(m).
	\end{equation*}
	Then we can conclude \(P(m)\) is true for all natural numbers \(m\geq m_0\).
\end{proposition}
\subsection{Multiplication}
\begin{definition}[Multiplication of natural numbers]
	Let \(m\) be a natural number. \(0\times m\coloneq 0\) and \((n++)\times m\coloneq (n\times m)+m\).
\end{definition}
\begin{proposition}
	\label{0m}
	\(m\times 0=0\).
\end{proposition}
	\begin{IEEEproof}
		From the definition of multiplication, \(0\times 0=0\). By inductive hypothesis suppose \(m\times 0=0\). Then \((m++)\times 0=(m\times 0)+0=0\).
	\end{IEEEproof}
\begin{proposition}
	\label{rightmult}
	For \(n,m\in\mathbb{N}\), \(n\times (m++)=(n\times m)+n\).
\end{proposition}
\begin{IEEEproof}
	Let \(n,m\in\mathbb{N}\). \(0\times (m++)=0=(0\times m)+0\). By inductive hypothesis, \((n\times (m++))=(n\times m)+n\). It follows that
	\begin{IEEEeqnarray*}{rCl}
		(n++)\times(m++)&=&(n\times(m++))+(m++)\\
		&=&(n\times m)+n+(m++)\\
		&=&(n\times m)+m+(n++)\\
		&=&((n++)\times m)+(n++)
	\end{IEEEeqnarray*}
\end{IEEEproof}
\begin{proposition}
	\label{1m}
	For \(m\in\mathbb{N}\), \(1m=m\).
\end{proposition}
	\begin{IEEEproof}
		If \(m\in\mathbb{N}\) \(0\times m=0\). Then \((0++)\times m=1\times m=0+m=m\).
	\end{IEEEproof}
\begin{lemma}[Commutivity of multiplication]
	Let \(n,m\in\mathbb{N}\). Then \(n\times m=m\times n\).
\end{lemma}
	\begin{IEEEproof}
		Let \(n,m\in\mathbb{N}\). \(0\times m=m\times 0=0\). By inductive hypothesis, \(n\times m=m\times n\). It follows from proposition \ref{rightmult} that
		\begin{IEEEeqnarray*}{rCl}
			(n++)\times m&=&(n\times m)+m\\
			&=&(m\times n)+m\\
			&=&m\times(n++)
		\end{IEEEeqnarray*}
	\end{IEEEproof}
\begin{proposition}[Distributive law]
	For any natural numbers \(a,b,c,\) we have \(a(b+c)=ab+ac\).
\end{proposition}
\begin{IEEEproof}
	TODO
\end{IEEEproof}
\begin{proposition}[Associativity of multiplication]
	If \(a,b,c\in\mathbb{N}\) then \((a\times b)\times c=a\times(b\times c)\).
\end{proposition}
\begin{IEEEproof}
	TODO
\end{IEEEproof}
\begin{proposition}
	\label{posinatmult}
	If \(a,b\in\mathbb{N}^+\), then \(ab\neq 0\).
\end{proposition}
	\begin{IEEEproof}
		Let \(a\in\mathbb{N}^+\). By proposition \ref{1m} \(1a=a\) and \(a\) is positive. By inductive hypothesis if \(n\in\mathbb{N}^+\) then \(na\) is positive. \(n++\) is a successor to \(n\), and no successor of a natural number is zero, so \(n++\) is positive. \((n++)a=na+a\). Both \(na\) and \(a\) are positive and by proposition \ref{nonzerosum}, \(na+a\) is positive and thus not zero.
	\end{IEEEproof}
\begin{proposition}
	If \(a,b\) are natural numbers such that \(a<b\), and \(c\) is positive, then \(ac<bc\).
\end{proposition}
\begin{corollary}
	Let \(a,b,c\in\mathbb{N}\) such that \(ac=bc\) and \(c\) is non-zero. Then \(a=b\).
\end{corollary}
\begin{proposition}[Euclid's division lemma]
	Let \(n\) be a natural number, and let \(q\) be a positive number. Then there exist natural numbers \(m,r\) such that \(0\leq r<q\) and \(n=mq+r\).
\end{proposition}
\begin{definition}[Exponentiation for natural numbers]
	Let \(m\in\mathbb{N}\). \(m^0\coloneq 1\), and \(m^{n++}=m^n\times m\).
\end{definition}
\section{Set theory}
\subsection{Fundamentals}
\begin{definition}[Axioms of sets]
	\label{setaxioms}
	\,
	\begin{enumerate}
		\item (Sets are objects) If \(A\) is a set, then \(A\) is also an object. In particular, given two sets \(A\) and \(B\), it is meaningful to ask whether \(A\) is also an element of \(B\).
		\item (Equality of sets) Two sets \(A\) and \(B\) are equal iff every element of \(A\) is an element of \(B\) and vice versa.
		\item (Empty set) There exists a set known as the empty set, denoted \(\emptyset\), which contains no elements. In other words, for all objects \(x\) we have \(x\notin\emptyset\).
		\item (Singleton sets) If \(a\) is an object, then there exists a set \(\{a\}\) whose only element is \(a\), i.e. for every object \(y\) we have \(y\in\{a\}\) iff \(y=a\). \(\{a\}\) is referred to as a singleton set.
		\item (Pairwise union) Given any two sets \(A\) and \(B\), there exists a set \(A\cup B\), called the union of \(A\) and \(B\), which consists of all the elements which belong to \(A\) or \(B\). In other words,
			\begin{equation*}
				x\in A\cup B\Leftrightarrow(x\in A\vee x\in B).
			\end{equation*}
		\item (Axiom of specification) Let \(A\) be a set, and for each \(x\in A\) let \(P(x)\) be a property pertaining to \(x\). Then there exists a set \(\{x\in A\,|\,P(x)\}\) whose elements are precisely the elements \(x\) in \(A\) for which \(P(x)\) is true.
		\item (Replacement) Let \(A\) be a set. For any object \(x\in A\) and any object \(y\), suppose we have a property \(P(x,y)\) that is true for at most one \(y\) for each \(x\in A\). Then
			\begin{equation*}
				z\in\{y\,|\,P(x,y),\, x\in A\}\Leftrightarrow P(x,z).
			\end{equation*}
		\item (Infinity) There exists a set \(\mathbb{N}\), whose elements are called natural numbers, as well as an object \(0\in\mathbb{N}\), and an object \(N++\) assigned to every natural number \(n\in\mathbb{N}\), such that the Peano axioms hold.
		\item (Universal specification) DANGER - Suppose for every object \(x\) we have a property \(P(x)\). Then there exists a set \(\{x\,|\,P(x)\}\).
		\item (Regularity). If \(A\) is a non-empty set, then there is at least one element \(x\) of \(A\) which is either not a set, or is disjoint from \(A\).
		\item (Power set) Let \(X\) and \(Y\) be sets. Then there exists a set, denoted \(Y^X\), which consists of all the functions from \(X\) to \(Y\), thus
			\begin{equation*}
				f\in Y^X\Leftrightarrow f\text{ is a function from } X\text{ to }Y.
			\end{equation*}
		\item (Union) Let \(A\) be a set whose elements are all sets. Then there exists a set \(\bigcup A\) defined
			\begin{equation*}
				x\in\bigcup A=\{x\,|\,\exists S\in A,\,x\in S\}.
			\end{equation*}
	\end{enumerate}
\end{definition}
\begin{remark}
	The axioms of set theory introduced, excluding universal specification, are known as the Zermelo-Fraenkel axioms of set theory.
\end{remark}
\begin{lemma}[Single choice]
\label{singlechoice}
	Let \(A\) be a non-empty set. Then there exists an object \(x\) such that \(x\in A\).
	\begin{IEEEproof}
		Suppose there does not exist any object \(x\) such that \(x\in A\). Simultaneously \(x\notin\emptyset\), so \(x\in A\Leftrightarrow x\in\emptyset\) and \(A=\emptyset\), a contradiction.
	\end{IEEEproof}
\end{lemma}
\begin{definition}[Subset]
	Let \(A,B\) be sets. We say that \(A\) is a subset of \(B\), denoted \(A\subseteq B\), iff every element of \(A\) is also an element of \(B\). We say that \(A\) is a proper subset of \(B\), denoted \(A\subsetneq B\), if \(A\subseteq B\) and \(A\neq B\).
\end{definition}
\begin{theorem}
	Let \(A\) be a set. Then \(\emptyset\subseteq A\).
	\begin{IEEEproof}
		If \(\emptyset\subseteq A\) then for all objects \(x\),
		\begin{equation*}
			x\in\emptyset\Rightarrow x\in A.
		\end{equation*}
		This is vacuously true because there does not exist \(x\) such that \(x\in\emptyset\).
	\end{IEEEproof}
\end{theorem}
\begin{definition}[Intersection]
	The intersection \(S_1\cap S_2\) of two sets is the set
	\begin{equation*}
		S_1\cap S_2=\{x\,|\,x\in S_1\wedge x\in S_2\}.
	\end{equation*}
\end{definition}
\begin{definition}[Union]
	The union \(S_1\cup S_2\) of two sets is the set
	\begin{equation*}
		S_1\cup S_2=\{x\,|\,x\in S_1\vee x\in S_2\}.
	\end{equation*}
\end{definition}
\begin{definition}[Disjoint]
	Two sets are disjoint if \(A\cap B=\emptyset\).
\end{definition}
\begin{definition}[Difference set]
	If \(A\) and \(B\) are sets, the set \(A\setminus B\) is the set \(A\) with any elements of \(B\) removed, i.e.
	\begin{equation*}
		A\setminus B\coloneq\{x\,|\,x\in A\wedge x\notin B\}.
	\end{equation*}
\end{definition}
\begin{proposition}
	Let \(A,B,C\) be subsets of set \(X\).
	\begin{enumerate}
		\item (Minimal element) \(A\cup\emptyset=A\) and \(A\cap\emptyset=\emptyset\).
		\item (Maximal element) \(A\cup X=X\) and \(A\cap X=A\).
		\item (Identity) \(A\cap A=A\) and \(A\cup A=A\).
		\item (Commutativity) \(A\cup B=B\cup A\) and \(A\cap B=B\cap A\).
		\item (Associativity) \((A\cup B)\cup C=A\cup(B\cup C)\) and \((A\cap B)\cap C=A\cap(B\cap C)\).
		\item (Distributivity) \(A\cap (B\cup C)=(A\cap B)\cup(A\cap C)\) and \(A\cup(B\cap C)=(A\cup B)\cap(A\cup C)\).
		\item (Partition) \(A\cup(X\setminus A)=X\) and \(A\cap (X\setminus A)=\emptyset\).
		\item (De Morgan Laws) \(X\setminus(A\cup B)=(X\setminus A)\cap(X\setminus B)\) and \(X\setminus (A\cap B)=(X\setminus A)\cup(X\setminus B)\).
	\end{enumerate}
\end{proposition}
\begin{definition}[Ordered pair]
	If \(x\) and \(y\) are any objects, we define the ordered pair \((x,y)\) to be a new object which consists of \(x\) as its "first component" and \(y\) as its "second component". Two ordered pairs \(x,y\) and \(x',y'\) are equal if
	\begin{equation*}
		x=x',\quad y=y'.
	\end{equation*}
\end{definition}
\begin{definition}[Cartesian product]
	Let \(A,B\) be sets. Then the cartesian product of \(A\) and \(B\), written \(A\times B\), is
	\begin{equation*}
		A\times B=\{(a,b)\,|,a\in A,\,b\in B\}.
	\end{equation*}
\end{definition}
\begin{definition}[Ordered \(n\)-tuple]
	Let \(n\) be a natural number. An ordered \(n\)-tuple \((x_i)_{1\leq i\leq n}\) is a collection of objects \(x_i\), one for every natural number \(i\) between \(1\) and \(n\). Two ordered \(n\)-tuples \((x_i)_{1\leq i\leq n}\) and \((y_i)_{1\leq i\leq n}\) are said to be equal iff \(x_i=y_i\) for all \(1\leq i\leq n\).
\end{definition}
\begin{definition}[\(n\)-fold Cartesian product]
	If \((X_i)_{1\leq i\leq n}\) is an ordered \(n\)-tuple of sets, their Cartisian product \(\prod_{i=1}^{n}X_i\) is defined
	\begin{equation*}
		\prod_{i=1}^n X_i=\{(x_i)_{1\leq i\leq n}\,|\, x_i\in X_i\}.
	\end{equation*}
\end{definition}
\begin{definition}[Indexed family]
	If for each element \(j\in J\) with \(J\neq\emptyset\), there corresponds a set \(A_j\), then
	\begin{equation*}
		\mathscr{A}=\{A_j\,|\,j\in J\}.
	\end{equation*}
	Is called an indexed family of sets with \(J\) as the index set. If \(J=\{1,2,\ldots,n\}\) we may index the set similarly to sum notation.
\end{definition}
\begin{definition}[Union and intersection of indexed family]
	The union of all sets in an indexed family \(\mathscr{A}\) with index set \(J\) is
	\begin{equation*}
		\bigcup_{j\in J}A_j=\{x\,|\,\exists A_j\in\mathscr{A},\,x\in A_j\}.
	\end{equation*}
	The intersection of all sets in \(\mathscr{A}\) is
	\begin{equation*}
		\bigcap_{j\in J}A_j=\{x\,|\,\forall A_j\in\mathscr{A},\,x\in A_j\}.
	\end{equation*}
\end{definition}
\begin{lemma}[Finite choice]
	Let \(n\geq 1\) be a natural number, and for each natural number \(1\leq i\leq n\), let \(X_i\) be a non-empty set. Then there exists an \(n\)-tuple \((x_i)_{1\leq i\leq n}\) such that \(x_i\in X_i\) for all \(1\leq i\leq n\). In other words if each \(X_i\) is non-empty, then its \(n\)-fold cartesian product is nonempty.
	\begin{IEEEproof}
		Let \(\mathscr{A}=\{A_i\,|\,1\leq i\leq n\}\) with \(n\in\mathbb{N}\) be an indexed family of nonempty sets. It follows from lemma \ref{singlechoice} that for each \(A_i\), \(1\leq i\leq n\), there exists \(a_i\in A_i\). Using this fact, define an ordered \(n\)-tuple \((a_i)_{1\leq i\leq n}\).
	\end{IEEEproof}
\end{lemma}
\subsection{Functions}
\begin{definition}[Relation]
	Let \(A,B\)	be sets. A relation between \(A\) and \(B\) is an subset of \(A\times B\).
\end{definition}
\begin{definition}[Equivalence relation]
	An equivalence relation on a set \(S\) is a relation such that for all \(x,y,z\in S\), the relation satisfies the following properties:
	\begin{enumerate}
		\item (Reflexive property) \(xRx\).
		\item (Symmetric property) \(xRy\Rightarrow yRx\).
		\item (Transitive property) \(xRy\wedge yRx\Rightarrow xRz\).
	\end{enumerate}
\end{definition}
\begin{definition}[Partition]
	A partition of a set \(S\) is a collection \(\mathscr{P}\) of nonempty subsets of \(S\) that are pairwise disjoint, and whose union is \(S\), i.e.
	\begin{enumerate}
		\item \(A=\bigcup\mathscr{P}\).
		\item \(\forall A,B\in\mathscr{P},\,A\neq B\Rightarrow A\cap B=\emptyset\).
	\end{enumerate}
\end{definition}
\begin{definition}[Function]
	A function from \(A\) to \(B\), denoted \(f:A\rightarrow B\) is a nonempty relation \(f\subseteq A\times B\) that satisfies the following properties:
	\begin{enumerate}
		\item (Existence) \(\forall a\in A,\,\exists b\in B,\,(a,b)\in f\).
		\item (Uniqueness) \((a,b)\in f\wedge(a,c)\in f\Rightarrow b=c\).
	\end{enumerate}
	Set \(A\) is called the domain of \(f\), and set \(B\) is called the codomain. The range of \(f\) is \(f(A)\), i.e. \(\{b\in B\,|\,(a,b)\in f\}\).
\end{definition}
\begin{definition}[Equality of functions]
	Two functions \(f:X\rightarrow Y\) and \(g:X'\rightarrow Y'\) are equal if their domains and codomains are equal, and furthermore that \(f(x)=g(x)\) for all \(x\in X\).
\end{definition}
\begin{definition}[Composition]
	Let \(f:X\rightarrow Y\) and \(g:Y\rightarrow Z\) be two functions such that the codomain of \(f\) is the same set as the domain of \(g\). Then the composition \(g\circ f:X\rightarrow Z\) of the two functions \(g\) and \(f\) is the function defined by the formula
	\begin{equation*}
		(g\circ f)(x)=g(f(x)).
	\end{equation*}
\end{definition}
\begin{lemma}
	Let \(f:Z\rightarrow W\), \(g:Y\rightarrow Z\), and \(h:X\rightarrow Y\) be functions. Then \(f\circ(g\circ h)=(f\circ g)\circ h\).
	\begin{IEEEproof}
		\(g\circ h\) is a function from \(X\) to \(Z\), and \(f\circ g\) is a function from \(Y\rightarrow W\), so \((f\circ g)\circ h\) and \(f\circ(g\circ h)\) are functions from \(X\) to \(W\). It follows from the definition of function composition that
		\begin{IEEEeqnarray*}{rCl}
			(f\circ(g\circ h))(x)&=&f((g\circ h)(x))\\
			&=&f(g(h(x)))\\
			&=&(f\circ g)(h(x))\\
			&=&((f\circ g)\circ h)(x)
		\end{IEEEeqnarray*}
	\end{IEEEproof}
\end{lemma}
\begin{definition}[Injective]
	A function \(f:X\rightarrow Y\) is injective (one-to-one) if for \(x,x'\in X\),
	\begin{equation*}
		x\neq x'\rightarrow f(x)\neq f(x')
	\end{equation*}
\end{definition}
\begin{definition}[Surjective]
	A function \(f:X\rightarrow Y\) is surjective (onto) if
	\begin{equation*}
		\forall y\in Y,\,\exists x\in X,\,f(x)=y.
	\end{equation*}
\end{definition}
\begin{definition}[Bijective]
	A function is bijective (invertible) if it is injective and surjective.
\end{definition}
%
\begin{proposition}
	Let \(f:A\rightarrow B\) and \(g:B\rightarrow C\). Then
	\begin{enumerate}
		\item If \(f\) and \(g\) are surjective, then \(g\circ f\) is surjective.
		\item If \(f\) and \(g\) are injective, then \(g\circ f\) is injective.
		\item If \(f\) and \(g\) are bijective, then \(g\circ f\) is bijective.
	\end{enumerate}
\end{proposition}
\begin{lemma}
	If \(f:X\rightarrow Y\) is bijective then \(f\) is invertible. In other words for all \(y\in Y\) there exists a unique \(x\in X\) denoted \(f^{-1}(y)\) such that \(f(x)=y\). Therefore the inverse of \(f\), \(f^{-1}:Y\rightarrow X\) exists and is defined
	\begin{equation*}
		f^{-1}(y)=x.
	\end{equation*}
\end{lemma}
\begin{definition}[Identity function]
	A function defined on a set \(A\) that maps each element in \(A\) onto itself is called the identity function on \(A\), and is denoted \(i_A\).
\end{definition}
\begin{proposition}
	Let \(f:A\rightarrow B\) be bijective. Then
	\begin{enumerate}
		\item \(f^{-1}:B\rightarrow A\) is bijective.
		\item \(f^{-1}\circ f=i_A\) and \(f\circ f^{-1}=i_B\).
	\end{enumerate}
\end{proposition}
\begin{theorem}
	Let \(f:A\rightarrow B\) and \(g:A\rightarrow B\) be bijective. Then the composition \(g\circ f:A\rightarrow C\) is bijective and \((g\circ f)^{-1}=f^{-1}\circ g^{-1}\).
\end{theorem}
%
\begin{definition}[Image]
	If \(f:X\rightarrow Y\) is a function from \(X\) to \(Y\), and \(S\subseteq X\), we define the image of \(S\) under \(f\), \(f(S)\) to be the set
	\begin{equation*}
		f(S)=\{f(x)\,|\,x\in S\}.
	\end{equation*}
\end{definition}
\begin{definition}[Inverse image]
	If \(U\) is a subset of \(Y\), we define the set \(f^{-1}(U)\) to be the set
	\begin{equation*}
		f^{-1}(U)=\{x\in X\,|\,f(x)\in U\}.
	\end{equation*}
	We call \(f^{-1}(U)\) the inverse image of \(U\).
\end{definition}
\begin{proposition}
	\label{imagessubset}
	If \(X,Y\) are sets and \(f:X\rightarrow Y\) then \(f(X)\subseteq Y\).
	\begin{IEEEproof}
		\(y\in f(X)\) implies \(y\in \{y\,|\,(x,y)\in f\}\) and \(f\) is a subset of \(X\times Y\), so it follows from the definition of the cartesian product that \(y\in Y\).
	\end{IEEEproof}
\end{proposition}
\begin{lemma}
	Let \(X\) be a set. Then the set
	\begin{equation*}
		\{Y\,|\,Y\subseteq X\}
	\end{equation*}
	Is a set.
	\begin{IEEEproof}
		Let \(X\) be a set and \(A\subseteq X\) with \(A\neq\emptyset\). Then there exists \(p\in A\), and we can define a function \(f:X\rightarrow A\) with \(x\in X\) by
		\begin{equation*}
			f(x)=\begin{cases}
				x\in A&f(x)=x\\
				x\notin A&f(x)=p
			\end{cases}
		\end{equation*}
		Thus for all \(a\in f(X)\), \(a\in A\) or \(a=p\in A\), so \(f(X)\subseteq A\). Next, for all \(x\in X\), \((x,f(x))\in f(X)\). Because for all \(a\in A\) we have \(a\in X\) then for all \(a\in A\), \((a,f(a))=(a,a)\in f(X)\) so from the definition of an image \(A\subseteq f(X)\). Thus \(A=F(X)\). From the power set axiom in definition \ref{setaxioms}, 
		\begin{equation*}
			\{f:X\rightarrow A\,|\,A\subseteq X\wedge A\neq\emptyset\}\subseteq X^X
		\end{equation*}
		From replacement, pairwise union, and singleton set axioms in definition \ref{setaxioms}, we can define a set \(\text{P}(X)\) that is the union of all images of functions in \(X^X\), and \(\{\emptyset\}\). As established above, all nonempty subsets of \(X\) are included in this set, and from proposition \ref{imagessubset} all images of functions in \(X^X\) are subsets of \(X\).
	\end{IEEEproof}
\end{lemma}
\begin{definition}[Power set]
	For a set \(X\), the set \(\{Y\,|\,Y\subseteq X\}\) is called the power set of \(X\), and is denoted \(\text{P}(X)\) or \(2^X\).
\end{definition}
\begin{definition}[Cardinality]
	We say that two sets \(X\) and \(Y\) have equal cardinality iff there exists a bijection \(f:X\rightarrow Y\) from \(X\) to \(Y\).
\end{definition}
\begin{proposition}
	Let \(X,Y,Z\) be sets.
	\begin{enumerate}
		\item \(X\) has equal cardinality with \(X\).
		\item If \(X\) has equal cardinality with \(Y\), then \(Y\) has equal cardinality with \(X\).
		\item If \(X\) has equal cardinality \(Y\) and \(Y\) has equal cardinality with \(Z\), then \(X\) has equal cardinality with \(Z\)
			\begin{IEEEproof}
			\end{IEEEproof}
	\end{enumerate}
\end{proposition}
\begin{definition}[Cardinality \(n\)]
	Let \(n\) be a natural number. A set \(X\) is said to have cardinality \(n\), if it has equal cardinality with \(\{\in\mathbb{N}\,|\,1\leq i\leq n\}\). In this case we say that \(X\) has \(n\) elements.
\end{definition}
\begin{lemma}
	Suppose that \(n\geq 1\), and set \(X\) has cardinality \(n\). Then \(X\) is non-empty, and if \(x\) is any element of \(X\), then the set \(X-\{x\}\) has cardinality \(n-1\).
\end{lemma}
\begin{proposition}
	Let \(X\) be a set with some cardinality \(n\). Then \(X\) cannot have any other cardinality, i.e. \(X\) cannot have cardinality \(m\) for any \(m\neq n\).
\end{proposition}
\begin{definition}[Finite set]
	A set is finite iff it has cardinality \(n\) for some natural number \(n\); otherwise, the set is called infinte.
\end{definition}
\begin{theorem}
	The set of natural numbers is infinite.
\end{theorem}
\section{Integers and rationals}
\subsection{The integers}
\begin{definition}[Integers]
	An integer is an expression of the form \(a-b\), where \(a\) and \(b\) are natural numbers. Two integers are considered to be equal, \(a-b=c-d\), iff \(a+d=c+b\). The set of all integers is denoted \(\mathbb{Z}\).
\end{definition}
\begin{remark}
	The use of \(-\) is purely notational (until subtraction is defined). \(a-b\) can be interpreted as an ordered pair in \(\mathbb{N}\times\mathbb{N}\).
\end{remark}
\begin{definition}[Integer addition]
	The sum of two integers \((a-b)+(c-d)\) is defined by the formula
	\begin{equation*}
		(a-b)+(c-d)=(a+c)-(c+d)
	\end{equation*}
\end{definition}
\begin{definition}[Integer multiplication]
	The product of two integers \((a-b)\times (c-d)\) is defined by the formula
	\begin{equation*}
		(a-b)\times(c-d)=(ac+bd)-(ad+bc).
	\end{equation*}
\end{definition}
\begin{remark}
	We may identify the integers with natural numbers by setting \(n\equiv n-0\). Definitions of equality and previously defined operations remain consistent with each other.
\end{remark}
\begin{proposition}
	\label{addzero}
	If \(a,b\in\mathbb{Z}\) and \(a+b=b\) then \(a=0\).
	\begin{IEEEproof}
prove
	\end{IEEEproof}
\end{proposition}
\begin{lemma}
	Addition and multiplication are well defined.
\end{lemma}
\begin{definition}[Negation of integers]
	If \((a-b)\) is an integer, we define the negation \(-(a-b)\) to be the integer \(b-a\).
\end{definition}
\begin{lemma}[Trichotomy of integers]
	Let \(x\) be an integer. Then either \(x\) is zero, equal to a positive natural number, or \(x\) negated is a positive natural number.
\end{lemma}
\begin{definition}[Positive integer]
	If \(n\) is a positive natural number, we call \(n\) a positive integer, and \(-n\) a negative integer.
\end{definition}
\begin{proposition}[Integer laws for algebra]
	Let \(x,y,z\) be integers. Then the following identities hold:
	\begin{IEEEeqnarray*}{l}
		x+y=y+x\\
		(x+y)+z=x+(y+z)\\
		x+0=0+x=x\\
		x+(-x)=0\\
		xy=yx\\
		(xy)z=x(yz)\\
		1x=x\\
		x(y+z)=xy+xz\\
	\end{IEEEeqnarray*}
\end{proposition}
\begin{proposition}
	\label{posiintmult}
	If \(a,b\in\mathbb{Z}\) with \(a,b>0\), then \(ab>0\).
	\begin{IEEEproof}
		If \(a,b\in\mathbb{Z}\) with \(a,b>0\), then for some \(x,y\in\mathbb{N}^+\), \(a=x-0\) and \(b=y-0\). Thus \(ab=(xy+0)=(0+0)=xy-0\). Because \(x,y\neq 0\), by proposition \ref{posinatmult} \(xy>0\) so from the definition of a positive integer \(ab>0\).
	\end{IEEEproof}
\end{proposition}
\begin{proposition}
	\label{posiintadd}
	If \(a,b\in\mathbb{Z}\) with \(a,b>0\), then \(a+b>0\).
	\begin{IEEEproof}
		If \(a,b\in\mathbb{Z}^+\), then for some \(x,y\in\mathbb{N}^+\) we have \(a=x-0\) and \(b=y-0\), so \(a+b=((x+y)-0)\). It follows from proposition \ref{nonzerosum} that \(x+y>0\) so from the definition of a positive integer, \(a+b>0\).
	\end{IEEEproof}
\end{proposition}
\begin{proposition}
	\label{monenegation}
	If \(x\in\mathbb{Z}\) with \(x=(a-b)\) then \(-1\cdot(a-b)=-(a-b)\).
	\begin{IEEEproof}
		\(-1\cdot (a-b)=(0-1)\cdot(a-b)=(0a+b)-(a+0b)=-(a-b)\).
	\end{IEEEproof}
\end{proposition}
\begin{proposition}[Integers have no zero divisors]
	\label{nozerodiv}
	If \(a,b\) are integers such that \(ab=0\), then \(a=0\) or \(b=0\).
\end{proposition}
\begin{corollary}[Cancellation law]
	If \(a,b,c\) are integers such that \(ac=bc\) and \(c\) is non-zero, then \(a=b\).
	\begin{IEEEproof}
		Let \(a,b,c\in\mathbb{Z}\) with \(c=\neq 0\). If \(a=0\) it follows from proposition \ref{nozerodiv} that \(ac=0\) so \(bc=0\) and thus \(b=0\), so \(a=b\). If \(a\neq 0\), suppose to the contrary that \(b\neq a\). It follows from proposition \ref{addzero} that there exists \(d\in\mathbb{Z}\) with \(d\neq 0\) such that \(a+d=b\). Using laws for algebra we see that \(ac=ac+dc\). By proposition \ref{nozerodiv} \(dc\neq 0\), a contradiction by proposition \ref{addzero}. Therefore \(a=b\).
	\end{IEEEproof}
\end{corollary}
\begin{definition}[Ordering of integers]
	Let \(n,m\in\mathbb{Z}\). We say that \(n\) is greater than or equal to \(m\) and write \(n\geq m\) or \(m\leq n\) iff we have \(n=m+a\) for some natural number \(a\). We say that \(n\) is strictly greater than \(m\) and write \(n>m\) or \(m<n\) iff \(n\geq m\) and \(n\neq m\).
\end{definition}
\subsection{The rationals}
\begin{definition}[Rational number]
	A rational number is an expression of the form \(a//b\), where \(a\) and \(b\) are integers and \(b\neq 0\). Two rational numbers are equal, \(a//b=c//d\), iff \(ad=bc\). The set of all rational numbers is denoted \(\mathbb{Q}\).
\end{definition}
\begin{remark}
We may indentify the rationals with natural numbers by setting \(n//1\equiv n\).
\end{remark}
\begin{definition}[Addition of rationals]
	If \(a//b\) and \(c//d\) are rationals, their sum is
	\begin{equation*}
		(a//b)+(c//d)=(ad+bc)//(bd).
	\end{equation*}
\end{definition}
\begin{definition}[Product of rationals]
	If \(a//b\) and \(c//d\) are rationals, their product is
	\begin{equation*}
		(a//b)\cdot(c//d)=(ac)//(bd).
	\end{equation*}
\end{definition}
\begin{definition}[Negation of rationals]
	The negation of a rational \((a//b)\), denoted \(=(a//b)\) is
	\begin{equation*}
		-(a//b)=(-a//b).
	\end{equation*}
\end{definition}
\begin{definition}[Reciprocal of rationals]
	If \(x=a//b\) is a non-zero rational number, then the reciprocal of \(x^{-1}\) of \(x\) is defined
	\begin{equation*}
		x^{-1}=b//a.
	\end{equation*}
\end{definition}
\begin{lemma}
	The sum, product, negation, and reciprocal operations on rational numbers are well-defined.
\end{lemma}
\begin{proposition}
	\label{doublenegation}
	The negation of the negation of \(x\in\mathbb{Q}\) is \(x\).
	\begin{IEEEproof}
		The negation of the negation of an integer \(x=(a-b)\) is \(--(a-b)=-(b-a)=(a-b)\) so \(--x=x\). The negation of the negation of a rational number \(y=(c//d)\) is \(--(c//d)=-(-c//d)=(--c//d)=c//d\).
	\end{IEEEproof}
\end{proposition}
\begin{definition}[Quotient]
	The quotient of two rationals \(x\) and \(y\) with \(y\neq 0\), denoted \(x/y\), is
	\begin{equation*}
		x/y=x\times y^{-1}.
	\end{equation*}
\end{definition}
\begin{definition}[Subtraction]
	The difference of two rationals \(x\) and \(y\), denoted \(x-y\), is defined
	\begin{equation*}
		x-y=x+(-y).
	\end{equation*}
\end{definition}
\begin{definition}[Positive rational number]
	A rational number \(x\) is said to be positive iff we have \(x=a/b\) for some positive integers \(a\) and \(b\). It is said to be negative iff \(x=-y\) for some positive rational \(y\).
\end{definition}
\begin{definition}[Ordering of rationals]
	Let \(x,y\in\mathbb{Q}\). We say that \(x>y\) iff \(x-y\) is a positive rational number, and \(x<y\) iff \(x-y\) is a positive negative rational number. We write \(x\geq y\) iff either \(x>y\) or \(x=y\), and \(x\leq y\) iff either \(x<y\) or \(x=y\).
\end{definition}
\begin{proposition}
	\label{xgzerop}
	\(x\in\mathbb{Q}\) is positive iff \(x>0\), and negative iff \(x<0\).
	\begin{IEEEproof}
		If \(x= a//b\) is a positive rational number then \(a,b>0\). Because \(0=0//d\) for some \(d\in\mathbb{N}\setminus\{0\}\), \(x-0=x+0=ad//bd=a//b\) and thus \(x>0\). If \(x>0\) then \(x-0\) is positive. Because \(0=0//d\) for some \(d\in\mathbb{N}\setminus\{0\}\) we have \(x-0=x+0=ad//bd=a//b\), which is positive.
	\end{IEEEproof}
\end{proposition}
\begin{proposition}[Laws of algebra for rationals]
	Let \(x,y,z\) be rationals. Then the following laws of algebra hold:
	\begin{IEEEeqnarray*}{l}
		x+y=y+x\\
		(x+y)+z=x+(y+z)\\
		x+0=x\\
		x+(-x)=0\\
		xy=yx\\
		(xy)z=d(yz)\\
		1x=x\\
		x(y+z)=xy+xz\\
	\end{IEEEeqnarray*}
\end{proposition}
\begin{proposition}
	\label{ronenegation}
	\(-1x=-x\).
	\begin{IEEEproof}
		If \(x=a//b\) then \(-1\cdot x\) is \((-1//1)\cdot(a//b)=-1a//b\). From proposition \ref{monenegation}, \(-1a=-a\) so \(-1a//b=-(a//b)=-x\).
	\end{IEEEproof}
\end{proposition}
\begin{proposition}
	\label{posiratadd}
	If \(a,b\in\mathbb{Q}\) with \(a>0\) and \(b>0\) then \(a+b>0\).
	\begin{IEEEproof}
		Suppose \(a,b\in\mathbb{Q}\) with \(a,b>0\). It follows from the definition of positive rational number that for some positive \(x,y,z,w\in\mathbb{Z}\), \(a=x//y\) and \(b=z//w\), so \(ab=xw+zy/yw\). By proposition \ref{posiintmult} \(xw,zy,yw>0\), so by \ref{posiintadd}, \(a+b>0\).
	\end{IEEEproof}
\end{proposition}
\begin{lemma}[Trichotomy of rationals]
	Let \(x\) be a rational number. Then exactly one of the following three statements is true:
	\begin{enumerate}
		\item \(x=0\).
		\item \(x\) is positive.
		\item \(x\) is negative.
	\end{enumerate}
\end{lemma}
\subsection{Absolute value and exponentiation}
\begin{definition}[Absolute value]
	If \(x\) is a rational number, the absolute value \(\abs{x}\) of \(x\) is defined as follows:
	\begin{equation*}
		\abs{x}=\begin{cases}
			x,&x\geq 0\\
			-x,&x<0
		\end{cases}
	\end{equation*}
\end{definition}
\begin{definition}[Distance]
	The distance between \(x,y\in\mathbb{Q}\), sometimes denoted \(d(x,y)\), is
	\begin{equation*}
		d(x,y)=\abs{x-y}.
	\end{equation*}
\end{definition}
\begin{proposition}
	For all \(x\in\mathbb{Q}\), \(\abs{x}\geq 0\).
	\begin{IEEEproof}
		If \(x\geq 0\) then \(\abs{x}=x\) so \(\abs{x}\geq 0\). If \(x<0\) then \(\abs{x}=-x\). By proposition \ref{xgzerop} \(x\) is negative. Therefore there exists \(y\in\mathbb{Q}^+\) such that \(x=-y\), so by proposition \ref{doublenegation} \(-x=--y=y\) and \(-x\) is positive. By proposition \ref{xgzerop}, \(-x>0\).
	\end{IEEEproof}
\end{proposition}
\begin{proposition}[Triangle inequality]
	For \(x,y\in\mathbb{Q}\), \(\abs{x+y}\leq\abs{x}+\abs{y}\).
	\begin{IEEEproof}
	\end{IEEEproof}
\end{proposition}
\begin{definition}[\(\epsilon\)-closeness]
	Let \(\epsilon>0\) be a rational number, and let \(x,y\) be rational numbers. We say that \(y\) is \(\epsilon\)-close to \(x\) iff \(d(y,x)<\epsilon\).
\end{definition}
\begin{definition}[Exponentiation to a natural number]
	Let \(x\) be a rational number. To raise \(x\) to the power \(0\), we define \(x^0=1\) and for all \(n\in\mathbb{N}\), \(x^{n+1}=x^n\times x\).
\end{definition}
\begin{definition}[Exponentiation to a negative number]
	Let \(x\) be a non-zero rational number. Then for any negative integer \(-n\),
	\begin{equation*}
		x^{-n}=1/x^n.
	\end{equation*}
\end{definition}
\section{The real numbers}
\subsection{Sequences}
\end{document}
