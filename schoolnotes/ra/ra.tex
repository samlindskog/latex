\documentclass{article}
\usepackage{settings}

\geometry{
a4paper,
total={140mm,257mm},
left=35mm,
top=20mm,
}

\title{Real Analysis}
\author{Samuel Lindskog}

\begin{document}
\maketitle
\addtocontents{toc}{\protect\hypertarget{toc}{}}
\tableofcontents
\pagenumbering{gobble}
\clearpage
\pagenumbering{arabic}
\setcounter{page}{1}

\section{The natural numbers}
\subsection{Peano axioms}
\begin{definition}[Peano axioms]
	Using \(++\) as the successor operation, the natural numbers are defined as follows:
	\begin{enumerate}
		\item \(0\) is a natural number.
		\item If \(n\) is a natural number, then \(n++\) is also a natural number.
		\item For all natural numbers \(n\), \(n++\neq 0\).
	\end{enumerate}
\end{definition}
\begin{definition}[Addition of natural numbers]
	Let \(m\) be a natural number. \(0+m\coloneq m\) and \((n++)+m\coloneq(n+m)++\).
\end{definition}
\begin{proposition}
	\label{zerocommute}
	\(m+0=m\).
	\begin{IEEEproof}
		Let \(n\in\mathbb{N}\). \(0+0\coloneq 0\), so by inductive hypothesis \(n+0=n\). \((n++)+0\coloneq(n+0)++\), and from the inductive hypothesis equals \(n++\).
	\end{IEEEproof}
\end{proposition}
\begin{lemma}
	\label{rightaddition}
	For any natural numbers \(n\) and \(m\), \(n+(m++)=(n+m)++\).
	\begin{IEEEproof}
		Suppose \(n,m\in\mathbb{N}\). \(0+(m++)\coloneq m++=(0+m)++\). By inductive hypothesis \(n+(m++)=(n+m)++\). From the definition of addition \((n++)+(m++)=(n+(m++))++\) and from the inductive hypothesis \(n+(m++)=(n+m)++\) so we have
		\begin{IEEEeqnarray*}{rCl}
			(n++)+(m++)&=&(n+(m++))++\\
			&=&((n+m)++)++\\
			&=&((n++)+m)++
		\end{IEEEeqnarray*}
	\end{IEEEproof}
\end{lemma}
\begin{proposition}[Commutivity of addition]
	For \(n,m\in\mathbb{N}\), \(n+m=m+n\).
	\begin{IEEEproof}
		Let \(n,m\in\mathbb{N}\). From proposition \ref{zerocommute}, \(0+m=m+0\), so by inductive hypothesis \(n+m=m+n\). \((n++)+m=(n+m)++\) and from inductive hypothesis this equals \((m+n)++\). From lemma \ref{rightaddition}, this equals \(m+(n++)\).
	\end{IEEEproof}
\end{proposition}
\begin{proposition}[Associativity of addition]
	Let \(a,b,c\in\mathbb{N}\). Then \((a+b)+c=a+(b+c)\).
	\begin{IEEEproof}
		exercise
	\end{IEEEproof}
\end{proposition}
\begin{proposition}[Cancellation law]
	Let \(a,b,c\in\mathbb{N}\). Iff \(a+b=a+c\), then \(b=c\).
	\begin{IEEEproof}
		If \(0+b=0+c\) then from the definition of addition \(b=c\). By inductive hypothesis for any \(n\in\mathbb{N}\), \(n+b=n+c\). \((n++)+b=(n+b)++\) and \((n++)+c=(n+c)++\), so from the inductive hypothesis and the axioms of natural numbers, \((n++)+b=(n++)+c\).
	\end{IEEEproof}
\end{proposition}
\begin{definition}[Positive natural number]
	A natural number \(n\) is said to be positive iff it is not \(0\).
\end{definition}
\begin{definition}[Ordering of natural numbers]
	Let \(n,m\in\mathbb{N}\). We write \(n\geq m\) or \(m\geq n\) iff \(n=m+a\) for some \(a\in\mathbb{N}\).
\end{definition}
\begin{proposition}
	Let \(m_0,m,m'\in\mathbb{N}\), and let \(P(x)\) be a property of arbitrary \(x\in\mathbb{N}\). Suppose that for each \(m\geq m_0\) the following implication holds:
	\begin{equation*}
		\bigg(\forall m'\in[m_0,m),\,P(m')\bigg)\Rightarrow P(m).
	\end{equation*}
	Then we can conclude \(P(m)\) is true for all natural numbers \(m\geq m_0\).
\end{proposition}
\subsection{Multiplication}
\begin{definition}[Multiplication of natural numbers]
	Let \(m\) be a natural number. \(0\times m\coloneq 0\) and \((n++)\times m\coloneq (n\times m)+m\).
\end{definition}
\begin{lemma}[Commutivity of multiplication]
	Let \(n,m\in\mathbb{N}\). Then \(n\times m=m\times n\).
	\begin{IEEEproof}
		exercise
	\end{IEEEproof}
\end{lemma}
\begin{lemma}
	Let \(n,m\in\mathbb{N}\). Then \(n\times m=0\) iff \(n\) or \(m\) is zero.
	\begin{IEEEproof}
		exercise
	\end{IEEEproof}
\end{lemma}
\begin{proposition}[Distributive law]
	For any natural numbers \(a,b,c,\) we have \(a(b+c)=ab+ac\).
\end{proposition}
\begin{proposition}[Associativity of multiplication]
	If \(a,b,c\in\mathbb{N}\) then \((a\times b)\times c=a\times(b\times c)\).
\end{proposition}
\begin{proposition}
	If \(a,b\) are natural numbers such that \(a<b\), and \(c\) is positive, then \(ac<bc\).
\end{proposition}
\begin{corollary}
	Let \(a,b,c\in\mathbb{N}\) such that \(ac=bc\) and \(c\) is non-zero. Then \(a=b\).
\end{corollary}
\begin{proposition}[Euclid's division lemma]
	Let \(n\) be a natural number, and let \(q\) be a positive number. Then there exist natural numbers \(m,r\) such that \(0\leq r<q\) and \(n=mq+r\).
\end{proposition}
\begin{definition}[Exponentiation for natural numbers]
	Let \(m\in\mathbb{N}\). \(m^0\coloneq 1\), and \(m^{n++}=m^n\times m\).
\end{definition}
\section{Set theory}
\subsection{Fundamentals}
\begin{definition}[Axioms of sets]
	\,
	\begin{enumerate}
		\item (Sets are objects) If \(A\) is a set, then \(A\) is also an object. In particular, given two sets \(A\) and \(B\), it is meaningful to ask whether \(A\) is also an element of \(B\).
		\item (Equality of sets) Two sets \(A\) and \(B\) are equal iff every element of \(A\) is an element of \(B\) and vice versa.
		\item (Empty set) There exists a set known as the empty set, denoted \(\emptyset\), which contains no elements. In other words, for all objects \(x\) we have \(x\notin\emptyset\).
		\item (Singleton sets) If \(a\) is an object, then there exists a set \(\{a\}\) whose only element is \(a\), i.e. for every object \(y\) we have \(y\in\{a\}\) iff \(y=a\). \(\{a\}\) is referred to as a singleton set.
		\item (Pairwise union) Given any two sets \(A\) and \(B\), there exists a set \(A\cup B\), called the union of \(A\) and \(B\), which consists of all the elements which belong to \(A\) or \(B\). In other words,
			\begin{equation*}
				x\in A\cup B\Leftrightarrow(x\in A\vee x\in B).
			\end{equation*}
		\item (Axiom of specification) Let \(A\) be a set, and for each \(x\in A\) let \(P(x)\) be a property pertaining to \(x\). Then there exists a set \(\{x\in A\,|\,P(x)\}\) whose elements are precisely the elements \(x\) in \(A\) for which \(P(x)\) is true.
		\item (Replacement) Let \(A\) be a set. For any object \(x\in A\) and any object \(y\), suppose we have a property \(P(x,y)\) that is true for at most one \(y\) for each \(x\in A\). Then
			\begin{equation*}
				z\in\{y\,|\,P(x,y),\, x\in A\}\Leftrightarrow P(x,z).
			\end{equation*}
		\item (Infinity) There exists a set \(\mathbb{N}\), whose elements are called natural numbers, as well as an object \(0\in\mathbb{N}\), and an object \(N++\) assigned to every natural number \(n\in\mathbb{N}\), such that the Peano axioms hold.
	\end{enumerate}
\end{definition}
\begin{lemma}
	Let \(A\) be a non-empty set. Then there exists an object \(x\) such that \(x\in A\).
\end{lemma}
\begin{definition}[Subset]
	Let \(A,B\) be sets. We say that \(A\) is a subset of \(B\), denoted \(A\subseteq B\), iff every element of \(A\) is also an element of \(B\). We say that \(A\) is a proper subset of \(B\), denoted \(A\subsetneq B\), if \(A\subseteq B\) and \(A\neq B\).
\end{definition}
\begin{definition}[Intersection]
	The intersection \(S_1\cap S_2\) of two sets is defined to be the set
	\begin{equation*}
		S_1\cap S_2\coloneq\{x\in S_1\,|\,x\in S_2\}.
	\end{equation*}
\end{definition}
\begin{definition}[Disjoint]
	Two sets are disjoint if \(A\cap B=\emptyset\).
\end{definition}
\begin{definition}[Difference set]
	If \(A\) and \(B\) are sets, the set \(A\setminus B\) is the set \(A\) with any elements of \(B\) removed, i.e.
	\begin{equation*}
		A\setminus B\coloneq\{x\in A\,|\,x\notin B\}.
	\end{equation*}
\end{definition}
\begin{proposition}
	Let \(A,B,C\) be subsets of set \(X\).
	\begin{enumerate}
		\item (Minimal element) \(A\cup\emptyset=A\) and \(A\cap\emptyset=\emptyset\).
		\item (Maximal element) \(A\cup X=X\) and \(A\cap X=A\).
		\item (Identity) \(A\cap A=A\) and \(A\cup A=A\).
		\item (Commutativity) \(A\cup B=B\cup A\) and \(A\cap B=B\cap A\).
		\item (Associativity) \((A\cup B)\cup C=A\cup(B\cup C)\) and \((A\cap B)\cap C=A\cap(B\cap C)\).
		\item (Distributivity) \(A\cap (B\cup C)=(A\cap B)\cup(A\cap C)\) and \(A\cup(B\cap C)=(A\cup B)\cap(A\cup C)\).
		\item (Partition) \(A\cup(X\setminus A)=X\) and \(A\cap (X\setminus A)=\emptyset\).
		\item (De Morgan Laws) \(X\setminus(A\cup B)=(X\setminus A)\cap(X\setminus B)\) and \(X\setminus (A\cap B)=(X\setminus A)\cup(X\setminus B)\).
	\end{enumerate}
\end{proposition}
\subsection{Functions}
\begin{definition}[Function]
	
\end{definition}
\end{document}

