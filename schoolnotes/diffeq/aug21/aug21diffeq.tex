\documentclass{article}
\usepackage[left=3cm,right=3cm,top=2cm,bottom=2cm]{geometry}
\usepackage{amsmath}
\usepackage{amsfonts}
\setlength{\parindent}{0mm}

\begin{document}
\title{Seperable Differential Equations}
\author{Samuel Lindskog}
\date{August 21, 2023}
\maketitle
\renewcommand{\abstractname}{}

\setcounter{secnumdepth}{2}

\section{Form of Differential Equations}
Differential equations will be of the form:
\begin{displaymath}
	\frac{dy}{dx}=y'=G(y)
\end{displaymath}
Where \(y\) is a function of \(x\)

\section{Seperable Differential Equations}
Most differential equations are not readily solvable. A common solvable form is the seperable differential equation:
\begin{displaymath}
	\frac{dy}{dx}=y'=G(y)F(x)
\end{displaymath}
Method of finding solutions for this:
\begin{displaymath}
	\int\frac{dy}{G(y)}=\int F(x)dx\text{\;\;i.e.}\int G(y)^{-1}dy=\int F(x)dx
\end{displaymath}
An example of a seperable differential equation:
\begin{displaymath}
	y'=\frac{x+1}{y^2+y+2}
\end{displaymath}
This can be solved via:
\begin{align*}
	y'=\frac{dy}{dx}&=(x+1)(y^2+y+2)^{-1}\\
	&=dy(y^2+y+2)=dx(x+1)\\
	&=dy(poopypants)
\end{align*}
\subsection*{Interesting Thoughts}
\begin{align*}
	&y=f(x)\text{\;and\;}y'=\lim_{h\rightarrow 0}\frac{f(x+h)-f(x)}{h}\\
	&dx:h\mapsto h\text{\;and\;}dy:h\mapsto h
\end{align*}
\(dx\) and \(dy\) represent the instantaneous rate of change of \(x\) or \(y\) with respect to themselves. This is why they are identity functions mapping \(h\mapsto h\). They form the bridge between the infinitesimal(instantaneous rate of change) and what is algebraically maniputable.
\begin{align*}
	&\text{With \(y=f(x)\), \(dy\) can be defined by\;} dy:x\mapsto f'(x)dx\\
	&\frac{dy(h;x)}{dx(h)}=\frac{f'(x)dx(h)}{dx(h)}=f'(x)
\end{align*}
\end{document}

