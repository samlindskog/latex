\documentclass[nobib,notoc]{tufte-handout}
\usepackage[utf8]{inputenc}
\usepackage[english]{babel}
\usepackage{amsmath}
\usepackage{amsthm}
\usepackage{amsfonts}
\usepackage{hyperref}
\usepackage{mathrsfs}
\usepackage{IEEEtrantools}
\usepackage{enumitem}

\renewcommand{\IEEEQED}{\IEEEQEDopen}

\begin{document}

\theoremstyle{definition}\newtheorem{defi}{Definition}[section]
\theoremstyle{definition}\newtheorem{axiom}{Axiom}[section]
\theoremstyle{definition}\newtheorem{thm}{Theorem}[section]
\theoremstyle{definition}\newtheorem{cor}{Corollary}[section]
\theoremstyle{definition}\newtheorem{lem}{Lemma}[section]
\theoremstyle{remark}\newtheorem*{notat}{Notation}
\theoremstyle{remark}\newtheorem*{rema}{Remark}
\theoremstyle{definition}\newtheorem{problem}{Problem}
%\renewcommand{\theproblem}{\arabic{problem}}
\newenvironment{prob}[1]{\protect\setcounter{problem}{#1}\addtocounter{problem}{-1}\begin{problem}}{\end{problem}}

\title{Differential Equations}
\author{Samuel Lindskog}
\maketitle

\setcounter{section}{1}

\section{First-Order Differential Equations}
\begin{defi}[Order]
	The order of a differential equation is the order of the highest dericatice appearing in the equation.
\end{defi}
\begin{defi}[Normal form]
	The normal form of a first-order equation is a function \(f\) which relates a function \(x=x(t)\) with its first derivative.
	\begin{equation*}
		x'=f(t,x).
	\end{equation*}
	A function \(x=x(t)\) is a solution of this equation on the time interval \(I:a<t<b\) if it is differentiabe on \(I\) and, when substituted into the equation, it satisfies the equation identically for every \(t\in I\), i.e.
	\begin{equation*}
		x'(t)=f(t,x(t)),\text{ for every }t\in I.
	\end{equation*}
	In other words to check if a function is a solution, substitute the function in question into the differential equation and check that it reduces to an identity.
\end{defi}
\begin{defi}[Initial value problem]
	Subjecting a differential equation involving \(x(t)\) and its derivatives to a condition \(x(t_0)=x_0\) is called an initial value problem. The interval of existance of an IVP is the largest time interval where the solution is valid.\footnote{A solution to an IVP is called a particular solution.}
\end{defi}
\begin{defi}[General solution]
	The infinite set of solutions of a first-order equation is called the general solution of the equation.
\end{defi}
\begin{defi}[Nullclines and isoclines]
	The sets of points \((t,x)\) where the slope field is zero are called nullclines\footnote{These constant solutions \(f(t,x)=k\) for some constant \(k\) are called equilibrium solutions.}, i.e. where
	\begin{equation*}
		x'=f(t,x)=0.
	\end{equation*}
	The set of points \(t,x\) where \(f(t,x)=k\) for some constant \(k\) are called isoclines.
\end{defi}
\begin{thm}[Fundamental theorem of calculus]
	\,
	\begin{equation*}
		\frac{d}{dt}\int_{a}^{t}g(s)ds=g(t)
	\end{equation*}
\end{thm}
\begin{defi}[Seperable equation]
	A differential equation of the form
	\begin{equation*}
		x'=f(x)g(t)
	\end{equation*}
	is called a seperable equation. We can obtain \(x\) through the following procedure:
	\begin{IEEEeqnarray*}{rCl}
		\frac{dx}{dt}&=&f(x)g(t)\\
		\int\frac{1}{f(x)}\frac{dx}{dt}dt&=&\int g(t)dt\\
		\int\frac{1}{f(x)}dx&=&\int g(t)dt+C
	\end{IEEEeqnarray*}
	The final form of the seperable equation is made possible by the chain rule, and a helpful step forward towards finding the solution is
	\begin{equation*}
		e^{\int g(t)dt}=f(x)+c.
	\end{equation*}
\end{defi}
\begin{defi}[Linear equation]
	A differential equation of the form
	\begin{equation*}
		x'+p(t)x=q(t)
	\end{equation*}
	is called a first-order linear equation\footnote{This is also called the normal form of a first-order linear equation.}. If a first-order equation can not be put into this form, the equation is called nonlinear.
\end{defi}
\begin{defi}[Homogeneous equation]
	A first-order linear differential equation is called homogeneous\footnote{A homogenous equation is seperable.} if it is of the form
	\begin{equation*}
		x'+p(t)x=0.
	\end{equation*}
The solution is
\begin{equation*}
	Ce^{-\int p(t)dt}.
\end{equation*}
\end{defi}
\begin{defi}[Integrating factor]
	A function \(\mu(t)\) exists such that
	\begin{equation*}
		\mu(t)(x'+p(t)x)=(\mu(t)x)'.
	\end{equation*}
	The function \(\mu(t)\) is called an integrating factor and is given by
	\begin{equation*}
		\mu(t)=e^{\int p(t)dt}
	\end{equation*}
	This can be used to solve linear equations by multiplying both sides by the integrating factor.
\end{defi}
\begin{defi}[Autonomous equation]
	An autonomous differential equation is a differential equation with no explicit time dependence, i.e.
	\begin{equation*}
		\frac{dx}{dt}=f(x).
	\end{equation*}
	As described above, constant solutions to an autonomous equation are called steady-state or equilibrium solutions.
\end{defi}
\begin{defi}[Stable and unstable equilibrium]
	For stable equilibrium solutions, solutions with values of \(x\) close to the phase-line converge to the phase line. For unstable equilibrium solutions are not stable.\footnote{If solutions near the phase line converge or diverge depending on how they approach, the solution is semi-stable. If all perturbations converge to the phase line, the solution is globally stable.} The roots of \(f(x)=0\) are the equilibrium solutions.
\end{defi}
\begin{thm}
	Let \(x^*\) be an isolated critical point, or equilibrium, for the autonomous equation
	\begin{equation*}
		\frac{dx}{dt}=f(x).
	\end{equation*}
	If \(f'(x^*)<0\), then \(x^*\) is stable. If \(f'(x^*)>0\), then \(x^*\) is unstable. If \(f'(x^*)=0\) then higher derivatives must be analysed to find information about stability.
\end{thm}
\begin{thm}[Existence and uniqueness]
	Assume the function \(f(t,x)\) and its partial derivative \(f_x(t,x)\) are continuous in a rectangle \(a<t<b, c<x<d\). Then, for any value \(t_0\) in \(a<t<b\) and \(x_0\) in \(c<x<d\), the initial value problem
	\begin{IEEEeqnarray*}{c}
		x'=f(t,x)\\
		x(t_0)=x_0
	\end{IEEEeqnarray*}
	has a unique solution valid on some open interval \(a<\alpha<t<\beta<b\) containing \(t_0\).
\end{thm}
\section{Second-order linear equations}
\begin{rema}
	Chapter two in the book deals with second-order differential equations of the form
	\begin{equation}
		\label{rema:eqn}
		ax''+bx'+cx=f(t).
	\end{equation}
\end{rema}
\begin{defi}[Hooke's law]
	Let \(x\) be displacement from equilibrium and \(k\) be the spring constant. Then
	\begin{equation*}
		F_s=-kx
	\end{equation*}
\end{defi}
\begin{defi}[Spring-mass equation]
	The spring-mass equation relates the acceleration of a mass on a spring with the force applied by the spring given by Hooke's law:
	\begin{equation*}
		mx''=-kx.
	\end{equation*}
	For initial conditions \(x(0)=x_0\) and \(x'(0)=0\) we find \(x(t)\) is
	\begin{equation*}
		x(t)=x_0\text{cos}\,\sqrt{k/m}t.
	\end{equation*}
\end{defi}
\begin{defi}[Damped Oscillator]
	If there is friction as the mass moves, the frictional force is a function of the velocity \(x'\) and the damping coefficient \(\gamma\)
	\begin{equation*}
		F_d=-\gamma x'.
	\end{equation*}
	Therefore the equation of motion is
	\begin{equation*}
		mx''=-\gamma x'-kx.
	\end{equation*}
\end{defi}
\begin{rema}
	The damped spring-mass equation has the form\footnote{An equation of this form is called a homogenour linear equation with constant coefficients.}
	\begin{equation*}
		ax''+bx'+cx=0.
	\end{equation*}
	For such an equation, there are always exactly two independent solutions \(x_1(t)\) and \(x_2(t)\), and so the general solution \(\phi(t)\)is of the form
	\begin{equation*}
		\phi(t)=c_1x_1(t)+c_2x_2(t).
	\end{equation*}
\end{rema}
\begin{defi}[Characteristic equation]
	To solve equation \ref{rema:eqn}, first note that \(x(t)=e^{\lambda t}\) for some constant \(\lambda\). Substituting \(e^{\lambda t}\), we can solve for \(\lambda\) with the characteristic equation\footnote{The roots of this equation are called eigenvalues.}
	\begin{equation*}
		a\lambda^2+b\lambda+c=0.
	\end{equation*}
	The values of \(\lambda\) can be real or complex. If \(b^2-4ac>0\), then there are two real unequal eigenvalues, and hence there are two indpendent solutions, so the general solution is
	\begin{equation*}
		x(t)=c_1e^{\lambda_1 t}+c_2e^{\lambda_2 t}.
	\end{equation*}
	In this case, if \(\lvert\lambda_1\rvert=\lvert\lambda_2\rvert\), then the general solution is
	\begin{equation*}
		x(t)=c_1e^{\alpha t}+c_2e^{-\alpha t}
	\end{equation*}
	Which is exponential. If \(\lvert \lambda_1\rvert=a\), this equation can be written in terms of hyperbolic functions cosh and sinh as
	\begin{equation*}
		x(t)=c_1\text{cosh}\,at+c_2\text{sinh}\,at
	\end{equation*}
	If \(b^2-4ac=0\) then the general solution is
	\begin{equation*}
		x(t)=c_1e^{\lambda t}+c_2te^{\lambda t}.
	\end{equation*}
	If \(b^2-4ac<0\) then the eigenvalues are complex.
\end{defi}
\begin{defi}[Euler's formula]
	\,
	\begin{equation*}
		e^{i\beta t}=\text{cos}\,\beta t+i\text{sin}\,\beta t.
	\end{equation*}
\end{defi}
\begin{thm}
	\label{thm:csolutions}
If \(x(t)=g(t)+ih(t)\) is a complex-valued solution of differential equation \ref{rema:eqn}, then its real and imaginary parts \(x_1(t)=g(t)\) and \(x_2(t)=h(t)\) are real-valued solutions.
\end{thm}
\begin{rema}
	As a consequence of theorem \ref{thm:csolutions}, if \(\lambda_1=\alpha+i\beta\), then the general solution to equation \ref{rema:eqn} is
	\begin{equation*}
		x(t)=c_1e^{\alpha t}\text{cos}\,\beta t+c_2e^{\alpha t}\text{sin}\,\beta t.
	\end{equation*}
	If \(\alpha<0\), these solutions represent decaying oscillations, and if \(\alpha>0\) then these solutions represent growing oscillations. If \(\alpha=0\) then the solutions are purely oscillatory with frequency \(\beta\) and period \(2\pi/\beta\).
\end{rema}
\end{document}
