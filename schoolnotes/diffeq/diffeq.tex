\documentclass[nobib,notoc]{tufte-handout}
\usepackage[utf8]{inputenc}
\usepackage[english]{babel}
\usepackage{amsmath}
\usepackage{amsthm}
\usepackage{amsfonts}
\usepackage{hyperref}
\usepackage{mathrsfs}
\usepackage{IEEEtrantools}
\usepackage{enumitem}

\renewcommand{\IEEEQED}{\IEEEQEDopen}

\begin{document}

\theoremstyle{definition}\newtheorem{defi}{Definition}[section]
\theoremstyle{definition}\newtheorem{axiom}{Axiom}[section]
\theoremstyle{definition}\newtheorem{thm}{Theorem}[section]
\theoremstyle{definition}\newtheorem{cor}{Corollary}[section]
\theoremstyle{definition}\newtheorem{lem}{Lemma}[section]
\theoremstyle{remark}\newtheorem*{notat}{Notation}
\theoremstyle{remark}\newtheorem*{rema}{Remark}
\theoremstyle{definition}\newtheorem{problem}{Problem}
%\renewcommand{\theproblem}{\arabic{problem}}
\newenvironment{prob}[1]{\protect\setcounter{problem}{#1}\addtocounter{problem}{-1}\begin{problem}}{\end{problem}}

\title{Differential Equations}
\author{Samuel Lindskog}
\maketitle

\setcounter{section}{1}

\section{First-Order Differential Equations}
\begin{defi}[Order]
	The order of a differential equation is the order of the highest dericatice appearing in the equation.
\end{defi}
\begin{defi}[Normal form]
	The normal form of a first-order equation is a function \(f\) which relates a function \(x=x(t)\) with its first derivative.
	\begin{equation*}
		x'=f(t,x).
	\end{equation*}
	A function \(x=x(t)\) is a solution of this equation on the time interval \(I:a<t<b\) if it is differentiabe on \(I\) and, when substituted into the equation, it satisfies the equation identically for every \(t\in I\), i.e.
	\begin{equation*}
		x'(t)=f(t,x(t)),\text{ for every }t\in I.
	\end{equation*}
	In other words to check if a function is a solution, substitute the function in question into the differential equation and check that it reduces to an identity.
\end{defi}
\begin{defi}[Initial value problem]
	Subjecting a differential equation involving \(x(t)\) and its derivatives to a condition \(x(t_0)=x_0\) is called an initial value problem. The interval of existance of an IVP is the largest time interval where the solution is valid.
\end{defi}
\begin{defi}[General solution]
	The infinite set of solutions of a first-order equation is called the general solution of the equation.
\end{defi}
\end{document}
