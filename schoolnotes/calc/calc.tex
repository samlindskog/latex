\documentclass[nobib,notoc]{tufte-handout}
\usepackage{amsmath}
\usepackage{amsthm}
\usepackage{amsfonts}
\usepackage{hyperref}
\usepackage{mathrsfs}
\usepackage{IEEEtrantools}

\renewcommand{\IEEEQED}{\IEEEQEDopen}

\begin{document}

\theoremstyle{definition}\newtheorem{defi}{Definition}[section]
\theoremstyle{definition}\newtheorem{thm}{Theorem}[section]
\theoremstyle{definition}\newtheorem{cor}{Corollary}[section]
\theoremstyle{definition}\newtheorem{lem}{Lemma}[section]
\theoremstyle{remark}\newtheorem*{notat}{Notation}
\theoremstyle{remark}\newtheorem*{rema}{Remark}
\theoremstyle{definition}\newtheorem{problem}{Problem}
%\renewcommand{\theproblem}{\arabic{problem}}
\newenvironment{prob}[1]{\protect\setcounter{problem}{#1}\addtocounter{problem}{-1}\begin{problem}}{\end{problem}}

\title{"Calculus: Early Trancendentals" Notes}
\author{Samuel Lindskog}
\maketitle

\setcounter{section}{1}
\setcounter{tocdepth}{1}

\section{Limits}
\begin{defi}[Average Rate of Change]
	The average rate of change of \(y=f(x)\) with respect to \(x\) over the interval \((x_1, x_2)\) is:
	\begin{equation*}
		\frac{\Delta y}{\Delta x}=\frac{f(x_2)-f(x_1)}{x_2-x_1}=\frac{f(x_1+h)-f(x_1)}{h}
	\end{equation*}
\end{defi}
\begin{thm}[Limit Laws]
	If \(L,M,c,k\) are real numbers and \(\lim_{x\rightarrow c}f(x)=L\), and \(\lim_{x\rightarrow c}g(x)=M\), then:
	\begin{IEEEeqnarray*}{lCl}
		\text{Sum rule:}&\qquad\qquad&\lim_{x\rightarrow c}(f(x)+g(x))=L+M\\
		\text{Difference Rule:}&&\lim_{x\rightarrow c}(f(x)-g(x))=L-M\\
		\text{Constant Multiple Rule:}&&\lim_{x\rightarrow c}(k\cdot f(x))=k\cdot L\\
		\text{Product Rule:}&&\lim_{x\rightarrow c}(f(x)\cdot g(x))=L\cdot M\\
		\text{Quotient Rule:}&&\lim_{x\rightarrow c}\frac{f(x)}{g(x)}=\frac{L}{M},\;M\neq 0\\
		\text{Power Rule:}&&\lim_{x\rightarrow c}\bigl[f(x)\bigr]^n=L^n,\;n\text{ a positive integer}\\
		\text{Root Rule:}&&\lim_{x\rightarrow c}\sqrt[n]{f(x)}=\sqrt[n]{L},\;\text{ n a positive integer}
	\end{IEEEeqnarray*}
(For the root rule, if \(n\) is even, we assume that \(f(x)\geq 0\) for \(x\) in an interval containing \(c\).)
\end{thm}
\begin{thm}[Sandwich theorem]
	Suppose that \(g(x)\leq f(x)\leq h(x)\) for all \(x\) in some neighborhood of \(c\). Suppose also that:
	\begin{equation*}
		\lim_{x\rightarrow c}g(x)=\lim_{x\rightarrow c}f(x)=L
	\end{equation*}
	Then \(\lim_{x\rightarrow c}f(x)=L\).
\end{thm}
\begin{defi}[Limit]
	Let \(f(x)\) be defined on an open neighborhood of \(c\) not necessarily including \(c\). We say that the limit of \(f(x)\) as \(x\) approaches \(c\) is the number \(L\) if for every number \(\epsilon>0\), there exits a corresponding number \(\delta>0\) such that:
	\begin{equation*}
		0<\lvert x-c\rvert<\delta\Rightarrow\lvert f(x)-L\rvert<\epsilon
	\end{equation*}
\end{defi}
\section{Extreme Values}
\begin{defi}[Absolute maximum and minimum]
	Let \(f\) be a function with domain \(D\). Then \(f\) has an absolute maximum value on \(D\) at point \(c\) if
	\begin{equation*}
		f(x)\leq f(c)\qquad\text{for all \(x\) in \(D\)}
	\end{equation*}
	and an absolute minimum value on \(D\) at \(c\) if
	\begin{equation*}
		f(x)\geq f(c)\qquad\text{for all \(x\) in \(D\).}
	\end{equation*}
\end{defi}
\begin{thm}[Extreme value theorem]
	If \(f\) is continuous on a finite closed interval \([a,b]\), then \(f\) attains both an absolute maximum value \(M\) and an absolute minimum value \(m\) in \([a,b]\).
\end{thm}
\begin{defi}[Local maximum and minimum]
	A function \(f\) has a local maximum value (or minimum) at a point \(c\) within its domain \(D\) if \(f(x)\leq f(c)\) (or \(f(x)\geq f(c)\)) for all \(x\in D\) lying in some open interval containing \(c\).\footnote{Local extrema are also called relative extrema.}
\end{defi}
\begin{rema}
	A set of all local maxima (minima) will automatically include the absolute maximum (minimum) if there is one. 
\end{rema}
\begin{thm}[First derivative theorem for local extreme values]
	If \(f\) has a local maximum or minimum value at an interior point \(c\) of its domain, and if \(f'\) is defined at \(c\), then
	\begin{equation*}
		f'(c)=0.
	\end{equation*}
\end{thm}
\begin{rema}
	The only places where a function \(f\) can possibly have an extreme value (local or global) are
	\begin{enumerate}
		\item interior points where \(f'=0\),
		\item interior points where \(f'\) is undefined,
		\item endpoints of the domain of \(f\)
	\end{enumerate}
\end{rema}
\begin{defi}[Critical point]
	An interior point of the domain of a function \(f\) where \(f'\) is zero or undefined is a critical point of \(f\).
\end{defi}
\end{document}
