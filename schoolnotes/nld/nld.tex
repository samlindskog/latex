\documentclass{article}
\usepackage{settings}

\geometry{
a4paper,
total={140mm,257mm},
left=35mm,
top=20mm,
}

\title{Nonlinear Dynamics}
\author{Samuel Lindskog}

\begin{document}
\maketitle
\addtocontents{toc}{\protect\hypertarget{toc}{}}
\tableofcontents
\pagenumbering{gobble}
\clearpage
\pagenumbering{arabic}
\setcounter{page}{1}

\section{Flows on the line}
\subsection{Introduction}
\begin{definition}[Fixed points]
	A fixed point on a phase diagram is a point in which there is no flow, i.e. \(x'=0\). Fixed points represent equilibrium solutions.
\end{definition}
\begin{definition}[Phase point]
	A phase point is an imaginary particle placed at a point \(x_0\) from which we can observe how it is carried along with the "flow". As time increases, the phase point moves along the \(x\)-axis according to some function \(x(t)\). \(x(t)\) is called the trajectory based at \(x_0\).
\end{definition}
\begin{theorem}
	Consider the IVP
	\begin{IEEEeqnarray*}{l}
		x'=f(x),\\
		x(0)=x_0.
	\end{IEEEeqnarray*}
	If \(f(x)\) and \(f'(x)\) are continuous on an open interval \(R\) of the \(x\)-axis, and \(x_0\in R\), then the initial value problem has a unique solution on some time interval \(-\tau,\tau\) about \(t=0\).
\end{theorem}
\end{document}
