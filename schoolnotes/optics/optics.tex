\documentclass[nobib,notoc]{tufte-handout}
\usepackage{amsmath}
\usepackage{amsthm}
\usepackage{amsfonts}
\usepackage{hyperref}
\usepackage{mathrsfs}
\usepackage{physics}
\usepackage{IEEEtrantools}
\usepackage{enumitem}

\renewcommand{\IEEEQED}{\IEEEQEDopen}

\begin{document}

\theoremstyle{definition}\newtheorem{defi}{Definition}[section]
\theoremstyle{definition}\newtheorem{axiom}{Axiom}[section]
\theoremstyle{definition}\newtheorem{thm}{Theorem}[section]
\theoremstyle{definition}\newtheorem{cor}{Corollary}[section]
\theoremstyle{definition}\newtheorem{lem}{Lemma}[section]
\theoremstyle{remark}\newtheorem*{notat}{Notation}
\theoremstyle{remark}\newtheorem*{rema}{Remark}
\theoremstyle{definition}\newtheorem{problem}{Problem}
%\renewcommand{\theproblem}{\arabic{problem}}
\newenvironment{prob}[1]{\protect\setcounter{problem}{#1}\addtocounter{problem}{-1}\begin{problem}}{\end{problem}}

\title{Physics 3}
\author{Samuel Lindskog}
\maketitle

\setcounter{section}{1}
\setcounter{tocdepth}{1}

\section{Mechanical waves}
\begin{defi}[Mechanical wave]
	A mechanical wave is a disturbance that travels through some medium or substance called the \emph{medium} for the wave.
\end{defi}
\begin{defi}[Transverse and longitudinal waves]
	A transverse wave is when the displacements of the medium are perpendicular to the direction of travel of the wave. The displacement of the medium in longitudinal waves is parallel to the direction of travel of the wave.
\end{defi}
\begin{defi}[Frequence and other stuff]
	\,
	\begin{IEEEeqnarray*}{rCl}
		f&=&\text{\,frequency}\\
		\omega &=&2\pi f=\text{\,angular frequency}\\
		T&=&\frac{1}{f}=\frac{2\pi}{\omega}=\text{\,period}
	\end{IEEEeqnarray*}
\end{defi}
\begin{defi}[Wave speed]
	Wave speed \(v\) is described by the equation
	\begin{equation*}
		v=\lambda f
	\end{equation*}
	Where \(\lambda\) is the wavelength. In this section, we will only be discussing waves with speed dependent on the mechanical properties of the medium.
\end{defi}
\begin{defi}[Compression and rarefaction]
	Compression and rarefaction are the names f.
\end{defi}
\begin{defi}[Wave number]
	The wave number \(k\) is given by the following equation:
	\begin{equation*}
		k=\frac{2\pi}{\lambda}
	\end{equation*}
	We can now express \(\omega\) in terms of \(k\) and \(v\) as
	\begin{equation*}
		\omega =vk
	\end{equation*}
\end{defi}
\begin{defi}[Wave equation]
	The wave equation can be expressed in different ways.
	\begin{IEEEeqnarray*}{rCl}
		y(x,y)&=&A\,\text{cos}\bigg[\omega\bigg(\frac{x}{v}-t\bigg)\bigg]\\
		y(x,y)&=&A\,\text{cos}\bigg[2\pi\bigg(\frac{x}{\lambda}-\frac{t}{T}\bigg)\bigg]\\
		y(x,y)&=&A\,\text{cos}(kx-\omega t)
	\end{IEEEeqnarray*}
	For a wave travelling in the negative \(x\) direction, the wave equation can be expressed as
	\begin{equation*}
		A\,\text{cos}(kx+\omega t)
	\end{equation*}
\end{defi}
\begin{defi}[Phase]
	The phase of a wave is equal to
	\begin{equation*}
		kx\pm\omega t
	\end{equation*}
\end{defi}
\begin{defi}[Transverse velocity]
	The transverse velocity of a particle in a transverse wave is given by the equation
	\begin{equation*}
		v_y(x,t)=\pdv{y(x,t)}{t}=\omega A\,\text{sin}(kx-\omega t)
	\end{equation*}
\end{defi}
\begin{defi}[Partial derivatives of wave equation]
	As a consequence of the fact that \(\omega=kv\), we have
	\begin{equation*}
		\pdv[2]{y(x,t)}{x}=\frac{1}{v^2}\pdv[2]{y(x,t)}{t}
	\end{equation*}
\end{defi}
\begin{defi}[Momentum]
	Momentum \(p\) can be expressed by the equation
	\begin{equation*}
		p=mv
	\end{equation*}
	In this case, we are referring to the magnitude of momentum \(\vec{p}\)
\end{defi}
\begin{defi}[Impulse]
	Impulse \(I\) can be expressed by the equation
	\begin{equation*}
		I=F\Delta t
	\end{equation*}
	In this case, we are referring to the magnitude of impulse \(\vec{J}\)
\end{defi}
\begin{defi}[Impulse-momentum theorem]
	\,
	\begin{equation*}	
		\vec{J}=\vec{p_2}-\vec{p_1}=\Delta\vec{p}
	\end{equation*}
\end{defi}
\begin{defi}[Specific Impulse]
	Specific Impulse\footnote{This is here because I like rockets.} \(I_{sp}\) is given by the equation
	\begin{equation*}
		I_{sp}=\frac{F\Delta t}{m}
	\end{equation*}
	where \(m\) is the mass of fuel.
\end{defi}
\begin{defi}[Speed of a transverse wave on a string]
	The speed \(v\) of a transverse wave in a string with tension \(F\) and mass per unit length \(\mu\) is given by the following equation:
	\begin{equation*}
		v=\sqrt\frac{F}{\mu}
	\end{equation*}
\end{defi}
\begin{defi}[Intensity]
	The intensity \(I\) of a wave is the time average rate at which energy is transported by the wave, per unit area, across a surface perpendicular to the direction of propagation. It is usually measured in watts per square meter, or \(W/m^2\).
\end{defi}
\begin{defi}[Boundary conditions]
	The conditions at the end of a string such as a rigid support or the complete absence of transverse force are called boundary conditions.
\end{defi}
\begin{defi}[Standing wave]
	A wave pattern that appears to remain in the same position is called a standing wave. The points of zero displacement in the wave pattern are called \emph{nodes}, and the points of maximum displacement are called \emph{antinodes}.
\end{defi}
\section{Electromagnetic Waves}
\begin{defi}[Maxwell's equations]
	\,
	\begin{enumerate}[label=\phantom{-}]
		\item \(\oint \vec{E}\cdot d\vec{A}=\frac{Q_{encl}}{\epsilon_0}\)\text{ (Gauss's law)}
		\item \(\oint \vec{B}\cdot d\vec{A}=0\)\text{ (Gauss's law for magnetism)}
		\item \(\oint \vec{E}\cdot d\vec{l}=-\frac{d\Phi_B}{dt}\)\text{ (Faraday's law)}
		\item \(\oint \vec{B}\cdot d\vec{l}=\mu_0(i_C+\epsilon_0\frac{d\Phi_E}{dt})_{encl}\)\text{ (Apere's law)}
	\end{enumerate}
\end{defi}
\begin{defi}[Electromagnetic wave in vacuum]
	The electric field magnitude \(E\), the speed of light \(c\), and the magnetic field magnitude \(B\) are related by the equation
	\begin{equation*}
		E=cB
	\end{equation*}
\end{defi}
\section{The nature and propogation of light}
\begin{defi}[c]
	The speed of light, \(c\) is
	\begin{equation*}
		c=2.99792458\times10^8 m/s.
	\end{equation*}
\end{defi}
\begin{defi}[Index of refraction]
	The index of refraction \(n\) is the ratio between the speed of light in vacuum \(c\) and the speed of light in the material
	\begin{equation*}
		n=\frac{c}{v}.
	\end{equation*}
	Light always travels more slowly in a material than in a vacuum.
\end{defi}
\begin{rema}
	The incident, reflected, and refracted rays and the normal to the surface all lie in the same plane. This plane is called the \emph{plane of incidence}, is perpendicular to the plane of the boundary surface between the two materials. The angle of reflection \(\theta_r\) is equal to the angle of incidence \(\theta_a\) for all wavelengths and for any pair of materials. These properties are referred to collectively as the \emph{law of reflection}.
\end{rema}
\begin{defi}[Snell's law]
	For a given pair of materials \(a\) and \(b\), on opposite sides of the interface, the ratio of the sines of the angles \(\theta_a\) and \(\theta_b\), where both angles are measured from the normal to the surface, is equal to the inverse ratio of the two indexes of refraction, i.e.
	\begin{equation*}
		n_a\text{sin}\,\theta_a=n_b\text{sin}\,\theta_b
	\end{equation*}
\end{defi}
\begin{defi}[Wavelength of light in a material]
	The wavelength of light in a material, \(\lambda\), is described by
	\begin{equation*}
		\lambda=\frac{\lambda_0}{n}
	\end{equation*}
	where \(\lambda_0\) is the wavelength of light in vacuum, and \(n\) is the index of refraction of the material.
\end{defi}
\begin{defi}[Critical angle]
	The angle of incidence for which the refracted ray emerges tangent to the surface is called the critical angle, denoted \(\theta_{crit}\).
	\begin{equation*}
		\text{sin}\,\theta_{crit}=\frac{n_b}{n_a}.
	\end{equation*}
If the angle of incidence is larger than the critical angle, the ray cannot pass into the upper material, and is reflected at the boundary surface. This is called \emph{total internal reflection}, and occurs only when a ray in material \(a\) is incident on a second material \(b\) whose index of refraction is smaller than that of material \(a\).
\end{defi}
\begin{rema}
	The speed of light in a material substance is different for different wavelengths. Therefore the index of refraction of a material depends on the wavelength.\footnote{This dependence of wave speed and index of refraction is called \emph{dispersion}.} It is this which is the driving force behind rainbows.
\end{rema}
\begin{defi}[Polarization]
	When a wave has only \(y\)-displacement, we say that it is linearly polarized in the \(y\)-direction; a wave with only \(z\)-displacements is linearly polarized in the \(z\)-direction. We always define the direction of polarization of an electromagneic wave to be the direction of the electric-field vector \(\vec{E}\), not the magnetic field.
\end{defi}
\begin{defi}[Polarizing angle]
	Unpolarized light can be polarized either partially or totally by reflection. If the angle of incicence is not equal to the polarizing angle, the waves for which the electric field vector \(\vec{E}\) is perpendicular to the plane of incidence, i.e. parallel to the reflecting surface, are reflected more strongly that those for which \(\vec{E}\) lies in the place of incidence, thereby partially polarizing the light.\footnote{Polarization by reflection is the reason polarizing filters are used in sunglasses. The polarizing axis of the lens material is made parallel to the angle of incidence (vertically), which reduces glare caused by reflection of light off of surfaces parallel to the ground).} If the light reflects at the polarizing angle, the light for which \(\vec{E}\) lies in the plane of incidence is completely refracted. In this case the light which lies perpenducular to the plane of incidence is partially reflected and partially refracted. The reflected light is therefore completely polarized perpendicular to the plane of incidence. The refracted light is partially polarized parallel to the plane of incidence. The polarizing angle is known as brewsters angle.
\end{defi}
\begin{defi}[Brewster's law]
	The when the angle of incidence is equal to the polarizing angle \(\theta_p\), the refracted and reflected rays are perpendicular to each other. Therefore \(\theta_p\) can be calculated via brewsters law
	\begin{equation*}
		\text{tan}\,\theta_p=\frac{n_b}{n_a}.
	\end{equation*}
\end{defi}
\begin{defi}[Scattering]
	Light is absorbed and re-radiated by a material through a process called scattering.
\end{defi}
\section{Geometric Optics}
\begin{defi}[Object and image point]
	An object is anything from which light rays radiate, and the object point is where the objects rays originate from. The image point is the point from which the rays of light appear to have originated from after reflection/refraction. We say that the reflecting surface forms an image of the object point. If the outgoing rays don't actually pass through the image point, we call the image a virtual image.
\end{defi}
\begin{defi}[Virtual and real images]
	If outgoing rays don't actually pass through the image point, we call the image a virtual image. If they do, the resulting image is a real image.
\end{defi}
\begin{rema}
	Sign rules for object distance. Incoming refers to light moving towards the reflecting/refracting surface.
	\begin{enumerate}
		\item Sign rule for the object distance: When the object is on the same side of the reflecting or refracting surface as the incoming light, object distance \(s\) is positive\footnote{Incoming light refers to light coming from an image towards a reflecting/refracting surface. Outgoing light refers to light moving away from a reflecting/refracting surface towards an image point.}. Otherwise \(s\) is negative.
		\item Sign rule for the image distance: When the image is on the same side of the reflecting or refracting surface as the outgoing light, image distance \(s'\) is positive; Otherwise \(s'\) is negative.
		\item Sign rule for the radius of curvature of a spherical surface: When the center of curvature \(C\) is on the same side as the outgoing light, the radius of curvature is positive; Otherwise \(C\) is negative.
	\end{enumerate}
\end{rema}
\begin{rema}
	For a plane mirror, \(s=-s'\).
\end{rema}
\begin{defi}[Optic axis]
	The line that goes from the center of curvature to the vertex of the mirror is called the optic axis.\footnote{Rays that are nearly parallel to the optic axis are called paraxial rays.}
\end{defi}
\begin{defi}[Object-image relationship, spherical mirror]
	The object and image distances from the vertex, \(s\) and \(s'\) respectively, can be calculated from the radius of curvature \(R\) by the equation
	\begin{equation*}
		\frac{1}{s}+\frac{1}{s'}=\frac{2}{R}.
	\end{equation*}
\end{defi}
\begin{defi}[Focal point]
	The point \(F\) at which the incident parallel rays converge is called the focal point. For mirrors, the distance from the vertex to the focal point is called the focal length, denoted \(f\). For thin lenses, the distance from the center of the lens to the focal point is the focal length.\footnote{For thin lenses, the focal distance is the same for both focal points.}
\end{defi}
\begin{defi}[Lateral magnification]
	The lateral magnification \(m\) can be calculated from the image height \(y'\) and the object height \(y\) by
	\begin{equation*}
	m=\frac{y'}{y}.
	\end{equation*}
	If \(m\) is positive, the image is erect in comparison to the object. If \(m\) is negative, the image is inverted relative to the object.
\end{defi}
\begin{defi}[Lateral magnification, spherical mirror]
	\,
	\begin{equation*}
		m=\frac{y'}{y}=-\frac{s'}{s}
	\end{equation*}
\end{defi}
\begin{defi}[Converging and diverging lens]
	A converging lens is one which is thicker at its center than at its edges, a diverging lense is thicker at its edges than it is at its center.
\end{defi}
\begin{defi}[Object-image relationship, spherical refracting surface]
	\,
	\begin{equation*}
		\frac{n_a}{s}+\frac{n_b}{s'}=\frac{n_b-n_a}{R}
	\end{equation*}
\end{defi}
\begin{defi}[Object-image relationship, thin lens]
	This equation applies to both converging and diverging lenses.
	\begin{equation*}
		\frac{1}{s}+\frac{1}{s'}=\frac{1}{f}
	\end{equation*}
\end{defi}
\begin{rema}
	For a converging lens, paraxial rays converge to the focal point. For a diverging lens, paraxial rays diverge, but appear to converge to a focal point on the other side of the lens, and therefore the focal distance is negative. Rays that travel through the center of the lens are not deviated, and their intersection with paraxial rays that have been deviated through the second focal point will be the same as the intersection of rays incident rays that pass through the first focal point.\footnote{Any ray from the object that strikes the lens will pass through the image point.} This is useful for calculating the image point.
\end{rema}
\begin{defi}[Lensmaker's equation]
	Focal length can be calculated from the index of refraction of the lens material \(n\), and the radius of curvature of the first and second surface (relative to the incident light), \(R_1\) and \(R_2\) respectively.
	\begin{equation*}
		\frac{1}{f}=(n-1)\big(\frac{1}{R_1}-\frac{1}{R_2}\big)
	\end{equation*}
\end{defi}
\begin{rema}
	Longer focal length lenses in cameras are more zoomed-in. This is because the angles of view for photons are tighter with larger focals lengths, given the same lens width. Therefore a smaller fov is packed onto the same camera sensor.
\end{rema}
\begin{defi}[\(f\)-number of a lens]
	The \(f\)-number can be calculated from the focal length \(f\) and aperture diameter \(D\) by the equation
	\begin{equation*}
		f\text{-number}=\frac{f}{D}.
	\end{equation*}
		The intensity of light reaching the sensor or film is proportional to \(\frac{D^2}{f^2}\).
\end{defi}
\begin{rema}
	The exposure (total amount of light reaching the sensor or film) is proportional to both the aperture area and the time of exposure.
\end{rema}
\begin{defi}[Farsighted and nearsighted]
	Nearsighted eyes are not able to focus objects at infinity, because they form an image in front of the cornea. In farsighted eyes, the image of an infinitely distant object is formed behind the cornea.
\end{defi}
\begin{defi}[Angular magnification]
	The angular magnfication \(M\) is the ratio between the angle subtended at the eye by an object at the near point\footnote{The near point is the closest distance an object can be from the eye at which the eye can form an image of that object on the retina. For a healthy adult it is 25cm.} \(\theta\), and the angle subtended at the magnifier \(\theta'\), and is given by the equation
	\begin{equation*}
		M=\frac{\theta'}{\theta}.
	\end{equation*}
	Assuming for small \(\theta\) that \(\text{tan}\,\theta\) is equal to \(\theta\), we can express \(M\) as
	\begin{equation*}
		M=\frac{y/f}{y/25\text{cm}}=\frac{25\text{cm}}{f}.
	\end{equation*}
\end{defi}
\begin{rema}
	Telescopes and microscopes use a converging lens to form a real, enlarged image at the first focal point of a converging eyepiece lens. The difference between a micrscope and a telescope, is that in a microscope the objective lens forms its real image from an object just beyond the first focal point, and a telescope objective lens forms a real image from an object at infinity.
\end{rema}
\begin{defi}[Angular magnification, microscope]
	The angular magnification \(M\) of a compound microscope is the product of two factors. The first factor is the lateral magnification \(m_1\) of the objective, which determines the linear size of the real image. Because the object distance \(s_1\) is so close to the focal length \(f_1\), and because \(f_1\) is very small in comparison to the image distance \(s_1'\), the objective lens produces a very large negative\footnote{(inverted)} magnification which we can approximate as
	\begin{equation*}
		m_1=-\frac{s_{1}^{'}}{f_1}
	\end{equation*}
	The second factor is the angular magnification of the eyepiece \(M_2\), which because \(\theta\) is small, we can approximate as the ratio between the human near point, and the focal length of the eyepiece. Thus
	\begin{equation*}
		M=m_1M_2=\frac{(25\text{cm})s_{1}^{'}}{f_1f_2}.
	\end{equation*}
	Customarily, the negative sign is ignored. The final image is inverted with respect to the object.
\end{defi}
\begin{defi}[Angular magnification, refracting telescope]
	Because the image is formed at the second focal point of the objective lens \(F_1'\), and because this focal point is located at the first focal point of the eyepiece \(F_2\), the length of the telescope is equal to \(f_1+f_2\). The angular magnification of the telescope \(M\) is equal to the angle subtended by the incident light ray, and the angle subtended at the eye by the image, i.e.
	\begin{equation*}
		M=\frac{\theta'}{\theta}=-\frac{y'/f_2}{y'}{f_1}=\frac{-f_1}{f_2}
	\end{equation*}
\end{defi}
\section{Interference}
\begin{defi}[Physical optics]
	Optical effects that depend on the wave nature of light are grouped under the subject of physical optics.
\end{defi}
\begin{defi}[Interference]
	The term interference refers to any situation in which two or more waves overlap in space. When this occurs, the total wave at any point at any instant of time is governed by the principle of superposition.\footnote{Recall the principle of superposition just means adding the instantaneous displacements of each individual wave to find the displacement of the combined wave.}
\end{defi}
\begin{defi}[Monochromatic light]
	Most light in nature is not monochromatic, i.e. it is a mixture of light of different frequencies. Laser light is nearly monochromatic. In this chapter and the next, assume we are working with monochromatic light unless otherwise stated.
\end{defi}
\begin{defi}[Coherent light]
	Two monochromatic sources of the same frequency and constant phase relationship are said to be coherent.\footnote{This does not mean the light waves are in phase with each other.}
\end{defi}
\begin{rema}
	If the wave emitted by two coherent sources are transverse, we will also assume that the waves have the same polarization.
\end{rema}
\begin{defi}[Constructive and destructive interference]
	Let \(r_1\) and \(r_2\) be the distance from the sources \(s_1\) and \(s_2\) to point \(p\) respectively, and let \(m\in\{\mathbb{N, 0}\}\). The sources \(s_1\) and \(s_2\) are coherent and are in phase with each other. Interference is constructive if
	\begin{equation*}
		r_2-r_1=m\lambda.
	\end{equation*}
	Interference is destructive if
	\begin{equation*}
		r_2-r_1=(m+\frac{1}{2})\lambda.
	\end{equation*}
	Antinodal curces are where waves interfere constructively. Nodal curves are where waves interfere destructively.
\end{defi}
\begin{defi}[Constructive and destructive interference, two slits]
	The bright regions on the screen occur at angles \(\theta\), where \(\theta\) is the angle between the long path and the normal to the screen, \(d\) is the distance between slits and \(m\in\{\mathbb{N},0\}\), as given by the equation\footnote{The following equations assume both sources have the same maximum electric field magnitude.}
	\begin{equation*}
		d\text{sin}\,\theta=m\lambda.
	\end{equation*}
	Destructive occurance occurs at angles \(\theta\) for which the following equation holds:
	\begin{equation*}
		d\text{sin}\,\theta=(m+\frac{1}{2})\lambda.
	\end{equation*}
	Using the constructive interference \(\theta\) relationship, we can find the position of the \(m^{\text{th}}\) bright band for small angles, with \(R\) the distance to the screen by the equation
	\begin{equation*}
		y_m=R\frac{m\lambda}{d}.
	\end{equation*}
	The distance between adjacent bright bands is inversely proportional to the distance \(d\) between slits.
\end{defi}
\begin{defi}[Electric-field amplitude, two-source interference]
	\,
	\begin{equation*}
		E_p=2E\lvert\text{cos}\,\frac{\phi}{2}\rvert.
	\end{equation*}
\end{defi}
\begin{defi}[Intensity, two-source interference]
	Intensity \(I\) is equal to the average magnitude of the Poynting vector, \(S_av\). Let \(I_0\) be the maximum intensity. The intensity at the point in question is
	\begin{equation*}
		I=I_0\text{cos}^2\,\frac{\phi}{2}
	\end{equation*}
\end{defi}
\begin{defi}[Phase difference, two-source interference]
	\,
	\begin{equation*}
		\phi=\frac{2\pi}{\lambda}(r_2-r_1).
	\end{equation*}
\end{defi}
\begin{rema}
	Interference in thin films occurs because partial refraction of incident light into the film, and subsequent refraction of that light out of the film. The refracted light emerges parallel but out of phase with the reflected light. This means that certain wavelengths of light constructively or destructively interfere with each other, depending on the speed of the wavelength of the light in the refracting material, as well as the thickness of this material. When a light ray is reflected of of an interface with a higher refractive index, the phase shifts by \(\pi\).
\end{rema}
\begin{defi}[Constructive/destructive relfection, no phase shift]
	Constructive reflection occurs at thicknesses \(t\) such that
	\begin{equation*}
		2t=m\lambda.
	\end{equation*}
	Destructive reflection occurs at thicknesses \(t\) such that
	\begin{equation*}
		2t=(m+1/2)\lambda.
	\end{equation*}
\end{defi}
\begin{defi}[Constructive/destructive relfection, half-cycle phase shift]
	Constructive reflection occurs at thicknesses \(t\) such that
	\begin{equation*}
		2t=(m+1/2)\lambda.
	\end{equation*}
	Destructive reflection occurs at thicknesses \(t\) such that
	\begin{equation*}
		2t=m\lambda.
	\end{equation*}
\end{defi}
\begin{rema}
	For thick films, there is no definite phase shift so outgoing rays are incoherent, and there is no fixed interference pattern.
\end{rema}
\begin{rema}
	The \(x\) values for which we see destructive interference for films with lower index of refractions are.
	\begin{equation*}
		x=\frac{ml\lambda}{2h}.
	\end{equation*}
	Light that is in a lower index of refraction material that reflects off of material with a higher index of refraction experiences a \(\pi\) phase shift. Otherwise, there is no phase shift on reflection.
\end{rema}
\begin{defi}[Nonreflective and reflective coatings]
	Both reflective and nonreflective coatings are deposited in a \(\lambda/4\) thickness coating. Nonreflective materials have a lower index of refraction from the material they coat, and reflective coatings have a higher index of refraction from the material they coat.
\end{defi}
\section{Diffraction}
\begin{defi}[Diffraction]
	The class of effects that occur when light strikes a barrier that has an aperture or an edge are called diffraction.
\end{defi}
\begin{defi}[Huygen's principle]
	We can consider every point of a wave front as a source of secondary wavelets. the position of the wave front at any later time is the envelope of the secondary wavelets at that time.
\end{defi}
\begin{defi}[Fresnel and Fraunhofer diffraction]
	Fresnel, or near field diffraction, occurs when both the point source and the screen are relatively close to the obstacle forming the diffraction patterns. Fraunhofer diffraction occurs when the source, obstacle, and screen are far enough apart that we can consider all lines from the source to the obstacle to be parallel.\footnote{The discussion in this chapter is restricted to Fraunhofer diffraction.}
\end{defi}
\begin{rema}
	There is no fundamental distinction between interference and diffraction.
\end{rema}
\begin{defi}[Dark finges, single-slit diffraction]
	If \(\theta\) is the angle from the center of the slit to the \(m^{th}\)dark fringe on the screen at point \(P\), and \(a\) is the slit width, then\footnote{Because \(\theta\) is very small, to calculate path difference we use right triangle with hypotenuse \(a/2\) normal to the ray originating from the center.}
	\begin{equation*}
		\text{sin}\,\theta=\frac{m\lambda}{a}\qquad(m\in\mathbb{Z}\setminus 0)
	\end{equation*}
	Because the screen is far away, light from two huygens strips \(\frac{a}{2m}\) always arrive out of phase at point \(P\), because they too have a path difference of \((m-1)+a/2\). Therefore if \(a\text{sin}\,\theta\) is a multiple of \(\lambda/2\), then all rays originating along \(a/2\) have a sister ray which destructively interferes with it. Thus at point \(P\) all light originating from the slit cancels at point \(P\). Because the light wavelength is so small in comparision to the slit width, we can approximate \(\text{sin}\,\theta=\theta\) and use the equation
	\begin{equation*}
		\theta=\frac{m\lambda}{a}.
	\end{equation*}
	and thus the vertical distance of the \(m^th\) dark band from the center of the pattern \(y_m\) for a screen at distance \(x\) is
	\begin{equation*}
		y_m=x\frac{m\lambda}{a}\qquad(\text{for\;}y_m<<x)
	\end{equation*}
\end{defi}
\begin{defi}[Intensity in single-slit diffraction]
	The intensity \(I\) at point \(p\) can be found from the angle from the center of the slit to \(p\), \(\theta\) as.
	\begin{equation*}
		I=I_0\bigg\{\frac{\text{sin}\,[\pi a(\text{sin}\,\theta)/\lambda]}
		{\pi a(\text{sin}\,\theta)/\lambda}\bigg\}^2
	\end{equation*}
	It may be easier to calculate beta first by
	\begin{equation*}
		\beta=\frac{2\pi}{\lambda}a\text{sin}\,\theta
	\end{equation*}
	And then calculate \(I\) from
	\begin{equation*}
		I=I_0\bigg[\frac{\text{sin}(\beta/2)}{\beta/2}\bigg]^2
	\end{equation*}
\end{defi}
\begin{defi}[Intensity of \(m\)\textsuperscript{th} side maximum]
	\,
	\begin{equation*}
		I_m\approx\frac{I_0}{(m+1/2)^2\pi^2}.
	\end{equation*}
\end{defi}
\section{Relativity}
\begin{defi}[Proper time]
	Proper time refers to the time interval between two events that occur at the same point, i.e. the time for an event in the rest frame. A person on a moving train is in a rest frame relative to the train.
\end{defi}
\begin{defi}[Lorentz factor]
	\begin{equation*}
		\gamma=\frac{1}{\sqrt{1-u^2/c^2}}
	\end{equation*}
\end{defi}
\begin{defi}[Constants]
	\,
	\begin{enumerate}
		\item \(h=6.626\times 10^{-34}\text{Js}\)
	\end{enumerate}
\end{defi}
\begin{defi}[Time dilation]
	The relationship between \(\Delta t\), the time in a frame of reference with speed relative to the rest frame \(u\), and the time in the rest frame \(\Delta t_0\) is:
	\begin{equation*}
		\Delta t=\frac{\Delta t_0}{\sqrt{1-u^2/c^2}}
	\end{equation*}
	In other words,
	\begin{equation*}
		\Delta t=\gamma\Delta t_0.
	\end{equation*}
\end{defi}
\begin{defi}[Length contraction]
	The relationship between \(l_0\), the length of the object in the reference frame, and \(l\) is
	\begin{equation*}
		l=l_0\sqrt{1-\frac{u^2}{c^2}}=\frac{l_0}{\gamma}
	\end{equation*}
\end{defi}
\begin{defi}
	The equation for the proper distance covered by an object in a moving reference frame, relative to its displacement in the observer frame \(x\) and the distance covered by the rest frame \(ut\) is:
	\begin{equation*}
		x'=\gamma(x-ut).
	\end{equation*}
\end{defi}
	The equation for the proper time \(t'\) elapsed in the rest frame is
\begin{defi}
	\begin{equation*}
		t'=\gamma(t-ux/c^2).
	\end{equation*}
\end{defi}
\begin{defi}[Lorentz velocity transformation]
	\begin{equation*}
		v_x'=\frac{v_x-u}{1-uv_x/c^2}
	\end{equation*}
\end{defi}
\begin{defi}[Relativistic doppler effect]
	If the source with frequency \(f_0\) is approaching the observer with speed \(u\), the frequency measured by the observer is
	\begin{equation*}
		f=\sqrt{\frac{c+u}{c-u}}f_0.
	\end{equation*}
\end{defi}
\begin{defi}[Relativistic momentum]
	\begin{equation*}
		\vec{p}=\frac{m\vec{v}}{\sqrt{1-v^2/c^2}}=\gamma m\vec{v}
	\end{equation*}
\end{defi}
\begin{defi}[Relativistic kinetic energy]
	\begin{equation*}
		K=(\gamma-1)mc^2
	\end{equation*}
\end{defi}
\begin{defi}[Total energy of a particle]
	\begin{equation*}
		E=K+mc^2=\gamma mc^2
	\end{equation*}
\end{defi}
\begin{defi}[Total enegry, rest energy, and momentum]
	\begin{equation*}
		(mc^2)^2+(pc)^2
	\end{equation*}
\end{defi}
\section{Photons and other shit}
\begin{defi}[Cathode and anode]
	The cathode emits electrons, the anode absorbs electrons.
\end{defi}
\begin{defi}[Energy of a photon]
	\begin{equation*}
		E=hf=\frac{hc}{\lambda}
	\end{equation*}
\end{defi}
\begin{defi}[Photoelectric effect]
	In terms of the stopping potential of emitted electrons \(V_0\), and the work function\footnote{Minimum energy needed to remove an electron from the surface.} \(\phi\), the frequency of incident light is
	\begin{equation*}
		eV_0=hf-\phi.
	\end{equation*}
\end{defi}
\begin{defi}[Photon momentum]
	\begin{equation*}
		p=\frac{E}{c}=\frac{hf}{c}=\frac{h}{\lambda}
	\end{equation*}
\end{defi}
\begin{defi}[X-ray generator]
	X-ray generators work by inducing electron emission in a cathode through heating\footnote{This is called thermionic emission}, and accelerating these electrons through a large potential difference (typically thousands of volts) with the anode. The electrons slam in to the anode so violently that x-rays are emitted as the electron accelerate to a stop. The frequency of photons emitted by the anode is given by the equation
	\begin{equation*}
		eV_{AC}=hf_{max}=\frac{hc}{\lambda{min}}
	\end{equation*}
\end{defi}
\begin{defi}[Compton scattering]
	The relationship between the wavelength of incident light \(\lambda\) and that of the scattered light \(\lambda'\) is given by the following equation.
	\begin{equation*}
		\lambda'-\lambda=\frac{h}{mc}(1-\text{cos}\phi)
	\end{equation*}
\end{defi}
\begin{defi}[Heisenberg uncertainty principle momentum and position]
	\begin{equation*}
		\Delta x\Delta p_x\geq \frac{h}{4\pi}
	\end{equation*}
\end{defi}
\begin{defi}[Heisinberg uncertainty principle energy and time]
	\begin{equation*}
		\Delta t\Delta E\geq\frac{h}{4\pi}
	\end{equation*}
\end{defi}
\section{Particles behaving as waves}
\begin{defi}[De broglie wavelength of a particle]
	\begin{equation*}
		\lambda=\frac{h}{p}=\frac{h}{mv}
	\end{equation*}
\end{defi}
\begin{defi}[Energy of a particle]
	\begin{equation*}
		E=hf
	\end{equation*}
\end{defi}
\begin{defi}[De Broglie wavelength of an electron]
	\begin{equation*}
		\lambda=\frac{h}{p}=\frac{h}{\sqrt{2meV_{ba}}}
	\end{equation*}
\end{defi}
\begin{defi}[Energy levels]
	\begin{equation*}
		hf=E_i=E_f
	\end{equation*}
\end{defi}
\begin{defi}[Quantization of angular momentum, H]
	\begin{equation*}
		L_n=mv_nr_n=n\frac{h}{2\pi}
	\end{equation*}
\end{defi}
\begin{defi}[Radius of \(n^{\text{th}}\) orbit]
	\begin{equation*}
		r_n=\epsilon_0\frac{n^2 h^2}{\pi me^2}
	\end{equation*}
\end{defi}
\begin{defi}[Orbital speed in \(n^{\text{th}}\) orbit]
	\begin{equation*}
		v_n=\frac{1}{\epsilon_0}\frac{e^2}{2nh}
	\end{equation*}
\end{defi}
\begin{defi}[Radius of \(n^{\text{th}}\) orbit]
	\begin{equation*}
		r_n=n^2a_0
	\end{equation*}
	Where \(a_0\) is
	\begin{equation*}
		a_0=5.29\times 10^{-11}\text{m}
	\end{equation*}
\end{defi}
\begin{defi}[Total energy for \(n^{\text{th}}\) orbit]
	\begin{equation*}
		E_n=-\frac{hcR}{n^2}\quad\text{Where }R=\frac{me^4}{8\epsilon_0^2h^3c}
	\end{equation*}
\end{defi}
\begin{defi}[Stefan-Boltzmann law for blackbody]
	\begin{equation*}
		I=\sigma T^4
	\end{equation*}
	Where \(T\) is temperature in kelvin, and sigma is
	\begin{equation*}
		\sigma=5.67\times 10^{-8}W/m^2\cdot K^4.
	\end{equation*}
\end{defi}
\begin{defi}[Wein displacement law for blackbody]
	\begin{equation*}
		\lambda_m T=2.90\times 10^{-3}m\cdot K
	\end{equation*}
	Where \(\lambda_m\) is the peak wavelength emitted.
\end{defi}
\begin{defi}[Planck radiation law]
	\begin{equation*}
		I(\lambda)=\frac{2\pi hc^2}{\lambda^5(e^{hc/\lambda kT}-1)}
	\end{equation*}
\end{defi}
\begin{defi}[Heisenberg uncertainty principle]
	\begin{IEEEeqnarray*}{l}
		\Delta x\Delta p_x\geq\frac{h}{4\pi}\\
		\Delta t\Delta E\geq\frac{h}{4\pi}\\
	\end{IEEEeqnarray*}
\end{defi}
\end{document}
