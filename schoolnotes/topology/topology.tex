\documentclass[nobib,notoc]{tufte-handout}
\usepackage[utf8]{inputenc}
\usepackage[english]{babel}
\usepackage{amsmath}
\usepackage{amsthm}
\usepackage{amsfonts}
\usepackage{hyperref}
\usepackage{mathrsfs}
\usepackage{IEEEtrantools}
\usepackage{enumitem}

\renewcommand{\IEEEQED}{\IEEEQEDopen}

\begin{document}

\theoremstyle{definition}\newtheorem{defi}{Definition}[section]
\theoremstyle{definition}\newtheorem{axiom}{Axiom}[section]
\theoremstyle{definition}\newtheorem{thm}{Theorem}[section]
\theoremstyle{definition}\newtheorem{cor}{Corollary}[section]
\theoremstyle{definition}\newtheorem{lem}{Lemma}[section]
\theoremstyle{remark}\newtheorem*{notat}{Notation}
\theoremstyle{remark}\newtheorem*{rema}{Remark}
\theoremstyle{definition}\newtheorem{problem}{Problem}
%\renewcommand{\theproblem}{\arabic{problem}}
\newenvironment{prob}[1]{\protect\setcounter{problem}{#1}\addtocounter{problem}{-1}\begin{problem}}{\end{problem}}

\title{Topology}
\author{Samuel Lindskog}
\maketitle

\setcounter{section}{1}
\setcounter{tocdepth}{1}

\section{Open and closed sets}
\begin{defi}[Metric]
	A \emph{metric} on a set \(X\) is a real-valued function \(d\) on \(X\times X\) that has the following properties:
	\begin{enumerate}[label=(\alph*)]
		\item For all \(x,y\in X\), \(d(x,y)\geq 0\).
		\item \(d(x,y)=0\) iff \(x=y\).
		\item For all \(x,y\in X\), \(d(x,y)=d(y,x)\).
		\item For all \(x,y,z\in X\), \(d(x,z)\leq d(x,y)+d(y,z)\).
	\end{enumerate}
\end{defi}
\begin{defi}[Metric space]
	A metric space \((X,d)\) is a set \(X\) equipped with a metric \(d\) on \(X\).
\end{defi}
\begin{defi}[Subspace]
	If \((X,d)\) is a metric space and \(Y\) is a subset of \(X\), then the restriction \(d'\) of \(d\) to \(Y\times Y\) is a metric on \(Y\), and \((Y,d')\) is called a subspace of \((X,d)\).
\end{defi}
\begin{rema}
	Any set \(X\) can be made into a discreet metric space by associating with \(X\) the metric \(d\) defined by
	\begin{equation*}
		d(x,y)=\begin{cases}
			1,&x\neq y\\
			0,&x=y
		\end{cases}
	\end{equation*}
\end{rema}
\begin{defi}[Open ball]
	The open ball \(B(x,r)\) with center \(x\in X\) and radius \(r>0\) is defined by
	\begin{equation*}
		B(x,r)=\{y\in X|d(x,y)<r\}.
	\end{equation*}
\end{defi}
\begin{defi}[Interior point]
	Let \(Y\) be a subset of \(X\). A point \(x\in X\) is an interior point of \(Y\) if there exists \(r>0\) such that \(B(x,r)\subseteq Y\). The set of interior points of \(y\) is the interior of \(Y\), and it is denoted by int\((Y)\).\footnote{int\((Y)\subseteq Y.\)}
\end{defi}
\begin{defi}[Open subset]
	A subset \(Y\) of \(X\) is open if int\((Y)=Y\).
\end{defi}
\begin{thm}
	Any open ball \(B(x,r)\) in a metric space \(X\) is an open subset of \(X\)
	\begin{IEEEproof}
		Suppose \(y\in B(x,r)\). Then \(d(x,y)<r\), and \(0<r-d(x,y)\). Suppose \(z\in B(y,r-d(x,y))\). If follows from the definition of a metric that \(d(x,z)\leq d(x,y) + d(y,z)\), so \(d(x,z)\leq d(x,y) + (r-d(x,y))=r\), so \(z\in B(x,r)\).
	\end{IEEEproof}
\end{thm}
\begin{thm}
	The union of a family of open subsets of a metric space \(X\) is an open subset of \(X\).
	\begin{IEEEproof}
		Suppose \(\{U_\alpha\}\,\alpha\in A\) a family of open subsets of \(X\). If \(x\in\bigcup_{\alpha\in A}U_\alpha\), then \(\exists\alpha\,(x\in U_\alpha)\), so there exits an open ball \(B(x,r)\) such that \(B(x,r)\subseteq U_\alpha\). Because \(x\in U_\alpha\Rightarrow x\in\bigcup_{\alpha\in A}{U_\alpha}\), then \(B(x,r)\subseteq\bigcup_{\alpha\in A}U_\alpha\).
	\end{IEEEproof}
\end{thm}
\begin{thm}
	A subset \(U\) of a metric space \(X\) is open iff \(U\) is a union of open balls in \(X\).
	\begin{IEEEproof}
		Theorem 1.1 and 1.3 prove the left implication. If \(U\) is an open subset of \(X\), then for all \(x\in U\), there exists \(r(x)>0\) such that \(B(x,r(x))\in U\), so \(\bigcup_{x\in U}B(x,r(x))=U\).
	\end{IEEEproof}
\end{thm}
\begin{thm}
	The intersection of any finite number of open subsets of a metric space is open.
	\begin{IEEEproof}
		Suppose \(x\in\bigcap_{n=1}^{m}U_n\), a finite union of open subsets of a metric space. Then for all \(n\), there exists \(r(n)>0\) such that \(B(x,r(n))\in U_n\). Let \(r=\text{min}(r(1)\ldots r(m))\). Then for all \(r(n)\) we see \(B(x,r)\subseteq B(x,r(n))\) and thus \(B(x,r)\subseteq\bigcap_{n=1}^{m}U_n\).
	\end{IEEEproof}
\end{thm}
\begin{thm}
	Let \(Y\) be a subspace of a metric space \(X\). Then a subset \(U\) of \(Y\) is open in \(Y\) iff \(U=V\cap Y\) for some open subset \(V\) of \(X\).
	\begin{IEEEproof}
		Suppose \(x\in V\cap Y\). Then there exists an open ball in \(X\) with radius \(r(x)\) such that \(B(x,r(x))\subseteq V\), and \(x\in Y\). Because \(Y\subseteq X\) we see that \(Y\cap B(x,r(x))=\{y\in X\cap Y|d(x,y)<r(x)\}=\{y\in Y|d(x,y)<r(x)\}\), by definition an open ball in \(Y\). Trivially \(V\cap Y\subseteq\bigcap_{x\in V\cap Y}Y\cap B(x,r(x))\) and by definition the reverse is true.\smallbreak
		To prove the converse, suppose \(x\in U\). Then there exits an open ball in \(Y\) with radius \(r(x)\) such that \(B(x, r(x))\in U\). It follows from conclusions reached above that if \(B'(x, r(x))\) is open in \(X\), then \(B'(x,r(x))\cap Y=B(x,r(x))\). Let \(V=\bigcup_{x\in U}B'(x,r(x))\). Then \(V\cap Y\subseteq U\), and \(x\in U\Rightarrow x\in V\).
\end{IEEEproof}
\end{thm}
\begin{defi}[Adherent point]
	Let \(Y\) be a subset of a metric space \(X\). A point \(x\in X\) is adherent to \(Y\) if for all \(r>0\)
	\begin{equation*}
		B(x,r)\cap Y\neq\emptyset
	\end{equation*}
\end{defi}
\begin{defi}[Closure]
	The closure of \(Y\) denoted by \(\overline{Y}\), consists of all points in \(X\) that are adherent to \(Y\).\footnote{\(Y\subseteq\overline{Y}\).}
\end{defi}
\begin{defi}[Closed subset]
	The subset \(Y\) is closed if \(Y=\overline{Y}\).\footnote{The empty set \(\emptyset\) and \(X\) are closed subsets of \(X\). Interestingly, \(X\) is also open in \(X\).}
\end{defi}
\end{document}
