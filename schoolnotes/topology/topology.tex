\documentclass[nobib,notoc]{tufte-handout}
\usepackage[utf8]{inputenc}
\usepackage[english]{babel}
\usepackage{amsmath}
\usepackage{amsthm}
\usepackage{amsfonts}
\usepackage{hyperref}
\usepackage{mathrsfs}
\usepackage{IEEEtrantools}
\usepackage{enumitem}

\renewcommand{\IEEEQED}{\IEEEQEDopen}

\begin{document}

\theoremstyle{definition}\newtheorem{defi}{Definition}[section]
\theoremstyle{definition}\newtheorem{axiom}{Axiom}[section]
\theoremstyle{definition}\newtheorem{thm}{Theorem}[section]
\theoremstyle{definition}\newtheorem{cor}{Corollary}[section]
\theoremstyle{definition}\newtheorem{lem}{Lemma}[section]
\theoremstyle{remark}\newtheorem*{notat}{Notation}
\theoremstyle{remark}\newtheorem*{rema}{Remark}
\theoremstyle{definition}\newtheorem{problem}{Problem}
%\renewcommand{\theproblem}{\arabic{problem}}
\newenvironment{prob}[1]{\protect\setcounter{problem}{#1}\addtocounter{problem}{-1}\begin{problem}}{\end{problem}}

\title{Topology}
\author{Samuel Lindskog}
\maketitle

\setcounter{section}{1}

\section{Open and closed sets}
\begin{defi}[Metric]
	A \emph{metric} on a set \(X\) is a real-valued function \(d\) on \(X\times X\) that has the following properties:
	\begin{enumerate}[label=(\alph*)]
		\item For all \(x,y\in X\), \(d(x,y)\geq 0\).
		\item \(d(x,y)=0\) iff \(x=y\).
		\item For all \(x,y\in X\), \(d(x,y)=d(y,x)\).
		\item For all \(x,y,z\in X\), \(d(x,z)\leq d(x,y)+d(y,z)\).
	\end{enumerate}
\end{defi}
\begin{defi}[Metric space]
	A metric space \((X,d)\) is a set \(X\) equipped with a metric \(d\) on \(X\).
\end{defi}
\begin{defi}[Subspace]
	If \((X,d)\) is a metric space and \(Y\) is a subset of \(X\), then the restriction \(d'\) of \(d\) to \(Y\times Y\) is a metric on \(Y\), and \((Y,d')\) is called a subspace of \((X,d)\).
\end{defi}
\begin{rema}
	Any set \(X\) can be made into a discreet metric space by associating with \(X\) the metric \(d\) defined by
	\begin{equation*}
		d(x,y)=\begin{cases}
			1,&x\neq y\\
			0,&x=y
		\end{cases}
	\end{equation*}
\end{rema}
\begin{defi}[Open ball]
	The open ball \(B(x,r)\) with center \(x\in X\) and radius \(r>0\) is defined by
	\begin{equation*}
		B(x,r)=\{y\in X|d(x,y)<r\}.
	\end{equation*}
\end{defi}
\begin{defi}[Interior point]
	Let \(Y\) be a subset of \(X\). A point \(x\in X\) is an interior point of \(Y\) if there exists \(r>0\) such that \(B(x,r)\subseteq Y\). The set of interior points of \(y\) is the interior of \(Y\), and it is denoted by int\((Y)\).\footnote{int\((Y)\subseteq Y.\)}
\end{defi}
\begin{defi}[Open subset]
	A subset \(Y\) of \(X\) is open if int\((Y)=Y\).
\end{defi}
\begin{thm}
	Any open ball \(B(x,r)\) in a metric space \(X\) is an open subset of \(X\)
	\begin{IEEEproof}
		Suppose \(y\in B(x,r)\). Then \(d(x,y)<r\), and \(0<r-d(x,y)\). Suppose \(z\in B(y,r-d(x,y))\). If follows from the definition of a metric that \(d(x,z)\leq d(x,y) + d(y,z)\), so \(d(x,z)\leq d(x,y) + (r-d(x,y))=r\), so \(z\in B(x,r)\).
	\end{IEEEproof}
\end{thm}
\begin{thm}
	The union of a family of open subsets of a metric space \(X\) is an open subset of \(X\).
	\begin{IEEEproof}
		Suppose \(\{U_\alpha\}\,\alpha\in A\) a family of open subsets of \(X\). If \(x\in\bigcup_{\alpha\in A}U_\alpha\), then \(\exists\alpha\,(x\in U_\alpha)\), so there exits an open ball \(B(x,r)\) such that \(B(x,r)\subseteq U_\alpha\). Because \(x\in U_\alpha\Rightarrow x\in\bigcup_{\alpha\in A}{U_\alpha}\), then \(B(x,r)\subseteq\bigcup_{\alpha\in A}U_\alpha\).
	\end{IEEEproof}
\end{thm}
\begin{thm}
	A subset \(U\) of a metric space \(X\) is open iff \(U\) is a union of open balls in \(X\).
	\begin{IEEEproof}
		Theorem 1.1 and 1.3 prove the left implication. If \(U\) is an open subset of \(X\), then for all \(x\in U\), there exists \(r(x)>0\) such that \(B(x,r(x))\in U\), so \(\bigcup_{x\in U}B(x,r(x))=U\).
	\end{IEEEproof}
\end{thm}
\begin{thm}
	The intersection of any finite number of open subsets of a metric space is open.
	\begin{IEEEproof}
		Suppose \(x\in\bigcap_{n=1}^{m}U_n\), a finite union of open subsets of a metric space. Then for all \(n\), there exists \(r(n)>0\) such that \(B(x,r(n))\in U_n\). Let \(r=\text{min}(r(1)\ldots r(m))\). Then for all \(r(n)\) we see \(B(x,r)\subseteq B(x,r(n))\) and thus \(B(x,r)\subseteq\bigcap_{n=1}^{m}U_n\).
	\end{IEEEproof}
\end{thm}
\begin{thm}
	Let \(Y\) be a subspace of a metric space \(X\). Then a subset \(U\) of \(Y\) is open in \(Y\) iff \(U=V\cap Y\) for some open subset \(V\) of \(X\).
	\begin{IEEEproof}
		Suppose \(x\in V\cap Y\). Then there exists an open ball in \(X\) with radius \(r(x)\) such that \(B(x,r(x))\subseteq V\), and \(x\in Y\). Because \(Y\subseteq X\) we see that \(Y\cap B(x,r(x))=\{y\in X\cap Y|d(x,y)<r(x)\}=\{y\in Y|d(x,y)<r(x)\}\), by definition an open ball in \(Y\). Trivially \(V\cap Y\subseteq\bigcap_{x\in V\cap Y}Y\cap B(x,r(x))\) and by definition the reverse is true.\smallbreak
		To prove the converse, suppose \(x\in U\). Then there exits an open ball in \(Y\) with radius \(r(x)\) such that \(B(x, r(x))\in U\). It follows from conclusions reached above that if \(B'(x, r(x))\) is open in \(X\), then \(B'(x,r(x))\cap Y=B(x,r(x))\). Let \(V=\bigcup_{x\in U}B'(x,r(x))\). Then \(V\cap Y\subseteq U\), and \(x\in U\Rightarrow x\in V\).
\end{IEEEproof}
\end{thm}
\begin{defi}[Adherent point]
	Let \(Y\) be a subset of a metric space \(X\). A point \(x\in X\) is adherent to \(Y\) if for all \(r>0\)
	\begin{equation*}
		B(x,r)\cap Y\neq\emptyset
	\end{equation*}
\end{defi}
\begin{defi}[Closure]
	The closure of \(Y\) denoted by \(\overline{Y}\), consists of all points in \(X\) that are adherent to \(Y\).\footnote{\(Y\subseteq\overline{Y}\).}
\end{defi}
\begin{defi}[Closed subset]
	The subset \(Y\) is closed if \(Y=\overline{Y}\).\footnote{The empty set \(\emptyset\) and \(X\) are closed subsets of \(X\). Interestingly, \(X\) is also open in \(X\).}
\end{defi}
\begin{thm}
If \(Y\) is a subset of a metric space \(X\), then the closure of \(Y\) is closed, i.e.
	\begin{equation*}
		\overline{\overline{Y}}=\overline{Y}
	\end{equation*}
\begin{IEEEproof}
	\(\overline{Y}\) contains all \(x\in X\) such that for all \(r>0\) in \(B(x,r)\cap Y\neq\emptyset\). Let \(y\in X\) with \(B(y,r')\cap\overline{Y}\neq\emptyset\) for \(r'>0\). Suppose to the contrary that there does not exist \(x\in X\) such that \(x=y\). Then there exists \(a=\text{min}\big(d(x,y)\big)>0\) such that \(\forall x\,(x\notin B(y,a))\), therefore \(B(y,a)\cap\overline{Y}=\emptyset\), a contradiction.
\end{IEEEproof}
\end{thm}
\begin{thm}
	A subset \(Y\) of a metric space \(X\) is closed iff the complement of \(Y\) is open.
	\begin{IEEEproof}
		If \(Y\) is closed, then \(Y\) contains all \(x\in X\) such that for all \(r>0\), \(B(x,r)\cap Y\neq\emptyset\). Therefore iff \(y\in Y^c\) the negation is true, i.e. there exists \(r'>0\) such that \(B(y,r')\cap Y=\emptyset\), and because \(Y^c\cup Y=X\) we have \(B(y,r')\subset Y^c\) and \(Y^c\) is open.
	\end{IEEEproof}
\end{thm}
\begin{thm}
	The intersection of any family of closed sets is closed. The union of any finite family of closed sets is closed.
	\begin{IEEEproof}
		Let \(\{Y_\alpha\}\) be a family of closed sets in \(X\), and \(\alpha\in A\), the number of elements in \(\{Y_\alpha\}\). Following the fact that a union of open subsets is open, and the intersection of finite open subsets is open, as well as the previous theorem, we see
		\begin{IEEEeqnarray*}{rCl}
			X\setminus\bigcup_{\alpha\in A}Y_{\alpha}&=&\bigcap_{\alpha\in A}X\setminus Y_\alpha\\
			X\setminus\bigcap_{\alpha\in A}Y_{\alpha}&=&\bigcup_{\alpha\in A}X\setminus Y_\alpha
		\end{IEEEeqnarray*}
	\end{IEEEproof}
\end{thm}
\begin{defi}[Convergent sequence]
	A sequence \(\{x_n\}_{n=1}^{\infty}\) in a metric space \(X\) converges to \(x\in X\) if
	\begin{equation*}
		\lim_{n\rightarrow\infty}d(x_n, x)=0
	\end{equation*}
	In this case, \(x\) is the limit of \(\{x_n\}\) and we write \(x_n\rightarrow x\), or
	\begin{equation*}
		\lim_{n\rightarrow\infty}x_n=x.
	\end{equation*}
\end{defi}
\begin{lem}
	The limit of a convergent sequence in a metric space is unique
	\begin{IEEEproof}
		Let \(lim_{n\rightarrow\infty}x_n=x,y\) and suppose to the contrary that \(x\neq y\). Then \(d(x,y)>0\) and for all \(\epsilon>0\) there exits \(\delta\) such that \(d(x_n,x)\) and \(d(x_n,y)\) are both less than \(\frac{\epsilon}{2}\). But then if \(\epsilon<d(x,y)\) then \(d(x_n,x)+d(x_n,y)<d(x,y)\), a contradiction.
	\end{IEEEproof}
\end{lem}
\begin{thm}
	Let \(Y\) be a subset of the metric space \(X\), then \(x\in X\) is adherent to \(Y\) iff there is a sequence in \(Y\) that converges to \(x\).
	\begin{IEEEproof}
		If \(x\) is adherent to \(Y\), then \(\forall r>0\), \(B(x,r)\cap Y\neq\emptyset\), i.e. for all \(r\) there exits \(y\in Y\) such that \(d(x,y_n)<r\). Using this fact we can construct a sequence that converges to \(x\). Let \(y_n\in Y\), and \(\{y_n\}\) be a sequence such that for all \(\epsilon>0\) there exists \(N\in\mathbb{N}\) such that \(n>N\) implies \(d(x,y)<\epsilon\).\bigbreak
		Let \(\{y_n\}\) be a sequence with \(y_n\in Y\), and let \(x\in X\). Let \(\{y_n\}\) be such that for all \(\epsilon>0\), \(n\in\mathbb{N}\) with \(n>N\) implies \(d(x,y_n)<\epsilon\). Then for all \(r>0\) there exists \(r=\epsilon\) such that \(y_n\in B(x,r)\), and thus \(B(x,r)\cap Y\neq\emptyset\) for all \(r>0\).
	\end{IEEEproof}
\end{thm}
\stepcounter{section}
\section{Completeness}
\begin{defi}[Cauchy sequence]
	A sequence \(\{x_n\}_{n=1}^{\infty}\) in a metric space \(X\) is a Cauchy sequence if
	\begin{equation*}
		\lim_{m,n\rightarrow\infty}d(x_n,x_m)=0.
	\end{equation*}
\end{defi}
\begin{lem}
	A convergent sequence is a cauchy sequence.
	\begin{IEEEproof}
		Suppose \(\{x_n\}\) a sequence that converges to \(x\). Then
		\begin{equation*}
			\forall\epsilon>0,\exists n,m>N\,\big(d(x_n,x),d(x_m,x)<\epsilon\big).
		\end{equation*}
		If we choose \(N\) such that \(d(x_n,x),d(x_m,x)<\frac{\epsilon}{2}\) then
		\begin{equation*}
			d(x_n,x)+d(x_m,x)<\epsilon\Rightarrow d(x_n,d_m)<\epsilon.
		\end{equation*}
	\end{IEEEproof}
\end{lem}
\begin{lem}
	If \(\{x_n\}\) is a Cauchy sequence and if there is a subsequence \(\{x_{n_k}\}_{k=1}^{\infty}\) of \(\{x_n\}\) that converges to \(x\), then \(\{x_n\}\) converges to \(x\).
	\begin{IEEEproof}
		Suppose \(\{x_n\}\) a convergent sequence and \(\{x_{n_k}\}_{k=1}^{\infty}\) a subsequence which converges to \(x\) then
		\begin{IEEEeqnarray*}{c}
			\forall\delta>0,\exists N\,(n_k>N\Rightarrow d(x_{n_k},x)<\delta)\\
			\forall\epsilon>0,\exists M\,(n>M\wedge n_k>M,N\Rightarrow d(x_n,x_{n_k})<\epsilon).
		\end{IEEEeqnarray*}
		Because \(d(x_n,x_{n_k})+d(x_{n_k},x)<\epsilon+\delta\) then \(d(x_n,x)<\epsilon+\delta\).
	\end{IEEEproof}
\end{lem}
\begin{defi}[Complete metric space]
	A metric space \(X\) is complete if every cauchy sequence in \(X\) converges.
\end{defi}
\begin{thm}
	A complete subspace \(Y\) of a metric space \(X\) is closed in \(X\)
	\begin{IEEEproof}
		If \(x\in\overline{Y}\), then \(\forall r>0,\exists B(x,r)\) such that \(B(x,r)\cap Y\neq\emptyset\), so \(\exists y\in Y\) such that \(d(x,y)<r\). It follows there exists a Cauchy sequence \(\{y_n\}\) in \(Y\) with limit \(x\) such that \(\forall r,\exists N\;(n>N\Rightarrow d(x,y_n)<r)\). And because every Cauchy sequence in \(Y\) converges, \(x\in Y\) and \(\overline{Y}=Y\).
	\end{IEEEproof}
\end{thm}
\begin{defi}[Uniform convergence]
	Let \(\{f_n\}_{n=1}^{\infty}\) be a sequence of functions from a set \(S\) to a metric space \(X\) and let \(f\) be a function from \(S\) to \(X\). The sequence \(\{f_n\}\) converges uniformly to \(f\) on \(S\) if for each \(\epsilon>0\) there exists an integer \(N\) such that \(d(f_n(s),f(s))<\epsilon\) for all integers \(n\geq N\) and for all \(s\in S\).
\end{defi}
\begin{defi}
	A sequence \(\{f_n\}\) of functions from \(S\) to \(X\) is a Cauchy sequence of functions if for each \(\epsilon>0\) there exists an integer \(N\) such that
	\begin{equation*}
		d(f_n(s),f_m(s))<\epsilon,\qquad\text{all }s\in S,\; n,m\geq N.
	\end{equation*}
\end{defi}
\begin{thm}
	Let \(S\) be a set, and let \(X\) be a complete metric space. If \(\{f_n\}_{n=1}^{\infty}\) is a Cauchy sequence of functions from \(S\) to \(X\), then there exists a function \(f\) from \(S\) to \(X\) such that \(\{f_n\}\) converges uniformly to \(f\)
	\begin{IEEEproof}
	\end{IEEEproof}
\end{thm}
\begin{defi}[Dense subsets]
	A subset \(T\) of a metric space \(X\) is dense in \(X\) if \(\overline{T}=X\).
\end{defi}
\begin{thm}[Baire Category Theorem]
	Let \(\{U_n\}_{n=1}^{\infty}\) be a sequence of dense open subsets of a complete metric space \(X\). Then \(\bigcap_{n=1}^{\infty}U_n\) is also dense in \(X\).
	\begin{IEEEproof}
		We shall prove that \(\bigcap_{n=1}^{\infty}U_n\) is dense in \(X\) by showing that for any ball with \(r>0\) and \(x\in X\)  there exists \(y\in\bigcap_{n=1}^{\infty}U_n\) such that \(y\in B(x,r)\).\bigbreak
		If \(\epsilon>0\) there exists \(y_1\in U_1\) such that \(y_1\in B(x,\epsilon)\). Because \(B(x,\epsilon)\) and \(U_1\) are both open, there exists \(1>r_1>0\) such that \(B(y_1,r_1)\subseteq U_1\cap B(x,\epsilon)\) and by shrinking \(r_1\) we have \(\overline{B(y_1,r_1)}\subseteq U_1\cap B(x,\epsilon)\). This procedure can be repeated by replacing \(B(x,\epsilon)\) by \(B(y_1,r_1)\), and finding \(y_2\in U_2\cap B(y_1,r_1)\) with \(1/2>r_2>0\) such that \(\overline{B(y_2,r_2)}\subseteq U_2\cap B(y_1,r_1)\).\footnote{Such \(y_2,r_2\) exist because \(y_1\in X\) and \(U_2\) is dense in \(X\), so for every \(r_1\)-ball of \(y_1\) there exists \(y_2\in U_2\) such that \(y_2\) is in this ball.}
		We can then define a cauchy sequence \(\{y_n\}_{n=1}^{\infty}\) with each \(y_n\) satisfying \(\overline{B(y_n,r_n)}\subseteq U_{n}\cap B(y_{n-1}, r_{n-1})\) with \(1/n>r_n>0\). Because \(X\) is complete, we know that \(\lim_{n\rightarrow\infty}y_n=y\) with \(y\in X\). If \(y\notin\bigcap_{n=1}^{\infty}U_n\) then there exists \(n\) such that \(y\notin B(y_n,r_n)\). If \(m>n\) Then \(y_m\in \overline{B(y_m,r_m)}\cap B(y_n,r_n)\). But then \((y\in B(y_n,r_n)\), a contradiction. Therefore \(y\in\bigcap_{n=1}^{\infty}U_n\).
	\end{IEEEproof}
\end{thm}
\begin{defi}[Nowhere dense]
	A subset \(Y\) of \(X\) is nowhere dense if \(\overline{Y}\) has no interior points, that is, if
	\begin{equation*}
		\text{int}(\overline{Y})=\emptyset.
	\end{equation*}
\end{defi}
\stepcounter{section}
\section{Products of metric spaces}
	The properties and metric definitions that follow are numbered after the properties in the Gamelin "Introduction to Topology book". Let \((X_1,d_1),\ldots,(X_n,d_n)\) be metric spaces. The product set \(X=X_1\times\ldots\times X_n\) consists of all n-tuples \((x_1,\ldots,x_n)\), where \(x_k\in X_k,\; 1\leq k\leq n\).
\begin{enumerate}[label=(4.\arabic*)]
	\item \(d(x,y)=\big[d_1(x_1,y_1)^2+\ldots+d_n(x_n,y_n)^2\big]^{1/2}\).
	\item \(\text{max}\big(d_1(x_1,y_1),\ldots,d_n(x_n,y_n)\big)\).
	\item \(d(x,y)=d(x_1,y_1)+\ldots+d_n(x_n,y_n)\).
	\item A sequence \(\{x^{j}=x_k^j\}_{j=1}^{\infty}\) converges to \(x=(x_1,\ldots,x_n)\) in \(X\) iff for each \(k\) the sequence of component entries \(\{x_k^j\}_{j=1}^{\infty}\) converges to \(x_k\) in \(X_k\).
	\item \(d_k(x_k,y_k)\leq d(x,y),\qquad x,y\in X, 1\leq K\leq n\).
\end{enumerate}
\begin{thm}
	Suppose that \(d\) is a metric on \(X=X_1\times\ldots\times X_n\) that satisfies property \(4.4\). Then the open sets in \((X,d)\) are the unions of product sets of the form \(U_1\times\ldots\times U_n\), where \(U_j\) is an open subset of \(X_j,\;1\leq j\leq n\).
\end{thm}
\begin{thm}
	Let \((X_1,d_1),\ldots,(X_n,d_n)\) be complete metric spaces. Let \(d\) be a metric on \(X=X_1\times\ldots\times X_n\) that satisfies \((4.4)\) and \((4.5)\). Then \((X,d)\) is complete.
\end{thm}
\begin{cor}
	The \(n\)-dimensional Euclidean space \(\mathbb{R}^n\), with the usual metric
	\begin{equation*}
		\lvert x-y\rvert=\big[(x_1-y_1)^2+\ldots+(x_n-y_n)^2\big]^{1/2},\qquad x,y\in\mathbb{R}^n,
	\end{equation*}
	Is complete.
\end{cor}
\section{Compactness}
\begin{defi}[Cover]
	A family \(\{U_{\alpha}\}_{\alpha\in A}\) of sets is said to cover a set \(S\) if \(S\) is contained in the union of the \(U_{\alpha}\)'s.
\end{defi}
\begin{defi}[Open cover]
	An open cover of a metric space \(X\) is a family of open subsets of \(X\) that covers \(X\).
\end{defi}
\begin{defi}[Compactness]
	A metric space \(X\) is compact if every open cover has a finite subcover.
\end{defi}
\begin{defi}[Totally bounded]
	A metric space \(X\) is totally bounded if for each \(\epsilon>0\), there exists a finite number of open balls of radius \(\epsilon\) that cover \(X\).
\end{defi}
\begin{thm}
	The following are equivalent for a metric space \(X\):
	\begin{enumerate}
		\item \(X\) is compact.
		\item Every sequence in \(X\) has a convergent subsequence.
		\item \(X\) is totally bounded and complete.
	\end{enumerate}
\end{thm}
\begin{defi}[Bounded]
	A metric space \(X\) is bounded if there exists \(b>0\) such that \(d(x,y)<b\) for all \(x,y\in X\).
\end{defi}
\begin{lem}
	A totally bounded metric space is bounded.
\end{lem}
\begin{lem}
	Any subspace of a totally bounded metric space is totally bounded.
\end{lem}
\begin{lem}
	A subset \(E\) of \(\mathbb{R}^n\) is totally bounded iff \(E\) is bounded.
\end{lem}
\begin{thm}[Heine-Borel theorem]
	The following are equivalent for a subspace \(E\) of \(\mathbb{R}^n\).
	\begin{enumerate}
		\item \(E\) is compact.
		\item Every sequence in \(E\) has a convergent subsequence.
		\item \(E\) is closed and bounded.
	\end{enumerate}
\end{thm}
\begin{thm}
	Let \(X\) be a totally bounded metric space. Then every sequence in \(X\) has a Cauchy subsequence.
\end{thm}
\begin{defi}[Seperability]
	A metric space \(X\) is seperable if there is a dense subset of \(X\) that is countable. In other words, \(X\) is seperable iff there is a sequence \(\{x_j\}_{j=1}^{\infty}\) in \(X\) that is dense in \(X\).
\end{defi}
\begin{thm}
	A subspace of a separable metric space is separable.
\end{thm}
\begin{defi}[Base]
	A base of open sets for a metric space \(X\) is a family \(\mathscr{B}\) of open subsets of \(X\) such that every open subset of \(X\) is the union of sets in \(\mathscr{B}\).
\end{defi}
\begin{lem}
	A family \(\mathscr{B}\) of open subsets of a metric space \(X\) is a base of open sets iff for each \(x\in X\) and each open neighborhood \(U\) of \(x\), there exists \(V\in\mathscr{B}\) such that \(x\in V\) and \(V\subseteq U\).
\end{lem}
\begin{defi}[Second-countable]
	A metric is second-countable if there is a base of open sets that is at most countable.
\end{defi}
\begin{thm}
	A metric space is second-countable iff it is seperable.
\end{thm}
\begin{thm}[Lindelof's theorem]
	Suppose the metric space \(X\) is second-countable. Then every open cover of \(X\) has a countable subcover.
\end{thm}
\end{document}
