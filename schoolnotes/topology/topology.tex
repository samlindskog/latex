\documentclass[nobib,notoc]{tufte-handout}
\usepackage[utf8]{inputenc}
\usepackage[english]{babel}
\usepackage{amsmath}
\usepackage{amsthm}
\usepackage{amsfonts}
\usepackage{amssymb}
\usepackage{hyperref}
\usepackage{mathrsfs}
\usepackage{mathtools}
\usepackage{IEEEtrantools}
\usepackage{enumitem}

\renewcommand{\IEEEQED}{\IEEEQEDopen}

\begin{document}

\theoremstyle{definition}\newtheorem{defi}{Definition}[section]
\theoremstyle{definition}\newtheorem{axiom}{Axiom}[section]
\theoremstyle{definition}\newtheorem{thm}{Theorem}[section]
\theoremstyle{definition}\newtheorem{cor}{Corollary}[section]
\theoremstyle{definition}\newtheorem{lem}{Lemma}[section]
\theoremstyle{remark}\newtheorem*{notat}{Notation}
\theoremstyle{remark}\newtheorem*{rema}{Remark}
\theoremstyle{definition}\newtheorem{problem}{Problem}
%\renewcommand{\theproblem}{\arabic{problem}}
\newenvironment{prob}[1]{\protect\setcounter{problem}{#1}\addtocounter{problem}{-1}\begin{problem}}{\end{problem}}

\DeclarePairedDelimiter{\ceil}{\lceil}{\rceil}
\DeclarePairedDelimiter{\floor}{\lfloor}{\rfloor}

\title{Topology}
\author{Samuel Lindskog}
\maketitle

\setcounter{section}{1}

\section{Open and closed sets}
\begin{defi}[Metric]
	A \emph{metric} on a set \(X\) is a real-valued function \(d\) on \(X\times X\) that has the following properties:
	\begin{enumerate}[label=(\alph*)]
		\item For all \(x,y\in X\), \(d(x,y)\geq 0\).
		\item \(d(x,y)=0\) iff \(x=y\).
		\item For all \(x,y\in X\), \(d(x,y)=d(y,x)\).
		\item For all \(x,y,z\in X\), \(d(x,z)\leq d(x,y)+d(y,z)\).
	\end{enumerate}
\end{defi}
\begin{defi}[Metric space]
	A metric space \((X,d)\) is a set \(X\) equipped with a metric \(d\) on \(X\).
\end{defi}
\begin{defi}[Subspace]
	If \((X,d)\) is a metric space and \(Y\) is a subset of \(X\), then the restriction \(d'\) of \(d\) to \(Y\times Y\) is a metric on \(Y\), and \((Y,d')\) is called a subspace of \((X,d)\).
\end{defi}
\begin{rema}
	Any set \(X\) can be made into a discreet metric space by associating with \(X\) the metric \(d\) defined by
	\begin{equation*}
		d(x,y)=\begin{cases}
			1,&x\neq y\\
			0,&x=y
		\end{cases}
	\end{equation*}
\end{rema}
\begin{defi}[Open ball]
	The open ball \(B(x,r)\) with center \(x\in X\) and radius \(r>0\) is defined by
	\begin{equation*}
		B(x,r)=\{y\in X|d(x,y)<r\}.
	\end{equation*}
\end{defi}
\begin{defi}[Interior point]
	Let \(Y\) be a subset of \(X\). A point \(x\in X\) is an interior point of \(Y\) if there exists \(r>0\) such that \(B(x,r)\subseteq Y\). The set of interior points of \(y\) is the interior of \(Y\), and it is denoted by int\((Y)\).\footnote{int\((Y)\subseteq Y.\)}
\end{defi}
\begin{defi}[Open subset]
	A subset \(Y\) of \(X\) is open if int\((Y)=Y\).
\end{defi}
\begin{thm}
	Any open ball \(B(x,r)\) in a metric space \(X\) is an open subset of \(X\)
	\begin{IEEEproof}
		Suppose \(y\in B(x,r)\). Then \(d(x,y)<r\), and \(0<r-d(x,y)\). Suppose \(z\in B(y,r-d(x,y))\). If follows from the definition of a metric that \(d(x,z)\leq d(x,y) + d(y,z)\), so \(d(x,z)\leq d(x,y) + (r-d(x,y))=r\), so \(z\in B(x,r)\).
	\end{IEEEproof}
\end{thm}
\begin{thm}
	The union of a family of open subsets of a metric space \(X\) is an open subset of \(X\).
	\begin{IEEEproof}
		Suppose \(\{U_\alpha\}\,\alpha\in A\) a family of open subsets of \(X\). If \(x\in\bigcup_{\alpha\in A}U_\alpha\), then \(\exists\alpha\,(x\in U_\alpha)\), so there exits an open ball \(B(x,r)\) such that \(B(x,r)\subseteq U_\alpha\). Because \(x\in U_\alpha\Rightarrow x\in\bigcup_{\alpha\in A}{U_\alpha}\), then \(B(x,r)\subseteq\bigcup_{\alpha\in A}U_\alpha\).
	\end{IEEEproof}
\end{thm}
\begin{thm}
	A subset \(U\) of a metric space \(X\) is open iff \(U\) is a union of open balls in \(X\).
	\begin{IEEEproof}
		Theorem 1.1 and 1.3 prove the left implication. If \(U\) is an open subset of \(X\), then for all \(x\in U\), there exists \(r(x)>0\) such that \(B(x,r(x))\in U\), so \(\bigcup_{x\in U}B(x,r(x))=U\).
	\end{IEEEproof}
\end{thm}
\begin{thm}
	The intersection of any finite number of open subsets of a metric space is open.
	\begin{IEEEproof}
		Suppose \(x\in\bigcap_{n=1}^{m}U_n\), a finite union of open subsets of a metric space. Then for all \(n\), there exists \(r(n)>0\) such that \(B(x,r(n))\in U_n\). Let \(r=\text{min}(r(1)\ldots r(m))\). Then for all \(r(n)\) we see \(B(x,r)\subseteq B(x,r(n))\) and thus \(B(x,r)\subseteq\bigcap_{n=1}^{m}U_n\).
	\end{IEEEproof}
\end{thm}
\begin{thm}
	Let \(Y\) be a subspace of a metric space \(X\). Then a subset \(U\) of \(Y\) is open in \(Y\) iff \(U=V\cap Y\) for some open subset \(V\) of \(X\).
	\begin{IEEEproof}
		Suppose \(x\in V\cap Y\). Then there exists an open ball in \(X\) with radius \(r(x)\) such that \(B(x,r(x))\subseteq V\), and \(x\in Y\). Because \(Y\subseteq X\) we see that \(Y\cap B(x,r(x))=\{y\in X\cap Y|d(x,y)<r(x)\}=\{y\in Y|d(x,y)<r(x)\}\), by definition an open ball in \(Y\). Trivially \(V\cap Y\subseteq\bigcap_{x\in V\cap Y}Y\cap B(x,r(x))\) and by definition the reverse is true.\smallbreak
		To prove the converse, suppose \(x\in U\). Then there exits an open ball in \(Y\) with radius \(r(x)\) such that \(B(x, r(x))\in U\). It follows from conclusions reached above that if \(B'(x, r(x))\) is open in \(X\), then \(B'(x,r(x))\cap Y=B(x,r(x))\). Let \(V=\bigcup_{x\in U}B'(x,r(x))\). Then \(V\cap Y\subseteq U\), and \(x\in U\Rightarrow x\in V\).
\end{IEEEproof}
\end{thm}
\begin{defi}[Adherent point]
	Let \(Y\) be a subset of a metric space \(X\). A point \(x\in X\) is adherent to \(Y\) if for all \(r>0\)
	\begin{equation*}
		B(x,r)\cap Y\neq\emptyset
	\end{equation*}
\end{defi}
\begin{defi}[Closure]
	The closure of \(Y\) denoted by \(\overline{Y}\), consists of all points in \(X\) that are adherent to \(Y\).\footnote{\(Y\subseteq\overline{Y}\).}
\end{defi}
\begin{defi}[Closed subset]
	The subset \(Y\) is closed if \(Y=\overline{Y}\).\footnote{The empty set \(\emptyset\) and \(X\) are closed subsets of \(X\). Interestingly, \(X\) is also open in \(X\).}
\end{defi}
\begin{thm}
	\label{thm:1:closure}
If \(Y\) is a subset of a metric space \(X\), then the closure of \(Y\) is closed, i.e.
	\begin{equation*}
		\overline{\overline{Y}}=\overline{Y}
	\end{equation*}
\begin{IEEEproof}
	\(\overline{Y}\) contains all \(x\in X\) such that for all \(r>0\) in \(B(x,r)\cap Y\neq\emptyset\). Let \(y\in X\) with \(B(y,r')\cap\overline{Y}\neq\emptyset\) for \(r'>0\). Suppose to the contrary that there does not exist \(x\in X\) such that \(x=y\). Then there exists \(a=\text{min}\big(d(x,y)\big)>0\) such that \(\forall x\,(x\notin B(y,a))\), therefore \(B(y,a)\cap\overline{Y}=\emptyset\), a contradiction.
\end{IEEEproof}
\end{thm}
\begin{thm}
	A subset \(Y\) of a metric space \(X\) is closed iff the complement of \(Y\) is open.
	\begin{IEEEproof}
		If \(Y\) is closed, then \(Y\) contains all \(x\in X\) such that for all \(r>0\), \(B(x,r)\cap Y\neq\emptyset\). Therefore iff \(y\in Y^c\) the negation is true, i.e. there exists \(r'>0\) such that \(B(y,r')\cap Y=\emptyset\), and because \(Y^c\cup Y=X\) we have \(B(y,r')\subset Y^c\) and \(Y^c\) is open.
	\end{IEEEproof}
\end{thm}
\begin{thm}
	The intersection of any family of closed sets is closed. The union of any finite family of closed sets is closed.
	\begin{IEEEproof}
		Let \(\{Y_\alpha\}\) be a family of closed sets in \(X\), and \(\alpha\in A\), the number of elements in \(\{Y_\alpha\}\). Following the fact that a union of open subsets is open, and the intersection of finite open subsets is open, as well as the previous theorem, we see
		\begin{IEEEeqnarray*}{rCl}
			X\setminus\bigcup_{\alpha\in A}Y_{\alpha}&=&\bigcap_{\alpha\in A}X\setminus Y_\alpha\\
			X\setminus\bigcap_{\alpha\in A}Y_{\alpha}&=&\bigcup_{\alpha\in A}X\setminus Y_\alpha
		\end{IEEEeqnarray*}
	\end{IEEEproof}
\end{thm}
\begin{defi}[Convergent sequence]
	A sequence \(\{x_n\}_{n=1}^{\infty}\) in a metric space \(X\) converges to \(x\in X\) if
	\begin{equation*}
		\lim_{n\rightarrow\infty}d(x_n, x)=0
	\end{equation*}
	In this case, \(x\) is the limit of \(\{x_n\}\) and we write \(x_n\rightarrow x\), or
	\begin{equation*}
		\lim_{n\rightarrow\infty}x_n=x.
	\end{equation*}
\end{defi}
\begin{lem}
	The limit of a convergent sequence in a metric space is unique
	\begin{IEEEproof}
		Let \(lim_{n\rightarrow\infty}x_n=x,y\) and suppose to the contrary that \(x\neq y\). Then \(d(x,y)>0\) and for all \(\epsilon>0\) there exits \(\delta\) such that \(d(x_n,x)\) and \(d(x_n,y)\) are both less than \(\frac{\epsilon}{2}\). But then if \(\epsilon<d(x,y)\) then \(d(x_n,x)+d(x_n,y)<d(x,y)\), a contradiction.
	\end{IEEEproof}
\end{lem}
\begin{thm}
\label{thm:1:converge}
	Let \(Y\) be a subset of the metric space \(X\), then \(x\in X\) is adherent to \(Y\) iff there is a sequence in \(Y\) that converges to \(x\).
	\begin{IEEEproof}
		If \(x\) is adherent to \(Y\), then \(\forall r>0\), \(B(x,r)\cap Y\neq\emptyset\), i.e. for all \(r\) there exits \(y\in Y\) such that \(d(x,y_n)<r\). Using this fact we can construct a sequence that converges to \(x\). Let \(y_n\in Y\), and \(\{y_n\}\) be a sequence such that for all \(\epsilon>0\) there exists \(N\in\mathbb{N}\) such that \(n>N\) implies \(d(x,y)<\epsilon\).\bigbreak
		Let \(\{y_n\}\) be a sequence with \(y_n\in Y\), and let \(x\in X\). Let \(\{y_n\}\) be such that for all \(\epsilon>0\), \(n\in\mathbb{N}\) with \(n>N\) implies \(d(x,y_n)<\epsilon\). Then for all \(r>0\) there exists \(r=\epsilon\) such that \(y_n\in B(x,r)\), and thus \(B(x,r)\cap Y\neq\emptyset\) for all \(r>0\).
	\end{IEEEproof}
\end{thm}
\stepcounter{section}
\section{Completeness}
\begin{defi}[Cauchy sequence]
	A sequence \(\{x_n\}_{n=1}^{\infty}\) in a metric space \(X\) is a Cauchy sequence if
	\begin{equation*}
		\lim_{m,n\rightarrow\infty}d(x_n,x_m)=0.
	\end{equation*}
	In other words
	\begin{equation*}
		\forall\epsilon>0,\exists N\,(n,m\geq N\Rightarrow d(x_n,x_m)<\epsilon)
	\end{equation*}
\end{defi}
\begin{lem}
	\label{conviscauchy}
	A convergent sequence is a Cauchy sequence.\footnote{In a complete metric space the reverse is true.}
	\begin{IEEEproof}
		Suppose \(\{x_n\}\) in \(X\) a sequence that converges to \(x\) in \(X\). Then
		\begin{equation*}
			\forall\epsilon>0,\exists n,m>N\,\big(d(x_n,x),d(x_m,x)<\epsilon\big).
		\end{equation*}
		If we choose \(N\) such that \(d(x_n,x),d(x_m,x)<\frac{\epsilon}{2}\) then
		\begin{equation*}
			d(x_n,x)+d(x_m,x)<\epsilon\Rightarrow d(x_n,d_m)<\epsilon.
		\end{equation*}
	\end{IEEEproof}
\end{lem}
\begin{lem}
	\label{cauchysubconverge}
	If \(\{x_n\}\) is a Cauchy sequence and if there is a subsequence \(\{x_{n_k}\}_{k=1}^{\infty}\) of \(\{x_n\}\) that converges to \(x\), then \(\{x_n\}\) converges to \(x\).
	\begin{IEEEproof}
		Suppose \(\{x_n\}\) a convergent sequence and \(\{x_{n_k}\}_{k=1}^{\infty}\) a subsequence which converges to \(x\) then
		\begin{IEEEeqnarray*}{c}
			\forall\delta>0,\exists N\,(n_k>N\Rightarrow d(x_{n_k},x)<\delta)\\
			\forall\epsilon>0,\exists M\,(n>M\wedge n_k>M,N\Rightarrow d(x_n,x_{n_k})<\epsilon).
		\end{IEEEeqnarray*}
		Because \(d(x_n,x_{n_k})+d(x_{n_k},x)<\epsilon+\delta\) then \(d(x_n,x)<\epsilon+\delta\).
	\end{IEEEproof}
\end{lem}
\begin{defi}[Complete metric space]
	A metric space \(X\) is complete if every cauchy sequence in \(X\) converges.
\end{defi}
\begin{thm}
	A complete subspace \(Y\) of a metric space \(X\) is closed in \(X\)
	\begin{IEEEproof}
		If \(x\in\overline{Y}\), then \(\forall r>0,\exists B(x,r)\) such that \(B(x,r)\cap Y\neq\emptyset\), so \(\exists y\in Y\) such that \(d(x,y)<r\). It follows there exists a Cauchy sequence \(\{y_n\}\) in \(Y\) with limit \(x\) such that \(\forall r,\exists N\;(n>N\Rightarrow d(x,y_n)<r)\). And because every Cauchy sequence in \(Y\) converges, \(x\in Y\) and \(\overline{Y}=Y\).
	\end{IEEEproof}
\end{thm}
\begin{defi}[Uniform convergence]
	Let \(\{f_n\}_{n=1}^{\infty}\) be a sequence of functions from a set \(S\) to a metric space \(X\) and let \(f\) be a function from \(S\) to \(X\). The sequence \(\{f_n\}\) converges uniformly to \(f\) on \(S\) if for each \(\epsilon>0\) there exists an integer \(N\) such that \(d(f_n(s),f(s))<\epsilon\) for all integers \(n\geq N\) and for all \(s\in S\).
\end{defi}
\begin{defi}
	A sequence \(\{f_n\}\) of functions from \(S\) to \(X\) is a Cauchy sequence of functions if for each \(\epsilon>0\) there exists an integer \(N\) such that
	\begin{equation*}
		d(f_n(s),f_m(s))<\epsilon,\qquad\text{all }s\in S,\; n,m\geq N.
	\end{equation*}
\end{defi}
\begin{thm}
	Let \(S\) be a set, and let \(X\) be a complete metric space. If \(\{f_n\}_{n=1}^{\infty}\) is a Cauchy sequence of functions from \(S\) to \(X\), then there exists a function \(f\) from \(S\) to \(X\) such that \(\{f_n\}\) converges uniformly to \(f\)
	\begin{IEEEproof}
		If \(\{x_n\}\) a Cauchy sequence in a complete metric space \(X\), then \(\{x_n\}\) converges. Therefore, for each \(s\in S\), there exists \(a_s\in X\) such that \(lim_{n\rightarrow\infty}f_n(s)=a_s\). Let a \(f\) be a function from \(S\) to \(X\) defined by \(f(s)=a_s\). It follows from the definition of a Cauchy sequence of functions that for all \(\epsilon>0\) there exists \(N\in \mathbb{N}\) such that for all \(s\in S\), \(n>N\) implies \(d(f_n(s),f(s))<\epsilon\), so \(\{f_n\}\) converges uniformly.
	\end{IEEEproof}
\end{thm}
\begin{defi}[Dense subsets]
	A subset \(T\) of a metric space \(X\) is dense in \(X\) if \(\overline{T}=X\).
\end{defi}
\begin{thm}[Baire Category Theorem]
	Let \(\{U_n\}_{n=1}^{\infty}\) be a sequence of dense open subsets of a complete metric space \(X\). Then \(\bigcap_{n=1}^{\infty}U_n\) is also dense in \(X\).
	\begin{IEEEproof}
		We shall prove that \(\bigcap_{n=1}^{\infty}U_n\) is dense in \(X\) by showing that for any open ball \(B(x,\epsilon)\) with \(\epsilon>0\) and \(x\in X\)  there exists \(y\in\bigcap_{n=1}^{\infty}U_n\) such that \(y\in B(x,\epsilon)\).\bigbreak
		Because for all \(n\), \(U_n\) is dense in \(X\), there exists \(y_1\in U_1\) such that \(y_1\in B(x,\epsilon)\). Because \(B(x,\epsilon)\) and \(U_1\) are both open, there exists \(0<r_1<1\) such that \(B(y_1,r_1)\subseteq U_1\cap B(x,\epsilon)\), and by shrinking \(r_1\) we have \(\overline{B(y_1,r_1)}\subseteq U_1\cap B(x,\epsilon)\). This procedure can be repeated in \(B(y_1,r_1)\) by finding \(y_2\in U_2\cap B(y_1,r_1)\) with \(0<r_2<1/2\) such that \(\overline{B(y_2,r_2)}\subseteq U_2\cap B(y_1,r_1)\).\footnote{Such \(y_2,r_2\) exist because \(B(y_1,r_1)\subseteq B(x,\epsilon)\subseteq X\), and \(U_2\) is dense in and open in \(X\). Therefore for every \(r_1\)-ball of \(y_1\) contains an \(r_2\) ball of \(y_2\).} We can then define a Cauchy sequence \(\{y_n\}_{n=1}^{\infty}\) using this procedure by
		\begin{equation*}
			y_n=\begin{cases}
				\text{if }n=1,&y_n\in B(x,\epsilon)\\
				\text{if }n>1,&\overline{B(y_n,r_n)}\subseteq B(y_{n-1}, r_{n-1})
			\end{cases}
		\end{equation*}
		with each \(r_n\) satisfying \(0<r_n<1/n\). Because \(X\) is complete, we know that \(\lim_{n\rightarrow\infty}y_n=y\) with \(y\in X\). Suppose to the contrary that \(y\notin\bigcap_{n=1}^{\infty}U_n\). Then there exists \(k\geq 1\) such that \(y\notin B(y_k,r_k)\). If \(m>k\) Then \(y_m\in \overline{B(y_m,r_m)}\cap B(y_k,r_k)\). By theorems \ref{thm:1:converge} and \ref{thm:1:closure}, the limit of any convergent sequence in \(\overline{B(y_m,r_m)}\) is in itself. It follows that \(y\in B(y_k,r_k)\), a contradiction. Therefore \(y\in\bigcap_{n=1}^{\infty}U_n\text{ and } y\in B(x,\epsilon)\).
	\end{IEEEproof}
\end{thm}
\begin{defi}[Nowhere dense]
	A subset \(Y\) of \(X\) is nowhere dense if \(\overline{Y}\) has no interior points, that is, if
	\begin{equation*}
		\text{int}(\overline{Y})=\emptyset.
	\end{equation*}
\end{defi}
\stepcounter{section}
\section{Products of metric spaces}
	The properties and metric definitions that follow are numbered after the properties in the Gamelin "Introduction to Topology book". Let \((X_1,d_1),\ldots,(X_n,d_n)\) be metric spaces. The product set \(X=X_1\times\ldots\times X_n\) consists of all n-tuples \((x_1,\ldots,x_n)\), where \(x_k\in X_k,\; 1\leq k\leq n\).
\begin{enumerate}[label=(4.\arabic*)]
	\item \(d(x,y)=\big[d_1(x_1,y_1)^2+\ldots+d_n(x_n,y_n)^2\big]^{1/2}\).
	\item \(\text{max}\big(d_1(x_1,y_1),\ldots,d_n(x_n,y_n)\big)\).
	\item \(d(x,y)=d(x_1,y_1)+\ldots+d_n(x_n,y_n)\).
	\item A sequence \(\{x^{j}=(x_k^j)\}_{j=1}^{\infty}\) converges to \(x=(x_1,\ldots,x_n)\) in \(X\) iff for each \(k\) the sequence of component entries \(\{x_k^j\}_{j=1}^{\infty}\) converges to \(x_k\) in \(X_k\).
	\item \(d_k(x_k,y_k)\leq d(x,y),\qquad x,y\in X, 1\leq K\leq n\).
\end{enumerate}
\begin{thm}
	Suppose that \(d\) is a metric on \(X=X_1\times\ldots\times X_n\) that satisfies property \(4.4\). Then the open sets in \((X,d)\) are the unions of product sets of the form \(U_1\times\ldots\times U_n\), where \(U_j\) is an open subset of \(X_j,\;1\leq j\leq n\).
	\begin{IEEEproof}
		Suppose that \(U\) an open subset of \(X\) and \(y=(y_1,\ldots, y_n)\in U\). If \(1\leq m\leq\infty\) and \(1\leq k\leq n\), because each \(y_k\in B(y_k,1/m)\) it follows that \(y\) is an element of the product of open balls \(B(y_1,1/m)\times\ldots\times B(y_n,1/m)\).\footnote{Note that these could be closed balls as well. Open balls are used to satisfy the proof.} Suppose to the contrary that there does not exist \(\epsilon>0\) such that \(B(y_1,\epsilon)\times\ldots\times B(y_n,\epsilon)\subseteq U\). Then for all \(m\) there exist \(x^m=(x^{m}_{1},\ldots,x^{m}_{n})\in U^{c}\) such that \(x^m\in B(y_1,1/m)\times\ldots\times B(y_n,1/m)\) i.e. for all \(k\), \(x^{m}_{k}\in B(y_k,1/m)\). It follows
		\begin{equation*}
			\lim_{m\rightarrow\infty}d_{k}(x^{m}_{k}, y_k)=0.
		\end{equation*}
		But following property 4.4 this means
		\begin{equation*}
			\lim_{m\rightarrow\infty}d(x^{m},y)=0.
		\end{equation*}
		It follows that because \(x^{m}\in U^{c}\), \(y\in \overline{U^{c}}\). \(U^{c}\) is closed so \(y\in U^{c}\), a contradiction. Therefore each \(y\in U\) is contained in a subset of \(U\) which is the product of open balls in \(X_k\).
	\end{IEEEproof}
\end{thm}
\begin{thm}
	Let \((X_1,d_1),\ldots,(X_n,d_n)\) be complete metric spaces. Let \(d\) be a metric on \(X=X_1\times\ldots\times X_n\) that satisfies \((4.4)\) and \((4.5)\). Then \((X,d)\) is complete.
	\begin{IEEEproof}
		Suppose \(\{y_m\}_{m=1}^{\infty}\) a Cauchy sequence in \(X\). Then
		\begin{equation*}
			\forall\epsilon,\,\exists N\big(l,l'>N\Rightarrow d(y_l,y_{l'})<\epsilon\big).
		\end{equation*}
		Because \(X\) satisfies property 4.5, for \(1\leq k\leq n\)
		\begin{equation*}
			\forall\epsilon,\,\exists N\big(l,l'>N\Rightarrow d(y_{l_k},y_{l'_k})<\epsilon\big).
		\end{equation*}
	And thus \(\{y_{m_k}\}_{m=1}^{\infty}\) is a Cauchy sequence in \(X_k\). Because \(X_k\) is complete, this Cauchy sequence converges to a point \(z_k\in X_k\). Following property 4.4, \(\{y_m\}_{m=1}^{\infty}\) converges to \(z=(z_1,\ldots,z_n)\in X\).
	\end{IEEEproof}
\end{thm}
\begin{cor}
	The \(n\)-dimensional Euclidean space \(\mathbb{R}^n\), with the usual metric
	\begin{equation*}
		\lvert x-y\rvert=\big[(x_1-y_1)^2+\ldots+(x_n-y_n)^2\big]^{1/2},\qquad x,y\in\mathbb{R}^n,
	\end{equation*}
	Is complete.
\end{cor}
\section{Compactness}
\begin{defi}[Cover]
	A family \(\{U_{\alpha}\}_{\alpha\in A}\) of sets is said to cover a set \(S\) if \(S\) is contained in the union of the \(U_{\alpha}\)'s.
\end{defi}
\begin{defi}[Open cover]
	An open cover of a metric space \(X\) is a family of open subsets of \(X\) that covers \(X\).
\end{defi}
\begin{defi}[Compactness]
	A metric space \(X\) is compact if every open cover has a finite subcover.
\end{defi}
\begin{defi}[Totally bounded]
	A metric space \(X\) is totally bounded if for each \(\epsilon>0\), there exists a finite number of open balls of radius \(\epsilon\) that cover \(X\).
\end{defi}
\begin{thm}
	The following are equivalent for a metric space \(X\):
	\begin{enumerate}
		\item \(X\) is compact.
		\item Every sequence in \(X\) has a convergent subsequence.
		\item \(X\) is totally bounded and complete.
	\end{enumerate}
	\begin{IEEEproof}
		\label{betterhb}
		\emph{PROOF 1 IMPLES 2} - Suppose \(X\) is compact, and \(\{x_n\}_{n=1}^{\infty}\) a sequence in \(X\). Suppose to the contrary that for all \(x\in X\) there exists \(\epsilon(x)>0\) such that only a finite number of terms in \(\{x_n\}\) lie in each \(B(x,\epsilon(x))\). The set of all such \(B(x,\epsilon(x))\) form an open cover for \(X\), so a finite subcover of said cover exists, and thus \(\{x_n\}\) has a finite number of elements. This implies \(\mathbb{N}\) is finite, a contradiction. Therefore there exists \(x\in X\) such that for all \(\epsilon\), an infinite number of elements in \(x_n\) lie in \(B(x,\epsilon)\). We can now construct a Cauchy subsequence of \(\{x_n\}\) using diagonalization\footnote{This technique is cracked and will be used again.} which converges to \(x\). Let \(\{x_{1n}\}\) be the original sequence \(\{x_n\}\). Let \(\{x_{kn}\}\), \(k\geq 2\) be a subsequence of \(\{x_{(k-1)n}\}\) such that \(x_{kn}\in B(x,1/k)\). Then the sequence \(\{x_{nn}\}_{n=1}^{\infty}\) is a Cauchy subsequence of \(\{x_n\}\) which converges to \(x\).
		\bigbreak
		\emph{PROOF 2 IMPLIES 3} - If every sequence in \(X\) has a convergent subsequence, then by lemma \ref{cauchysubconverge}, every Cauchy sequence in \(X\) converges and \(X\) is complete. Suppose \(\mathscr{F}=\{B(x,\epsilon)\,|\,x\in X\text{ and } \epsilon>0\}\). Then there exists \(\mathscr{T}\subseteq\mathscr{F}\) such that \(\mathscr{T}\) is finite and covers \(X\), so \(X\) is totally bounded.
		\bigbreak
		\emph{PROOF 3 IMPLIES 1} - Suppose \(X\) is totally bounded and complete. Following theorem \ref{tbcontcauchy}, every sequence in \(X\) contains a Cauchy subsequence, and thus every sequence in \(X\) has a convergent subsequence. If \(X\) is totally bounded and complete, and every sequence in \(X\) has a convergent subsequence, then following theorems \ref{tbseperable} and \ref{scsep} \(X\) is second-countable, and following theorem \ref{lindelofs} every open cover of \(X\) has a countable subcover. Let \(\{U_n\}_{n=1}^{\infty}\) cover \(X\). Suppose to the contrary that no finite subcover \(\{U_n\}_{n=1}^{m}\) exists. Then for all \(m\) we see that \(X\setminus\bigcup\{U_n\}_{n=1}^{m}\neq\emptyset\). We can then define a sequence \(\{x_j\}_{j=1}^{\infty}\) such that
		\begin{equation*}
			x_j\in X\setminus\bigcup\{U_n\}_{n=1}^{j}
		\end{equation*}
		The complement of any union of open sets is closed, and every sequence in \(X\) has a convergent subsequence. Therefore following lemma \ref{conviscauchy} a Cauchy subsequence of \(\{x_j\}\) exists such that this subsequence converges to a point \(x\) not in the open cover \(\{U_n\}\), and therefore not in \(X\). Thus \(X\) is not complete, a contradiction.
	\end{IEEEproof}
\end{thm}
\begin{defi}[Bounded]
	A metric space \(X\) is bounded if there exists \(b>0\) such that \(d(x,y)<b\) for all \(x,y\in X\).
\end{defi}
\begin{lem}
	A totally bounded metric space is bounded.
	\begin{IEEEproof}
		Let \(X\) be a totally bounded metric space. Then every point \(x,y\in X\) is contained in an element of a finite family \(\mathscr{F}\) of \(\epsilon\)-balls centered at \(z,w\in X\) respectively. It follows that
		\begin{IEEEeqnarray*}{rCl}
			d(x,y)&<&d(x,z)+d(z,w)+d(w,y)\\
			&<&2\epsilon+d(z,w).
		\end{IEEEeqnarray*}
		Because \(z,w\) are in a finite number of \(\epsilon\)-balls, let
		\begin{equation*}
		c=\text{max}\{d(z,w)\,|\,B(z,\epsilon),B(w,\epsilon)\in\mathscr{F}\}.
		\end{equation*}
		It follows that
		\begin{equation*}
			d(x,y)<2\epsilon+c.
		\end{equation*}
		Thus \(X\) is bounded.
	\end{IEEEproof}
\end{lem}
\begin{rema}
	A bounded metric space is not necessarily totally bounded. For example an infinite set with the discreet metric.
\end{rema}
\begin{lem}
	Any subspace of a totally bounded metric space is totally bounded.
	\begin{IEEEproof}
		If \(X\) be totally bounded and \(L\subseteq X\), then for each \(x\in L\) we have \(x\in X\) and thus \(x\) is an element of any open cover of \(X\), and \(L\) is totally bounded.
	\end{IEEEproof}
\end{lem}
\begin{lem}
	A subset \(E\) of \(\mathbb{R}^n\) is totally bounded iff \(E\) is bounded.
	\begin{IEEEproof}
	Suppose \(E\subseteq [-l,l]^n\subseteq\mathbb{R}^n\) with \(l>0\), let \(c=\ceil{\frac{ln}{\epsilon}}\), let \(k\in\mathbb{N}\) with \(k\leq n\), let \(\epsilon\in\mathbb{R}\) with \(\epsilon>0\), and \(A\) a set of \(n\)-tuples with
	\begin{equation*}
			  A=\{(a_1,\ldots,a_n)\;| \;a_k=\frac{i\epsilon}{n},\; -c\leq i\leq c,\;i\in\mathbb{Z}\}
	\end{equation*}
		Suppose \(x=(x_1,\ldots,x_n)\in[-l,l]^n\). For all \(x\) there exists \(y=(y_1,\ldots,y_n)\in A\) such that for all \(j\in\mathbb{N}\) with \(j\leq n\) we have component \(x_j\) of \(x\) and component \(y_j\) of \(y\) with \(\lvert x_j-y_j\rvert<\frac{\epsilon}{n}\). It follows from the triangle inequality that \(d(x,y)<\epsilon\), so \(x\in B(y,\epsilon)\), and \(\bigcap_{\alpha\in A}B(\alpha,\epsilon)\) is a cover for \([-l,l]\). It follows that because \(E\subseteq [-l,l]^n\) then \(E\) is totally bounded.
	\end{IEEEproof}
\end{lem}
\begin{thm}[Heine-Borel theorem]
	The following are equivalent for a subspace \(E\) of \(\mathbb{R}^n\).
	\begin{enumerate}
		\item \(E\) is compact.
		\item Every sequence in \(E\) has a convergent subsequence.
		\item \(E\) is closed and bounded.
	\end{enumerate}
\end{thm}
\begin{thm}
	\label{tbcontcauchy}
	Let \(X\) be a totally bounded metric space. Then every sequence in \(X\) has a Cauchy subsequence.\footnote{Proof of this in the first implication of theorem \ref{betterhb}.}
\end{thm}
\begin{defi}[Seperability]
	A metric space \(X\) is seperable if there is a dense subset of \(X\) that is countable. In other words, \(X\) is seperable iff there is a sequence \(\{x_j\}_{j=1}^{\infty}\) in \(X\) that is dense in \(X\).
\end{defi}
\begin{thm}
	A subspace of a separable metric space is separable.
	\begin{IEEEproof}
		Suppose \(X\) seperable metric space, and \(E\subseteq X\). Then there exists a sequence \(\{x_n\}_{n=1}^\infty\) in \(X\) such that \(\{x_n\}\) is dense in \(X\). It follows \(\overline{E}\subseteq\overline{\{x_n\}}\). Let \(y\) be an adherent point of \(E\). Then \(y\) is adherent to \(X\) and there exists \(x_n\) such that for all \(k\in\mathbb{N}\) we have \(y\in B(x_n, 1/2k)\). Define a sequence \(\{y_n\}_{n=1}^\infty\) by choosing single element of the intersection of \(B(x_n, 1/2k)\) and \(E\) for all \(k\).\footnote{Such a sequence is possible because all combinations of these balls are countable}. It follows that
		\begin{equation*}
			d(y_n,y)\leq d(y_n,x_n)+d(x_n, y)\leq 1/k
		\end{equation*}
		So \(\{y_n\}\) is dense and countable in \(E\), and \(E\) is seperable.
	\end{IEEEproof}
\end{thm}
\begin{thm}
	\label{tbseperable}
	A totally bounded metric space is separable.
	\begin{IEEEproof}
		Let \(X\) be a metric space and \(\mathscr{F}\) be a family containing finite covers of \(X\) comprised of open \(1/k\)-balls, \(k\in\mathbb{N}\) with one cover for each \(k\). \(\mathscr{F}\) is obviously countable. If \(B(x_\alpha, 1/k)\in\mathscr{F}\) then \(\{x_\alpha\}_{\alpha\in\mathscr{F}}\) is a countable dense subset of \(X\), and \(X\) is separable.
	\end{IEEEproof}
\end{thm}
\begin{defi}[Base]
	A base of open sets for a metric space \(X\) is a family \(\mathscr{B}\) of open subsets of \(X\) such that every open subset of \(X\) is the union of sets in \(\mathscr{B}\).
\end{defi}
\begin{lem}
	\label{baseinocover}
	A family \(\mathscr{B}\) of open subsets of a metric space \(X\) is a base of open sets iff for each \(x\in X\) and each open neighborhood \(U\) of \(x\), there exists \(V\in\mathscr{B}\) such that \(x\in V\) and \(V\subseteq U\).
	\begin{IEEEproof}
		Suppose \(\mathscr{B}\) is a base for \(X\). Evidently, every open neighborhood of any point in \(X\) is a union of sets in \(\mathscr{B}\). Suppose \(U\) an open neighborhood of \(x\in X\). Then there is an open ball \(B(x,\epsilon)\subseteq U\), and \(B(x,\epsilon)\) is a union of sets in \(\mathscr{B}\). Therefore there exists \(V\) such that \(x\in V\subseteq B(x,\epsilon)\) and \(x\in V\subseteq U\).
	\end{IEEEproof}
\end{lem}
\begin{defi}[Second-countable]
	A metric is second-countable if there is a base of open sets that is at most countable.
\end{defi}
\begin{thm}
	\label{scsep}
	A metric space is second-countable iff it is seperable.
	\begin{IEEEproof}
		Suppose \(X\) is second countable, and \(\{U_n\}_{n=1}^{\infty}\) a countable base for \(X\). If \(y\in X\) then for all \(\epsilon>0\) there exists an open set \(U_k\), \(k\geq 1\) such that
		\begin{equation*}
		U_k\subseteq B(y,\epsilon)
		\end{equation*}
		Thus we can construct a sequence \(\{x_n\,|\, x\in U_n\}_{n=1}^\infty\) where for every \(\epsilon\)-ball of \(y\) there exists \(k\geq 1\) such that \(x_k\) in this ball. Therefore \(\{x_n\}\) is countable and dense in \(X\) and \(X\) is seperable.\footnote{Note that a countable base does not mean a countable amount of open sets, i.e. \(\lvert\mathbb{N}\rvert<\lvert\mathscr{P}(\mathbb{N})\rvert\).}\bigbreak
		Suppose \(X\) is seperable. Then there exists a sequence \(\{x_n\}_{n=1}^\infty\) that is dense in \(X\). Let \(U\) be an open subset of \(X\) and \(y\in U\). Then there exists \(k\in\mathbb{N}\) such that \(B(y,2/k)\subseteq U\). Because \(\{x_n\}\) is dense in \(X\) there exists \(x_n\in B(y,1/k)\), and it follows from the triangle inequality that \(y\in B(x_n,1/k)\subseteq U\). Thus \(\{B(x_n,1/k)\,|\,n\geq 1,\,k\in\mathbb{N}\}\) is a countable base for \(X\) and \(X\) is second-countable.
	\end{IEEEproof}
\end{thm}
\begin{thm}[Lindelof's theorem]
	\label{lindelofs}
	Suppose the metric space \(X\) is second-countable. Then every open cover of \(X\) has a countable subcover.
	\begin{IEEEproof}
		Suppose \(\mathscr{B}\) a countable base of \(X\), and \(\{U_\alpha\}_{\alpha\in A}\) an open cover of \(X\). For all \(x\) there exists an open ball of \(x\), therefore \(x\) is contained in a set \(V\in\mathscr{B}\). We know that \(X\subseteq\bigcup\{U_\alpha\}_{\alpha\in A}\), so following lemma \ref{baseinocover}, for all \(x\) there exists \(\alpha(V)\) such that \(x\in U_{\alpha(V)}\) and \(V\subseteq U_{\alpha(V)}\). Therefore \(\{U_{\alpha(v)}\,|\,V\in\mathscr{B}\}\) is a countable subcover of \(\{U_{\alpha}\}_{\alpha\in A}\).
	\end{IEEEproof}
\end{thm}
\begin{thm}
	A compact metric space is seperable and second-countable.
\end{thm}
\section{Continuity}
\begin{defi}[Continuous function]
	Let \((X,d)\) and \((Y,\rho)\) be metric spaces. A function \(f:X\rightarrow Y\) is continuous at \(x\in X\) if whenever \(\{x_n\}\) is a sequence in \(X\) such that \(x_n\rightarrow x\), then \(f(x_n)\rightarrow f(x)\).
\end{defi}
\begin{thm}
	The function \(f:X\rightarrow Y\) is continuous at the point \(x\in X\) iff for each \(\epsilon>0\) there exists \(\delta>0\) such that whenever \(z\in X\) satisfies \(d(x,z)<\delta\) then \(\rho(f(x),f(x))<\epsilon\).
\end{thm}
\begin{thm}
	The following are equivalent for a function \(f\) from a metric space \(X,d\) to a metric space \((Y,\rho)\):
	\begin{enumerate}
		\item \(f\) is continuous
		\item For each \(x\in X\) and \(\epsilon>0\), there exists \(\delta>0\) such that whenever \(z\in X\) satisfies \(d(x,z)<\delta\), then \(\rho(f(x),f(z))<\epsilon\)
		\item \(f^{-1}(V)\) is an open subset of \(X\) for every open subset \(V\) of \(Y\).
	\end{enumerate}
\end{thm}
\begin{defi}[Uniform continuity]
	A function \(f:X\rightarrow Y\) is uniformly continuous if for each \(\epsilon>0\) there exists \(\delta>0\) such that whenever \(x,z\in X\) satisfies \(d(x,z)<\delta\), then \(\rho(f(x),f(z))<\epsilon\).
\end{defi}
\begin{thm}
	Let \(X\) and \(Y\) be metric spaces and suppose that \(X\) is compact. Then every continuous function \(f\) from \(X\) to \(Y\) is uniformly continuous.
\end{thm}
\begin{defi}[Homeomorphism]
	A function \(f\) from one metric space to another is a homeomorphism if \(f\) is continuous, one-to-one, and onto, and if the inverse function \(f^{-1}\) is continuous.\footnote{Such a function \(f\) is called bicontinuous. A homeomorphism preserves all properties of a metric space that are definable in terms of open sets only.}
\end{defi}
\clearpage
\section{Topological spaces}
\begin{defi}[Topology]
	Let \(X\) be a set. A family \(\mathscr{T}\) of subsets of \(X\) is a topology for \(X\) if \(\mathscr{T}\) has the following three properties
	\begin{enumerate}
		\item Both \(X\) and the empty set belong to \(\mathscr{T}\).
		\item Any union of sets in \(\mathscr{T}\) belongs to \(\mathscr{T}\).
		\item Any finite intersection of sets in \(\mathscr{T}\) belongs to \(\mathscr{T}\).
	\end{enumerate}
	A \emph{topological space} is a pair \((X,\mathscr{T})\) where \(X\) is a set and \(\mathscr{T}\) is a topology for \(X\). The sets in \(\mathscr{T}\) are called open sets.
\end{defi}
\begin{defi}[Metrizable]
	A topological space is metrizable if the topology for \(X\) is the metric topology associated with some metric \(X\).\footnote{Some topologies cannot be determined by any metric i.e. their open sets are not open under any metric.}
\end{defi}
\begin{defi}[Closed subset]
	A subset \(S\) of \(X\) is defined to be closed if \(X\setminus S\) is open.
\end{defi}
\begin{defi}[Neighborhood]
	A subset \(S\) of \(X\) is a neighborhood of a point \(x\) is there is an open set \(U\) such that \(x\in U\) and \(U\subseteq S\).
\end{defi}
\begin{defi}[Interior point]
	A point \(x\in X\) is an interior point of \(S\) if \(S\) is a neighborhood of \(x\). The set of interior points of \(S\) is called the interior of \(S\) and is denoted \(\text{int}(S)\).
\end{defi}
\begin{thm}
	A subset \(S\) of a topological space \(X\) is open iff \(S=\text{int}(S)\).
\end{thm}
\begin{thm}
	If \(S\) is a subset of a topological space \(X\), then \(\text{int}(S)\) is an open subset of \(X\).
\end{thm}
\begin{defi}[Adherent point]
	A point \(x\in X\) is adherent to a subset \(S\) of \(X\) if \(S\) meets every neighborhood of \(x\). The closure of \(S\), denoted \(\overline{S}\), is the set of points in \(X\) which are adherent to \(S\).
\end{defi}
\begin{thm}
	A subset \(S\) of a topological space \(X\) is closed iff \(S=\overline{S}\)
\end{thm}
\begin{thm}
	If \(S\) is a subset of topological space \(X\), then \(\overline{S}\) is closed.
\end{thm}
\begin{defi}[Convergence]
	A sequence of points \(\{x_i\}\) in a topological space \(X\) coverges to \(x\in X\) if for every open neighborhood \(U\) of \(x\), there is an integer \(N\) such that \(x_i\in U\) for all \(i>N\).
\end{defi}
\begin{thm}
	If \(S\) is a subset of a topological space \(X\) and if a sequence \(\{x_i\}_{i=1}^{\infty}\) is \(S\) converges to \(x\in X\), then \(x\in\overline{S}\).
\end{thm}
\begin{defi}[Boundary point]
	A point \(x\in X\) is a boundary point of a subset \(S\) of \(X\) if \(x\) is adherent to both \(S\) and \(X\setminus S\). The boundary of \(S\), denoted \(\partial S\), is the set of boundary points of \(S\).
\end{defi}
\begin{thm}
	\(\overline{S}\) is the disjoint union of \(\text{int}(S)\) and \(\partial S\).
\end{thm}
\section{Subspaces}
\begin{defi}[Relative topology]
	Let \((x,\mathscr{T})\) be a topological space and let \(S\) be a subset of \(X\). Then the family
	\begin{equation*}
		\mathscr{L}=\{U\cap S|U\in\mathscr{T}\}
	\end{equation*}
	of subsets of \(S\) is a topology for \(S\) called the relative topology inherited from \((X,\mathscr{T})\). The sets \(V\in\mathscr{L}\) are relatively open subsets of \(S\), and the sets \(S\setminus V,\,V\in\mathscr{L}\) are relatively closed subsets of \(S\). We call \((S,\mathscr{L})\) a subspace of \((X,\mathscr{T})\).
\end{defi}
\begin{rema}
	If \(X\) is a metric space and if \(Y\) is a metric subspace of \(X\), then the metric topology for \(Y\) coincides with the relative topology for \(Y\) inherited from the metric topology of \(X\).\footnote{Going forward subspace will often refer to the relative topology inherited from a parent space.}
\end{rema}
\begin{thm}
Let \(S\) be a subspace of a topological space \(X\). A subset \(E\) of \(S\) is relatively closed in \(S\) iff \(E\) is the intersection of \(S\) and a closed subset of \(X\).
\end{thm}
\begin{thm}
	Let \(S\) be a subspace of a topological space \(X\) and let \(E\) be a subset of \(S\). Then the relative closure of \(E\) in \(S\) is \(\overline{E}\cap S\), where \(\overline{E}\) is the closure of \(E\) in \(X\).
\end{thm}
\end{document}
