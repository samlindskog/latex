\documentclass{article}
\usepackage[utf8]{inputenc}
\usepackage[english]{babel}
\usepackage{enumitem}
\usepackage{amsmath}
\usepackage{amssymb}
\usepackage{amsfonts}
\usepackage{mathrsfs}
\usepackage{mathtools}
\usepackage{IEEEtrantools}
\usepackage{geometry}
\geometry{
a4paper,
total={170mm,257mm},
left=20mm,
top=20mm,
}

\renewcommand{\IEEEQED}{\IEEEQEDopen}
\DeclarePairedDelimiter{\ceil}{\lceil}{\rceil}

\title{Homework 2}
\author{Samuel Lindskog}
\date\today
%\clearpage clears page

\begin{document}
\maketitle %This command prints the title based on information entered above

\subsection*{Problem 1}
\begin{enumerate}
	\item Show that \(f_n\) converges to \(0\) pointwisely
		\begin{IEEEproof}
			For all \(x\) in \((0,1)\), for all real \(\epsilon>0\) there exists an integer \(N=\ceil{\text{log}_x\epsilon}\) such that for all integers \(n>N\) we have \(d(x^n,0)<\epsilon\).
		\end{IEEEproof}
	\item Does \(f_n\) converge to \(0\) uniformly?
		\begin{IEEEproof}
			No. Suppose to the contrary that \(\{f_n\}\) converges uniformly. Then \(\forall\epsilon>0,\exists N\in\mathbb{Z}\) such that \(n>N\) implies \(\forall x\in X,d(f_n(x),0)<\epsilon\). But there exists \(\epsilon\) in \((0,1)\) such that \(\forall N\in\mathbb{Z}\), there exists \(x_n=\epsilon^{1/(n+1)}\) with \(n>0, N\) such that \((x_n)^n>\epsilon\).
		\end{IEEEproof}
\end{enumerate}
\subsection*{Problem 2}
Let \(Y\) be a subspace of \(X\) and let \(S\) be a subset of \(Y\). show that the closure of \(S\) in \(Y\) coincides with \(\overline{S}\cap Y\) where \(\overline{S}\) is the closure of \(S\) in \(X\).
\begin{IEEEproof}
	Suppose \(y\) is in the closure of \(S\) in \(Y\). Then for all \(r>0\) there exists \(B(y,r)\cap S\neq 0\). Then a sequence \(\{y_n\}_{n=1}^{\infty}\) exists with \(y_n\in Y\) such that \(\forall r>0\) there exists \(N\in\mathbb{N}\) such that \(n>N\) implies \(d(y_n,y)<r\). Therefore \(\{y_n\}\) is a sequence in \(Y\) which converges to \(y\) and following theorem \(1.11\), because \(y\in X\) this implies \(y\in\overline{S}\wedge y\in Y\), which is logically equivalent to \(y\in\overline{S}\cap Y\).
\end{IEEEproof}
\subsection*{Problem 3}
A sequence \(\{x_k\}_{k=1}^{\infty}\) in a metric space \((X,d)\) is a fast cauchy sequence if
\begin{equation*}
	\sum_{k=1}^{\infty}d(x_k,x_{k+1})<\infty.
\end{equation*}
Show that a fast Cauchy sequence is a Cauchy sequence.
\begin{IEEEproof}
	If \(\{x_k\}_{k=1}^{\infty}\) a fast sequence then the sum of all \(x_k\) is a finite real number. Then there exists \(a\in\mathbb{R}\) with \(a>0\), and \(N\in\mathbb{N}\) such that
	\begin{IEEEeqnarray}{c}
		a-\lim_{n\rightarrow\infty}\sum_{k=1}^{n}d(x_k,x_{k+1})=0\\
		a-\sum_{k=1}^{N}d(x_k,x_{k+1})=\lim_{n\rightarrow\infty}\sum_{N+1}^{n}d(x_k,x_{k+1}).
	\end{IEEEeqnarray}
	It follows from the definition of a metric that for \(l,m\in\mathbb{N}\) with \(l,m>N\)
	\begin{IEEEeqnarray}{c}
		a-\sum_{k=1}^{N}d(x_k,x_{k+1})\geq d(x_l,x_m).
	\end{IEEEeqnarray}
	Following equation one, we can establish that for all \(\epsilon>0\), with \(\epsilon>a-\sum_{k=1}^{N}d(x_k,x_{k+1})\), there exists \(N\) such that \(l,m>N\) implies \(d(x_l,x_m)<\epsilon\), and thus \(\{x_k\}\) is Cauchy.
\end{IEEEproof}
\end{document}
