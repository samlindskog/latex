\documentclass{article}
\usepackage[utf8]{inputenc}
\usepackage[english]{babel}
\usepackage{enumitem}
\usepackage{amsmath}
\usepackage{amsfonts}
\usepackage{mathrsfs}
\usepackage{IEEEtrantools}
\usepackage{geometry}
\geometry{
 a4paper,
 total={170mm,257mm},
 left=20mm,
 top=20mm,
}

\renewcommand{\IEEEQED}{\IEEEQEDopen}

\title{Homework}
\author{Samuel Lindskog}
\date\today
%\clearpage clears page

\begin{document}
\maketitle %This command prints the title based on information entered above

\subsection*{Problem 1}
Let \(n\in\mathbb{N}\) be a positive integer. Suppose \(x_1,\ldots, x_n\), \(y_1,\ldots, y_n\), \(z_1,\ldots,z_n\in\mathbb{R}\). Show that
\begin{equation*}
	\sqrt{\sum_{i=1}^{n}(x_i-z_i)^2}\leq\sqrt{\sum_{i=1}^{n}(x_i-y_i)^2}+\sqrt{\sum_{i=1}^{n}(y_i-z_i)^2}.
\end{equation*}
\bigskip
\begin{IEEEproof}
	We shall prove by induction on \(n\). Suppose \(n=1\). Then
	\begin{IEEEeqnarray*}{rCl}
		\sqrt{(x_i-y_i)^2}+\sqrt{(y_i-z_i)^2}&=&(x_i-y_i)+(y_i-z_i)\\
		&=&(x_i-z_i)\\
		&=&\sqrt{(x_i-z_i)^2}
	\end{IEEEeqnarray*}
	Suppose by inductive hypothesis that
	\begin{IEEEeqnarray*}{rCl}
		\sqrt{\sum_{i=1}^n(x_i-z_i)^2}\leq\sqrt{\sum_{i=1}^{n}(x_i-y_i)^2}+\sqrt{\sum_{i=1}^{n}(y_i-z_i)^2}.
	\end{IEEEeqnarray*}
	It follows that \(\sum_{i=1}^{n}(x_i-z_i)\leq\sum_{i=1}^{n}(x_i-y_i)+\sum_{i=1}^{n}(x_i-z_i)\). Suppose \(x,y,z,w\in\mathbb{R}\). Then
	\begin{IEEEeqnarray*}{rCl}
		\sqrt{x^2+w^2}+\sqrt{y^2+z^2}&\geq&\sqrt{(x+y)^2+(w+z)^2}\\
		x^2+y^2+w^2+z^2+2\sqrt{x^2+w^2}\sqrt{y^2+z^2}&\geq&x^2+y^2+w^2+z^2+2xy+2wz\\
	\end{IEEEeqnarray*}
\end{IEEEproof}
\subsection*{Problem 2}
\begin{IEEEproof}
	For all \((x,y)\in (0,1)\times (0,1)\), \(0<x,y<1\). So, for any \((x,y)\) if \(r=\text{min}\big(\lvert x-1\rvert, x, \lvert y-1\rvert, y\big)\) and \((z,w)\in B((x,y), r)\), then \(\sqrt{(x-z)^2+(y-w)^2}<\frac{r}{2}\). Therefore \(z=x\pm\frac{r}{2}\) and \(w=y\pm\frac{r}{2}\), so \((z,w)\in U\).
\end{IEEEproof}
\medbreak
\begin{IEEEproof}
	Suppose \((x,y)\in \{(0,1)\times (0,1)\}\). Then \(d\big((x,y), (0,0)\big)=\sqrt{x^2+y^2}\). Because there exists \(r>0\) such that \(B(x,r)\subset U\), if \(r>\text{min}\big(\lvert x-1\rvert, x, \lvert y-1\rvert, y\big)\), then \(B(x,r)\not\subset U\). But because \(x,y>0\), we see that \(d\big((x,y), (0,0)\big)>r\), so \(\exists a\in B\bigg((0,0), d\big((x,y), (0,0)\big)-r\bigg)\)\footnote{Not the empty set} such that \(a\notin B((x,y), r)\) for any open ball.
\end{IEEEproof}
\subsection*{Problem 3}
\begin{IEEEproof}
	Let \(X=\mathbb{R}\). If \(U_i\in\{(\frac{-1}{i},\frac{1}{i})\}_{i=1}^{\infty}\) is a sequence of open subsets, then \(\cap_{i=1}^{\infty}U_i={0}\). Let \(x\in U_i\), then \(\forall\epsilon>0,\exists N=\frac{2}{\epsilon}\,(i>N\Rightarrow \lvert x\rvert>\epsilon\Rightarrow x\notin U_N)\). So \(\cap_{i=1}^{\infty}U_i=\{0\}\), and \(\{0\}\) is a closed subset of \(X\).
\end{IEEEproof}
\end{document}
