\documentclass{article}
\usepackage[utf8]{inputenc}
\usepackage[english]{babel}
\usepackage{enumitem}
\usepackage{amsmath}
\usepackage{amssymb}
\usepackage{amsfonts}
\usepackage{mathrsfs}
\usepackage{mathtools}
\usepackage{IEEEtrantools}
\usepackage{geometry}
\geometry{
a4paper,
total={170mm,257mm},
left=20mm,
top=20mm,
}

\renewcommand{\IEEEQED}{\IEEEQEDopen}
\DeclarePairedDelimiter{\ceil}{\lceil}{\rceil}

\title{Homework 2}
\author{Samuel Lindskog}
\date\today
%\clearpage clears page

\begin{document}
\maketitle %This command prints the title based on information entered above

\subsection*{Problem 1}
\begin{enumerate}[label=(\alph*)]
	\item \(A=(1,2)\cup(2,\infty)\).
		\begin{IEEEproof}
		Let \((n,2n)=A_n\) with \(n\in\mathbb{N}\), then \(A_1=(1,2)\). Iff \(x\in\bigcup_{n=2}^{3}A_n\), then 
		\begin{equation*}
			x\in (2,4)\wedge x\in(3,6) \Leftrightarrow 2<x<6\Leftrightarrow x\in (2,6).
		\end{equation*}
			So \(\bigcup_{n=2}^{3}A_n=(2,6)\). By inductive hypothesis, if \(m \in\mathbb{N}\) with \(m\geq 2\) then \(\bigcup_{n=2}^{m}A_n=(2,2m)\). If \(x\in\bigcup_{n=2}^{m+1}A_n\) then
			\begin{equation*}
			x\in\bigcup_{n=2}^{m}A_n\cup (m+1, 2(m+1))\Leftrightarrow(2<x<2m)\vee(m+1<x<2(m+1))
			\Leftrightarrow x\in(2,2(m+1))
		\end{equation*}
			Therefore \(\bigcup_{n=1}^{m+1}A_n=(2,2(m+1))\). Because for all \(\epsilon>0\) there exists \(N=\ceil{\epsilon}\) such that \(2N>\epsilon\), We see that \(\bigcup_{n=2}^{\infty}A_n=(2,\infty)\), so \(A=(1,2)\cup(2,\infty)\).
\end{IEEEproof}
	\item \(C=(-\infty, \infty)\).
		\begin{IEEEproof}
			Let \(C_n=(n,n+2)\). Then \(C_0=(0,2)\), \(C_1=(1,3)\), and \(C_2=(2,4)\), and thus \(\bigcup_{n=0}^{2}C_n=(0,4)\). By inductive hypothesis, if \(m\in\mathbb{N}\) then \(\bigcup_{n=1-m}^{1+m}C_m=(1-m, 3+m)\). Then
			\begin{equation*}
				\bigcup_{n=1-(m+1)}^{1+(m+1)}C_n\Leftrightarrow
				(-m<x<2-m)\vee(2+m<x<4+m)\vee(1-m<x<3+m)\Leftrightarrow x\in(1-(m+1),3+(m+1)).
			\end{equation*}
			Because for all \(\epsilon>0\) there exists \(N=3\ceil{\epsilon}\) such that \(\lvert 3+N\rvert,\lvert 1-N\rvert>\epsilon\), so \(C=(-\infty,\infty)\).
		\end{IEEEproof}
\end{enumerate}

\subsection*{Problem 2}
Define \(f: (1,4)\rightarrow\mathbb{R}\)
\begin{enumerate}[label=(\alph*)]
	\item Is \(f\) one-to-one? No. \(f^{-1}(\{-\frac{1}{2}\})=\{2,3\}\).
	\item What is \(f\big((1,2)\big)\)? \((-\infty, -1/2)\). \(f(x)\) has no critical points on the interval \((1,2)\), so \(f\big((1,2)\big)\) is the interval \((\lim_{x\rightarrow 1^+}f(x),f(2))\).
	\item What is \(f\big((1,3)\big)\)? \((-\infty, -\frac{4}{9})\). \(f(x)\) has one critical point at \(x=2.5\), and because \(lim_{x\rightarrow 1^+}f(x)=-\infty\) and \(f(3=-1/2)\), \(f(2.5)=\frac{4}{9}\) is the relative maximum value on \((1,3)\).
\end{enumerate}
\subsection*{Problem 3}
\begin{enumerate}[label=(\alph*)]
	\item Is \(f\) one-to-one? No. \(f(\frac{4}{3})=f(-\frac{2}{3})\).
	\item Is \(f\) onto? No. The codomain of \(f\) is always positive so there exists \(x\) in the codomain \(\mathbb{Z}\) such that \(x\notin f(\mathbb{Z})\).
\end{enumerate}
\subsection*{Problem 4}
\begin{enumerate}[label=(\alph*)]
	\item The pre-image of \((0,1)\) in \(\mathbb{R}\), \(f^{-1}\big((0,1)\big)=(-\infty, \frac{1}{2})\).
	\item \(f^{-1}\big((1,2)\big)=(\frac{1}{2}, 1)\).
	\item \(f^{-1}\big((-\infty, 2)\big)=(-\infty, 1)\).
\end{enumerate}
\subsection*{Problem 5}
\begin{enumerate}[label=(\alph*)]
	\item The image of \(\{1,2\}\) in \(\mathbb{R}\), \(g\big(\{1,2\}\big)=\{1,2\}\).
	\item \(g\big((1,2)\big)=\{1,\frac{3}{2}\}\).
	\item \(g\big([1,2]\big)=\{1,\frac{3}{2},2\}\).
\end{enumerate}
\end{document}
