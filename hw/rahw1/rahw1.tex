\documentclass{article}
\usepackage[utf8]{inputenc}
\usepackage[english]{babel}
\usepackage{enumitem}
\usepackage{amsmath}
\usepackage{amsfonts}
\usepackage{mathrsfs}
\usepackage{IEEEtrantools}

\renewcommand{\IEEEQED}{\IEEEQEDopen}

\title{Homework 1}
\author{Samuel Lindskog}
\date\today
%\clearpage clears page

\begin{document}
\maketitle %This command prints the title based on information entered above

\subsection*{Problem 1}
\begin{enumerate}[label=(\alph*)]
	\item \(\exists x\in A,\forall y\in B\,(y\neq x^2)\)
	\item \(x\in A\wedge x\notin B\)
	\item \(\forall x\in A\,(x\notin B)\)
\end{enumerate}

\subsection*{Problem 2}
\begin{enumerate}[label=(\alph*)]
	\item \(\forall x,y\in\mathbb{R}(x^2=-y^2)\).
		\begin{IEEEproof}
			False. Counterexample is if \(x,y=2\) then \(2^2=4\) and \\\(-2^2=-4\).
		\end{IEEEproof}
	\item \(\forall x\in\mathbb{N},\exists y\in\mathbb{Z}(y^2=x)\).
		\begin{IEEEproof}
			False. \(2\in\mathbb{N}\), and because \(y=\sqrt{x}\), \(y=\sqrt{2}\), and \(y\notin\mathbb{Z}\).
		\end{IEEEproof}
	\item \(\forall y\in\mathbb{R}\,(3y=0\vee y\neq 0)\).
		\begin{IEEEproof}
			True. If \(y=0\), then \(3y=3\cdot 0=0\). Or \(y\neq 0\).
		\end{IEEEproof}
	\item \(\forall x\in\mathbb{R},\exists y\in\mathbb{R}\,(xy=\pi)\).
		\begin{IEEEproof}
			True. Let \(x\) be arbitrary. If \(x,y\in\mathbb{R}\) and \(xy=\pi\), then \(y=\frac{\pi}{x}\) \\and \(y\in\mathbb{R}\).
		\end{IEEEproof}
	\item \(\forall x\in\mathbb{R}, \exists y\in\mathbb{R}\,(xy=0)\).
		\begin{IEEEproof}
			True. Let \(x\) be arbitrary. If \(x,y\in\mathbb{R}\), then if \(y=0\),\\\(xy=x\cdot 0=0\).
		\end{IEEEproof}
\end{enumerate}

\subsection*{Problem 3}
		\begin{IEEEproof}
			Suppose \(x\in\mathbb{Q}\) and \(\lvert x-3\rvert < 1\). Then \(-1<x-3<1\), so \(2<x<4\). Because \(x\) is always positive \(8<x^3<64\) and \(-4<-x<-2\), so \(4<x^3-x<62\) and \(4\leq x^3-x\leq 62\).
		\end{IEEEproof}

\subsection*{Problem 4}
	\begin{IEEEproof}
		Suppose \(n,a\in\mathbb{Z}\). Case \(n\) is even, then \(n=2a\) for some \(a\in\mathbb{Z}\). Then
		\begin{IEEEeqnarray*}{rCl}
			\frac{1}{2}(n^2-n^3)&=&\frac{1}{2}\big((2a)^2-(2a)^3\big)\\
			&=&\frac{1}{2}(4a^2-8a^3)\\
			&=&2a^2-4a^3
		\end{IEEEeqnarray*}
		and because \(a\) is an integer, \(\frac{1}{2}(n^2-n^3)\) is an integer. Case \(n\) is odd, then \(n=2a+1\) for some \(a\in\mathbb{Z}\). Then
		\begin{IEEEeqnarray*}{rCl}
			\frac{1}{2}(n^2-n^3)&=&\frac{1}{2}\big((2a+1)^2-(2a+1)^3\big)\\
			&=&\frac{1}{2}(4a^2+4a+1+(2a+1)(4a^2+4a+1))\\
			&=&\frac{1}{2}(2a+2)(4a^2+4a+1)\\
			&=&(a+1)(4a^2+4a+1)
		\end{IEEEeqnarray*}
		and because \(a\) is an integer, \(\frac{1}{2}(n^2-n^3)\) is an integer.
	\end{IEEEproof}

\end{document}
