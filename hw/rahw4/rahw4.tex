\documentclass{article}
\usepackage[utf8]{inputenc}
\usepackage[english]{babel}
\usepackage{enumitem}
\usepackage{amsmath}
\usepackage{amssymb}
\usepackage{amsfonts}
\usepackage{mathrsfs}
\usepackage{mathtools}
\usepackage{IEEEtrantools}
\usepackage{geometry}
\geometry{
a4paper,
total={140mm,257mm},
left=35mm,
top=20mm,
}

\renewcommand{\IEEEQED}{\IEEEQEDopen}
\DeclarePairedDelimiter{\ceil}{\lceil}{\rceil}
\DeclarePairedDelimiter{\floor}{\lfloor}{\rfloor}

\title{Homework 4}
\author{Samuel Lindskog}
\date\today
%\clearpage clears page

\begin{document}
\maketitle
\subsection*{Problem 1}
Suppose \(n\in\mathbb{N}\). By definition \(a_1=4\), and \(a_2=9\). By inductive hypothesis if \(n\geq 2\) then \(a_{n}> a_{n-1}\). Suppose to the contrary that \(a_{n+1}\leq a_{n}\). Then
\begin{IEEEeqnarray*}{c}
	3(a_n-1)\leq a_n\\
	a_n\leq\frac{3}{2}
\end{IEEEeqnarray*}
a contradiction because \(a_2=9\). Let \(n,m\in\mathbb{N}\), and suppose to the contrary that \(a_n=a_m\) with \(n\neq m\). But if \(n\neq m\), then wlog \(n<m\) and \(a_n<a_m\), a contradiction because \(<\) is transitive.
\subsection*{Problem 2}
\end{document}
